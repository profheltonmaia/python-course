%% Generated by Sphinx.
\def\sphinxdocclass{jupyterBook}
\documentclass[letterpaper,10pt,english]{jupyterBook}
\ifdefined\pdfpxdimen
   \let\sphinxpxdimen\pdfpxdimen\else\newdimen\sphinxpxdimen
\fi \sphinxpxdimen=.75bp\relax
\ifdefined\pdfimageresolution
    \pdfimageresolution= \numexpr \dimexpr1in\relax/\sphinxpxdimen\relax
\fi
%% let collapsible pdf bookmarks panel have high depth per default
\PassOptionsToPackage{bookmarksdepth=5}{hyperref}
%% turn off hyperref patch of \index as sphinx.xdy xindy module takes care of
%% suitable \hyperpage mark-up, working around hyperref-xindy incompatibility
\PassOptionsToPackage{hyperindex=false}{hyperref}
%% memoir class requires extra handling
\makeatletter\@ifclassloaded{memoir}
{\ifdefined\memhyperindexfalse\memhyperindexfalse\fi}{}\makeatother

\PassOptionsToPackage{warn}{textcomp}

\catcode`^^^^00a0\active\protected\def^^^^00a0{\leavevmode\nobreak\ }
\usepackage{cmap}
\usepackage{fontspec}
\defaultfontfeatures[\rmfamily,\sffamily,\ttfamily]{}
\usepackage{amsmath,amssymb,amstext}
\usepackage{polyglossia}
\setmainlanguage{english}



\setmainfont{FreeSerif}[
  Extension      = .otf,
  UprightFont    = *,
  ItalicFont     = *Italic,
  BoldFont       = *Bold,
  BoldItalicFont = *BoldItalic
]
\setsansfont{FreeSans}[
  Extension      = .otf,
  UprightFont    = *,
  ItalicFont     = *Oblique,
  BoldFont       = *Bold,
  BoldItalicFont = *BoldOblique,
]
\setmonofont{FreeMono}[
  Extension      = .otf,
  UprightFont    = *,
  ItalicFont     = *Oblique,
  BoldFont       = *Bold,
  BoldItalicFont = *BoldOblique,
]



\usepackage[Bjarne]{fncychap}
\usepackage[,numfigreset=1,mathnumfig]{sphinx}

\fvset{fontsize=\small}
\usepackage{geometry}


% Include hyperref last.
\usepackage{hyperref}
% Fix anchor placement for figures with captions.
\usepackage{hypcap}% it must be loaded after hyperref.
% Set up styles of URL: it should be placed after hyperref.
\urlstyle{same}

\addto\captionsenglish{\renewcommand{\contentsname}{Parte 1 - Python Básico}}

\usepackage{sphinxmessages}



        % Start of preamble defined in sphinx-jupyterbook-latex %
         \usepackage[Latin,Greek]{ucharclasses}
        \usepackage{unicode-math}
        % fixing title of the toc
        \addto\captionsenglish{\renewcommand{\contentsname}{Contents}}
        \hypersetup{
            pdfencoding=auto,
            psdextra
        }
        % End of preamble defined in sphinx-jupyterbook-latex %
        

\title{Curso de Python}
\date{Jan 13, 2024}
\release{}
\author{Helton Maia}
\newcommand{\sphinxlogo}{\vbox{}}
\renewcommand{\releasename}{}
\makeindex
\begin{document}

\pagestyle{empty}
\sphinxmaketitle
\pagestyle{plain}
\sphinxtableofcontents
\pagestyle{normal}
\phantomsection\label{\detokenize{intro::doc}}


\sphinxAtStartPar
Bem\sphinxhyphen{}vindo ao curso de Python para Iniciantes

\sphinxAtStartPar
Este livro foi cuidadosamente elaborado para estudantes universitários que desejam adquirir habilidades fundamentais em programação por meio da linguagem Python.

\sphinxAtStartPar
Dividido em 8 capítulos, este guia proporciona uma introdução completa aos conceitos essenciais de Python. No primeiro capítulo, exploraremos o universo da programação de computadores, desde os conceitos básicos até o funcionamento de programas em Python. Você terá a oportunidade de escrever seu primeiro programa e compreender a estrutura fundamental.

\sphinxAtStartPar
No segundo capítulo, mergulharemos nas variáveis e tipos de dados, abordando desde a definição até os tipos compostos como listas, tuplas, dicionários e conjuntos. Os exercícios práticos ajudarão a solidificar o conhecimento adquirido.

\sphinxAtStartPar
O terceiro capítulo é dedicado aos operadores e expressões, explorando operadores aritméticos, lógicos, de comparação e muito mais. Desafios estimulantes estão incluídos para aprimorar suas habilidades.

\sphinxAtStartPar
No quarto capítulo, abordaremos o controle de fluxo por meio de estruturas condicionais e de repetição. Exercícios práticos oferecem a oportunidade de aplicar esses conceitos de forma efetiva.

\sphinxAtStartPar
O capítulo cinco introduz as funções, desde a definição até funções recursivas e integradas. A resolução de exercícios proporcionará uma compreensão sólida desses conceitos essenciais.

\sphinxAtStartPar
O sexto capítulo mergulha no fascinante mundo das strings, explorando operações, métodos e expressões regulares. Exercícios práticos são projetados para consolidar seu conhecimento.

\sphinxAtStartPar
No sétimo capítulo, elevamos o nível ao explorar as bibliotecas NumPy e Matplotlib para manipulação de arrays e visualização de dados. Exercícios práticos aplicam essas poderosas ferramentas.

\sphinxAtStartPar
Finalmente, no oitavo capítulo, concentramo\sphinxhyphen{}nos na manipulação de arquivos, tratamento de exceções e trabalhando com diferentes tipos de arquivos, como csv, json e bin. Exercícios práticos oferecem a chance de aplicar esses conhecimentos de forma concreta.

\sphinxAtStartPar
Este livro é um recurso valioso para qualquer estudante que deseje iniciar e aprimorar seus conhecimentos em Python. A clareza, a concisão e os numerosos exemplos e exercícios tornam o aprendizado eficiente e agradável.
\begin{itemize}
\item {} 
\sphinxAtStartPar
Parte 1 \sphinxhyphen{} Python Básico

\begin{itemize}
\item {} 
\sphinxAtStartPar
{\hyperref[\detokenize{chapters/ch1/ch1::doc}]{\sphinxcrossref{Capítulo 1: Introdução à programação em Python}}}

\item {} 
\sphinxAtStartPar
{\hyperref[\detokenize{chapters/ch2/ch2::doc}]{\sphinxcrossref{Capítulo 2: Váriáveis e Tipos de dados}}}

\item {} 
\sphinxAtStartPar
{\hyperref[\detokenize{chapters/ch3/ch3::doc}]{\sphinxcrossref{Capítulo 3: Operadores e expressões}}}

\item {} 
\sphinxAtStartPar
{\hyperref[\detokenize{chapters/ch4/ch4::doc}]{\sphinxcrossref{Capítulo 4: Controle de Fluxo}}}

\item {} 
\sphinxAtStartPar
{\hyperref[\detokenize{chapters/ch5/ch5::doc}]{\sphinxcrossref{Capítulo 5: Funções}}}

\item {} 
\sphinxAtStartPar
{\hyperref[\detokenize{chapters/ch6/ch6::doc}]{\sphinxcrossref{Capítulo 6: Strings}}}

\item {} 
\sphinxAtStartPar
{\hyperref[\detokenize{chapters/ch7/ch7::doc}]{\sphinxcrossref{Capítulo 7: NumPy e Matplotlib}}}

\item {} 
\sphinxAtStartPar
{\hyperref[\detokenize{chapters/ch8/ch8::doc}]{\sphinxcrossref{Capítulo 8: Manipulação de Arquivos}}}

\end{itemize}
\end{itemize}

\sphinxstepscope


\chapter{Capítulo 1: Introdução à programação em Python}
\label{\detokenize{chapters/ch1/ch1:capitulo-1-introducao-a-programacao-em-python}}\label{\detokenize{chapters/ch1/ch1::doc}}

\section{O que é programação de computadores?}
\label{\detokenize{chapters/ch1/ch1:o-que-e-programacao-de-computadores}}
\sphinxAtStartPar
Programar computadores é a arte e ciência de conceber e criar conjuntos de instruções que capacitam computadores a realizar tarefas específicas. Esse processo envolve a expressão lógica de algoritmos por meio de uma linguagem de programação, atuando como a ponte entre a mente humana e a máquina.

\sphinxAtStartPar
Essa habilidade é fundamental para aqueles que buscam atuar no universo da computação, desempenhando um papel essencial em diversas disciplinas, como engenharia, ciência, negócios, saúde, educação e entretenimento. A capacidade de programar não apenas possibilita a automação de processos, mas também estimula a resolução criativa de problemas e impulsiona a inovação tecnológica.

\sphinxAtStartPar
Na prática da programação, os desenvolvedores convertem conceitos abstratos em linguagem compreensível pelos computadores, proporcionando\sphinxhyphen{}lhes a habilidade de executar tarefas complexas. Essa interação entre humanos e máquinas desempenha um papel fundamental na contínua evolução da sociedade digital, moldando desde avanços científicos até transformações sociais significativas.

\sphinxAtStartPar
A habilidade de programar transcende a mera condição técnica, transformando\sphinxhyphen{}se em uma ferramenta importante para explorar novas ideias e o aprimoramento pessoal. Filosoficamente falando, programar é também uma forma de enxergar o mundo sob diferentes perspectivas, processos e abstrações.

\sphinxAtStartPar
Dentro do contexto do Python, esta é uma linguagem considerada de alto nível, sendo interpretada e multiparadigma. Tal caracterização implica que o Python destaca\sphinxhyphen{}se por sua facilidade de aprendizado e uso, sendo aplicável a uma ampla gama de propósitos. Sua versatilidade se destaca ainda mais pela capacidade de suportar diversos paradigmas de programação, proporcionando aos desenvolvedores uma abordagem flexível e adaptável para resolver problemas em diferentes domínios.

\sphinxAtStartPar
O Python é uma linguagem de programação versátil que desempenha papéis significativos em diversas áreas. A seguir, um breve resumo de algumas dessas possibilidades:
\begin{itemize}
\item {} 
\sphinxAtStartPar
\sphinxstylestrong{Engenharia e Ciências:} Utilizado em simulação, análise e visualização de dados, assim como em projetos de aprendizado de máquina. Sua sintaxe clara e concisa, bem como sua ampla biblioteca de módulos científicos, tornam\sphinxhyphen{}o uma escolha popular para tais abordagens.

\item {} 
\sphinxAtStartPar
\sphinxstylestrong{Negócios:} Como uma ferramenta essencial para a análise de dados, automação de processos e desenvolvimento de aplicativos web. Sua flexibilidade e eficiência o tornam uma escolha versátil para soluções empresariais.

\item {} 
\sphinxAtStartPar
\sphinxstylestrong{Educação:} O Python é a linguagem de programação mais popular para o ensino de programação em escolas e universidades. Sua sintaxe simples e intuitiva torna\sphinxhyphen{}o uma linguagem fácil de aprender, mesmo para iniciantes.

\item {} 
\sphinxAtStartPar
\sphinxstylestrong{Entretenimento:} É uma linguagem de programação versátil e robusta que é frequentemente utilizada na criação de jogos, aplicativos móveis e outros softwares de entretenimento. Sua flexibilidade e robustez permitem que seja usada para criar aplicações de alta qualidade e de todos os tipos.

\item {} 
\sphinxAtStartPar
\sphinxstylestrong{Saúde:} Fortemente utilizado em análise de dados médicos e desenvolvimento de softwares especializados e pesquisa médica. Sua capacidade analítica e adaptabilidade o tornam uma ferramenta valiosa para a pesquisa e a inovação em saúde.

\end{itemize}

\sphinxAtStartPar
Python é uma linguagem de programação versátil que atende a uma ampla gama de necessidades, desde automatização de tarefas repetitivas até desafios avançados. Suas aplicações incluem automação, análise de dados, desenvolvimento de jogos, inteligência artificial, automação de redes, aplicações científicas, desenvolvimento de aplicativos de desktop e web, construção de APIs, segurança cibernética, simulações científicas e matemáticas, Internet das Coisas (IoT) e produção de mídia. A versatilidade do Python o torna uma ferramenta indispensável em diversas áreas, proporcionando uma base sólida para inovação no cenário tecnológico atual. Dominar Python não é apenas uma habilidade essencial, mas também uma maneira de explorar as constantes inovações e desafios tecnológicos em evolução.




\section{O que você precisa para começar?}
\label{\detokenize{chapters/ch1/ch1:o-que-voce-precisa-para-comecar}}
\sphinxAtStartPar
Para iniciar seu aprendizado em Python, além desta documentação, é fundamental contar com os seguintes elementos:
\begin{enumerate}
\sphinxsetlistlabels{\arabic}{enumi}{enumii}{}{.}%
\item {} 
\sphinxAtStartPar
\sphinxstylestrong{Computador com Acesso à Internet:} Recomenda\sphinxhyphen{}se utilizar um computador com conexão à internet para facilitar o download de pacotes adicionais e o acesso à documentação online, enriquecendo sua experiência de aprendizado. No entanto, é válido destacar que é possível programar em Python em um ambiente offline, o que se torna uma opção viável em situações em que a conexão à internet não está disponível.

\item {} 
\sphinxAtStartPar
\sphinxstylestrong{Editor de Texto ou IDE (Ambiente de Desenvolvimento Integrado):} Escolha um editor de texto que atenda às suas preferências e necessidades. Pode ser um editor simples como o Notepad ou algo mais avançado como o Sublime Text, Atom, Visual Studio Code, ou editores online como o \sphinxhref{https://replit.com}{Replit}, Google Colab ou o Jupyter Notebook. Além disso, o PyCharm é uma poderosa IDE específica para Python que oferece recursos avançados e é amplamente utilizada por desenvolvedores.

\item {} 
\sphinxAtStartPar
\sphinxstylestrong{Interpretador Python:} Faça o download do interpretador Python diretamente do site da Python Software Foundation (\sphinxurl{https://www.python.org/}). Alternativamente, você pode utilizar ambientes online como o Replit, que já incluem um interpretador Python e um editor de texto integrados.

\end{enumerate}

\sphinxAtStartPar
Equipado com esses recursos, você estará pronto para explorar e aprimorar suas habilidades em Python. Seja trabalhando localmente em seu computador ou em ambientes online, você terá a flexibilidade necessária para mergulhar no mundo da programação, adaptando\sphinxhyphen{}se ao seu estilo de aprendizado preferido.

\sphinxAtStartPar
\sphinxstylestrong{Observação:} O Python é uma linguagem de programação interpretada, ou seja, seu código é executado diretamente pelo interpretador, sem ser convertido para um formato de código de máquina. Isso torna o Python uma linguagem mais fácil de aprender e usar, pois não é necessário compilar o código antes de executá\sphinxhyphen{}lo.


\section{Escrevendo seu primeiro programa em python}
\label{\detokenize{chapters/ch1/ch1:escrevendo-seu-primeiro-programa-em-python}}
\sphinxAtStartPar
Neste material introdutório, apresentamos um exemplo conciso de um programa em Python que realiza a soma de dois números e exibe o resultado na tela. O arquivo pode ser nomeado “\sphinxhref{http://soma.py}{soma.py}”, e o código é o seguinte:

\begin{sphinxVerbatim}[commandchars=\\\{\}]
\PYG{n}{a} \PYG{o}{=} \PYG{l+m+mi}{1}
\PYG{n}{b} \PYG{o}{=} \PYG{l+m+mi}{2}
\PYG{n}{soma} \PYG{o}{=} \PYG{n}{a} \PYG{o}{+} \PYG{n}{b}
\PYG{n+nb}{print}\PYG{p}{(}\PYG{n}{soma}\PYG{p}{)}
\end{sphinxVerbatim}

\begin{sphinxVerbatim}[commandchars=\\\{\}]
\PYG{l+m+mi}{3}
\end{sphinxVerbatim}

\sphinxAtStartPar
Neste exemplo didático, valores são atribuídos às variáveis \sphinxcode{\sphinxupquote{a}} e \sphinxcode{\sphinxupquote{b}}. Posteriormente, uma nova variável chamada \sphinxcode{\sphinxupquote{soma}} armazena o resultado da adição desses valores. Finalmente, o comando \sphinxcode{\sphinxupquote{print}} é utilizado para exibir o resultado da soma.

\sphinxAtStartPar
Aqui está a explicação de cada linha do código:
\begin{itemize}
\item {} 
\sphinxAtStartPar
\sphinxstylestrong{\sphinxcode{\sphinxupquote{a = 1}}}: Atribui o valor inteiro 1 à variável \sphinxcode{\sphinxupquote{a}}. Variáveis armazenam dados em Python.

\item {} 
\sphinxAtStartPar
\sphinxstylestrong{\sphinxcode{\sphinxupquote{b = 2}}}: Atribui o valor 2 à variável \sphinxcode{\sphinxupquote{b}}. Agora, \sphinxcode{\sphinxupquote{a}} contém 1 e \sphinxcode{\sphinxupquote{b}} contém 2.

\item {} 
\sphinxAtStartPar
\sphinxstylestrong{\sphinxcode{\sphinxupquote{soma = a + b}}}: Cria a variável \sphinxcode{\sphinxupquote{soma}} e atribui a ela a soma dos valores em \sphinxcode{\sphinxupquote{a}} e \sphinxcode{\sphinxupquote{b}}, resultando em 3.

\item {} 
\sphinxAtStartPar
\sphinxstylestrong{\sphinxcode{\sphinxupquote{print(soma)}}}: Utiliza a função \sphinxcode{\sphinxupquote{print}} para exibir o valor armazenado em \sphinxcode{\sphinxupquote{soma}}, que é 3.

\end{itemize}

\sphinxAtStartPar
Resumidamente, o programa define dois valores (1 e 2) em duas variáveis (\sphinxcode{\sphinxupquote{a}} e \sphinxcode{\sphinxupquote{b}}), realiza a soma desses valores e armazena o resultado em uma terceira variável (\sphinxcode{\sphinxupquote{soma}}), e finalmente, imprime o resultado (3) na tela.

\sphinxAtStartPar
\sphinxstylestrong{O que são variáveis?}

\sphinxAtStartPar
São identificadores nomeados que armazenam e representam dados, essenciais para a manipulação dinâmica de informações em um programa.

\sphinxAtStartPar
\sphinxstylestrong{O que são funções?}

\sphinxAtStartPar
São blocos de código reutilizáveis que realizam tarefas específicas. A função \sphinxcode{\sphinxupquote{print}} foi usada para exibir informações no console ou terminal.

\sphinxAtStartPar
Para executar o código Python:
\begin{enumerate}
\sphinxsetlistlabels{\arabic}{enumi}{enumii}{}{.}%
\item {} 
\sphinxAtStartPar
Escreva o código em um editor de texto.

\item {} 
\sphinxAtStartPar
Salve o arquivo como “.py”.

\item {} 
\sphinxAtStartPar
Abra um terminal ou prompt de comando.

\item {} 
\sphinxAtStartPar
Navegue até o diretório do arquivo.

\item {} 
\sphinxAtStartPar
Execute com \sphinxcode{\sphinxupquote{python nome\_do\_arquivo.py}}.

\item {} 
\sphinxAtStartPar
Observe a saída no terminal.

\end{enumerate}

\begin{sphinxVerbatim}[commandchars=\\\{\}]
python\PYG{+w}{ }soma.py
\end{sphinxVerbatim}

\sphinxAtStartPar
A saída será o número 3, resultado da soma de 1 e 2.


\section{Como um programa em Python funciona?}
\label{\detokenize{chapters/ch1/ch1:como-um-programa-em-python-funciona}}
\sphinxAtStartPar
O processo de execução de um código Python no computador é composto por várias etapas, cada uma desempenhando um papel fundamental. Tudo começa com o desenvolvimento do código fonte, usualmente armazenado em arquivos com extensão “.py”. Esse código é então submetido ao interpretador Python.

\sphinxAtStartPar
O interpretador é o programa encarregado de ler e processar o código fonte. Ele converte o código Python em código objeto (bytecode), uma forma intermediária que é independente da arquitetura do hardware. Este bytecode é uma representação de baixo nível que servirá como entrada para a Máquina Virtual Python (PVM).

\sphinxAtStartPar
A PVM é a camada que efetivamente executa o programa. Ela gerencia a execução do bytecode, cuida do gerenciamento de memória e interage com o sistema operacional. Além disso, algumas implementações do interpretador Python, como o CPython, podem incorporar um Compilador Just\sphinxhyphen{}In\sphinxhyphen{}Time (JIT).




\section{Estrutura básica de um programa em Python}
\label{\detokenize{chapters/ch1/ch1:estrutura-basica-de-um-programa-em-python}}
\sphinxAtStartPar
Um programa em Python segue uma estrutura básica, um algoritmo, que é uma sequência de passos definidos para realizar uma tarefa. Similar a uma receita culinária, o algoritmo tem entrada (dados), processamento (passos a seguir), e saída (resultado).

\sphinxAtStartPar
Conceitualmente, a estrutura fundamental de um programa Python é:

\begin{sphinxVerbatim}[commandchars=\\\{\}]
\PYG{k}{def} \PYG{n+nf}{main}\PYG{p}{(}\PYG{p}{)}\PYG{p}{:}
    \PYG{c+c1}{\PYGZsh{} Bloco de código principal}

\PYG{k}{if} \PYG{n+nv+vm}{\PYGZus{}\PYGZus{}name\PYGZus{}\PYGZus{}} \PYG{o}{==} \PYG{l+s+s2}{\PYGZdq{}}\PYG{l+s+s2}{\PYGZus{}\PYGZus{}main\PYGZus{}\PYGZus{}}\PYG{l+s+s2}{\PYGZdq{}}\PYG{p}{:}
    \PYG{n}{main}\PYG{p}{(}\PYG{p}{)}
\end{sphinxVerbatim}
\begin{itemize}
\item {} 
\sphinxAtStartPar
\sphinxcode{\sphinxupquote{main()}}: Função principal do programa, chamada quando o programa é executado.

\item {} 
\sphinxAtStartPar
Bloco de código principal: Onde a execução do programa ocorre, identado para indicar que faz parte do bloco principal.

\item {} 
\sphinxAtStartPar
\sphinxcode{\sphinxupquote{if \_\_name\_\_ == "\_\_main\_\_":}}: Verifica se o arquivo está sendo executado como um programa principal, garantindo que a função \sphinxcode{\sphinxupquote{main()}} seja executada.

\end{itemize}

\sphinxAtStartPar
Exemplo de um programa básico:

\begin{sphinxVerbatim}[commandchars=\\\{\}]
\PYG{k}{def} \PYG{n+nf}{main}\PYG{p}{(}\PYG{p}{)}\PYG{p}{:}
    \PYG{n+nb}{print}\PYG{p}{(}\PYG{l+s+s2}{\PYGZdq{}}\PYG{l+s+s2}{Hello, world!}\PYG{l+s+s2}{\PYGZdq{}}\PYG{p}{)}

\PYG{k}{if} \PYG{n+nv+vm}{\PYGZus{}\PYGZus{}name\PYGZus{}\PYGZus{}} \PYG{o}{==} \PYG{l+s+s2}{\PYGZdq{}}\PYG{l+s+s2}{\PYGZus{}\PYGZus{}main\PYGZus{}\PYGZus{}}\PYG{l+s+s2}{\PYGZdq{}}\PYG{p}{:}
    \PYG{n}{main}\PYG{p}{(}\PYG{p}{)}
\end{sphinxVerbatim}

\sphinxAtStartPar
Para executar, salve o código em um arquivo “.py”, como “hello\_world.py”, e no terminal, execute:

\begin{sphinxVerbatim}[commandchars=\\\{\}]
\PYG{n}{python} \PYG{n}{hello\PYGZus{}world}\PYG{o}{.}\PYG{n}{py}
\end{sphinxVerbatim}

\sphinxAtStartPar
Lembre\sphinxhyphen{}se da importância da formatação adequada (indentação) para indicar a estrutura do programa. A consistência, preferencialmente utilizando quatro espaços por nível, é recomendada pela PEP 8. Essa prática aprimora a legibilidade e mantém um estilo uniforme.


\section{Exemplo aprimorado: capturando nome e idade}
\label{\detokenize{chapters/ch1/ch1:exemplo-aprimorado-capturando-nome-e-idade}}
\sphinxAtStartPar
Vamos aprimorar nosso código inicial para aprender a interagir com o usuário. Neste exemplo, criaremos um programa que solicita e armazena o nome e a idade do usuário, para então exibir essas informações. Siga os passos abaixo:
\begin{enumerate}
\sphinxsetlistlabels{\arabic}{enumi}{enumii}{}{.}%
\item {} 
\sphinxAtStartPar
Crie um novo arquivo para armazenar o código do programa.

\item {} 
\sphinxAtStartPar
Copie ou digite o seguinte código no arquivo:

\begin{sphinxVerbatim}[commandchars=\\\{\}]
\PYG{n}{nome} \PYG{o}{=} \PYG{n+nb}{input}\PYG{p}{(}\PYG{l+s+s2}{\PYGZdq{}}\PYG{l+s+s2}{Qual é o seu nome? }\PYG{l+s+s2}{\PYGZdq{}}\PYG{p}{)}
\PYG{n}{idade} \PYG{o}{=} \PYG{n+nb}{input}\PYG{p}{(}\PYG{l+s+s2}{\PYGZdq{}}\PYG{l+s+s2}{Qual é a sua idade? }\PYG{l+s+s2}{\PYGZdq{}}\PYG{p}{)}
\PYG{n+nb}{print}\PYG{p}{(}\PYG{n}{nome}\PYG{p}{)}
\PYG{n+nb}{print}\PYG{p}{(}\PYG{n}{idade}\PYG{p}{)}
\end{sphinxVerbatim}

\sphinxAtStartPar
O código utiliza a função \sphinxcode{\sphinxupquote{input}} para solicitar que o usuário insira seu nome e idade. Os dados fornecidos são armazenados nas variáveis \sphinxcode{\sphinxupquote{nome}} e \sphinxcode{\sphinxupquote{idade}}, respectivamente. Em seguida, o programa imprime essas informações na tela.

\item {} 
\sphinxAtStartPar
Salve o arquivo e execute\sphinxhyphen{}o usando o interpretador Python.

\item {} 
\sphinxAtStartPar
O programa solicitará que você insira o nome e a idade.

\begin{sphinxVerbatim}[commandchars=\\\{\}]
Qual é o seu nome? Ana Maria
Qual é a sua idade? 25
\end{sphinxVerbatim}

\item {} 
\sphinxAtStartPar
Após fornecer as informações, o programa imprimirá o nome e a idade na tela.

\begin{sphinxVerbatim}[commandchars=\\\{\}]
\PYG{n}{Ana} \PYG{n}{Maria}
\PYG{l+m+mi}{25}
\end{sphinxVerbatim}

\end{enumerate}

\sphinxAtStartPar
Este exemplo ilustra o uso da função \sphinxcode{\sphinxupquote{input}} para interagir com o usuário, capturando dados e exibindo as informações posteriormente. A utilização de \sphinxcode{\sphinxupquote{input}} é fundamental para criar programas interativos e dinâmicos.


\section{Estratégias cientificamente fundamentadas para estudar programação}
\label{\detokenize{chapters/ch1/ch1:estrategias-cientificamente-fundamentadas-para-estudar-programacao}}
\sphinxAtStartPar
Estudar programação é uma tarefa desafiadora, mas também gratificante. Com uma abordagem estruturada e focada, é possível aprender os conceitos básicos e avançar rapidamente na carreira de desenvolvimento de software. Aqui estão algumas estratégias respaldadas por princípios de aprendizagem e cognição que podem ajudá\sphinxhyphen{}lo a aprender programação de forma mais eficaz:

\sphinxAtStartPar
\sphinxstylestrong{Repetição espaçada:}
A repetição espaçada é uma técnica de aprendizado que envolve revisar conceitos em intervalos crescentes, fortalecendo a retenção a longo prazo. Utilize programas de flashcards ou aplicativos de repetição espaçada para implementar essa técnica.

\sphinxAtStartPar
\sphinxstylestrong{Interleaving:}
O interleaving é uma técnica que envolve misturar diferentes tópicos durante o estudo. Isso evita a dependência excessiva de um único conceito e promove a aplicação flexível de conhecimentos. Alternar entre tópicos relacionados durante as sessões de estudo é uma prática eficaz.

\sphinxAtStartPar
\sphinxstylestrong{Prática ativa:}
Engaje\sphinxhyphen{}se em práticas ativas, como resolver problemas e desenvolver projetos. Essa abordagem estimula a aplicação prática do conhecimento, consolidando a compreensão e fortalecendo as habilidades práticas.

\sphinxAtStartPar
\sphinxstylestrong{Compreensão profunda:}
Foque na compreensão profunda dos conceitos, indo além da memorização para entender os princípios subjacentes. Concentre\sphinxhyphen{}se no “porquê” de algo funcionar, não apenas no “como”.

\sphinxAtStartPar
\sphinxstylestrong{Foco e mínima interrupção:}
Mantenha sessões de estudo focadas, minimizando interrupções para preservar a profundidade da concentração. Escolha ambientes tranquilos e utilize técnicas de meditação para melhorar a concentração.

\sphinxAtStartPar
\sphinxstylestrong{Aprendizagem baseada em problemas:}
Aborde a programação como resolução de problemas, tornando a aprendizagem mais contextual e eficaz. A resolução prática de desafios promove a aplicação real dos conceitos.

\sphinxAtStartPar
\sphinxstylestrong{Recursos diversificados:}
Utilize diversos recursos de aprendizagem, como livros, vídeos, tutoriais interativos e cursos online. Essa variedade enriquece a compreensão dos tópicos.

\sphinxAtStartPar
\sphinxstylestrong{Espaço para reflexão:}
Reserve tempo após o estudo para reflexão, consolidando o conhecimento e aplicando\sphinxhyphen{}o a novas situações. Escrever um diário de aprendizado ou discutir com um mentor pode ser útil.

\sphinxAtStartPar
\sphinxstylestrong{Revisão regular:}
Mantenha o conhecimento fresco na memória por meio de revisões regulares. Crie um cronograma ou utilize aplicativos de revisão para identificar lacunas na compreensão.

\sphinxAtStartPar
\sphinxstylestrong{Aprendizagem colaborativa:}
Participe de grupos de estudo, fóruns online ou projetos colaborativos. A aprendizagem colaborativa proporciona feedback valioso e a oportunidade de compartilhar ideias.

\sphinxAtStartPar
Ao seguir essas estratégias, você pode aumentar suas chances de sucesso no aprendizado de programação e desenvolver uma base sólida de habilidades.

\sphinxstepscope


\chapter{Capítulo 2: Váriáveis e Tipos de dados}
\label{\detokenize{chapters/ch2/ch2:capitulo-2-variaveis-e-tipos-de-dados}}\label{\detokenize{chapters/ch2/ch2::doc}}

\section{Definindo Variáveis e Tipos de Dados}
\label{\detokenize{chapters/ch2/ch2:definindo-variaveis-e-tipos-de-dados}}
\sphinxAtStartPar
Uma variável é um identificador que representa um valor. Elas são utilizadas para armazenar informações como nomes, números ou textos. A declaração de uma variável é feita usando o operador de atribuição (\sphinxcode{\sphinxupquote{=}}). Por exemplo, a linha a seguir cria uma variável chamada \sphinxcode{\sphinxupquote{nome}} e atribui o valor “Maria” a ela:

\begin{sphinxVerbatim}[commandchars=\\\{\}]
\PYG{n}{nome} \PYG{o}{=} \PYG{l+s+s2}{\PYGZdq{}}\PYG{l+s+s2}{Maria}\PYG{l+s+s2}{\PYGZdq{}}
\end{sphinxVerbatim}

\sphinxAtStartPar
Para acessar o valor armazenado em uma variável, basta utilizar o seu nome. O código a seguir, por exemplo, imprime o valor da variável \sphinxcode{\sphinxupquote{nome}}:

\begin{sphinxVerbatim}[commandchars=\\\{\}]
\PYG{n+nb}{print}\PYG{p}{(}\PYG{n}{nome}\PYG{p}{)}
\end{sphinxVerbatim}

\sphinxAtStartPar
Isso resultará na saída:

\begin{sphinxVerbatim}[commandchars=\\\{\}]
\PYG{n}{Maria}
\end{sphinxVerbatim}


\section{Tipos de dados Básicos}
\label{\detokenize{chapters/ch2/ch2:tipos-de-dados-basicos}}

\subsection{Números Inteiros}
\label{\detokenize{chapters/ch2/ch2:numeros-inteiros}}
\sphinxAtStartPar
O tipo inteiro é utilizado para representar números inteiros, tanto positivos quanto negativos, como 1, 2, 3, \sphinxhyphen{}1, \sphinxhyphen{}2, \sphinxhyphen{}3, entre outros. Esses valores são empregados para expressar quantidades e índices em programação, desempenhando um papel fundamental em diversas aplicações.

\sphinxAtStartPar
Exemplos de uso de inteiros incluem:

\begin{sphinxuseclass}{cell}\begin{sphinxVerbatimInput}

\begin{sphinxuseclass}{cell_input}
\begin{sphinxVerbatim}[commandchars=\\\{\}]
\PYG{c+c1}{\PYGZsh{} Declaração de uma variável do tipo inteiro}
\PYG{n}{x} \PYG{o}{=} \PYG{l+m+mi}{10}

\PYG{c+c1}{\PYGZsh{} Atribuição de um valor a uma variável do tipo inteiro}
\PYG{n}{x} \PYG{o}{=} \PYG{l+m+mi}{20}

\PYG{c+c1}{\PYGZsh{} Soma de dois inteiros}
\PYG{n+nb}{print}\PYG{p}{(}\PYG{n}{x} \PYG{o}{+} \PYG{l+m+mi}{10}\PYG{p}{)}
\end{sphinxVerbatim}

\end{sphinxuseclass}\end{sphinxVerbatimInput}
\begin{sphinxVerbatimOutput}

\begin{sphinxuseclass}{cell_output}
\begin{sphinxVerbatim}[commandchars=\\\{\}]
30
\end{sphinxVerbatim}

\end{sphinxuseclass}\end{sphinxVerbatimOutput}

\end{sphinxuseclass}

\subsection{Números de Ponto Flutuante}
\label{\detokenize{chapters/ch2/ch2:numeros-de-ponto-flutuante}}
\sphinxAtStartPar
O tipo de dado ponto flutuante, representado pelo tipo float em Python, é utilizado para expressar valores decimais, como 1.0, 2.5, 3.14, \sphinxhyphen{}1.0, \sphinxhyphen{}2.5, \sphinxhyphen{}3.14, entre outros. Esses valores são apropriados para representar grandezas que envolvem medidas, preços, e outras grandezas fracionárias.

\sphinxAtStartPar
Aqui estão alguns exemplos de uso de números de ponto flutuante:

\begin{sphinxuseclass}{cell}\begin{sphinxVerbatimInput}

\begin{sphinxuseclass}{cell_input}
\begin{sphinxVerbatim}[commandchars=\\\{\}]
\PYG{c+c1}{\PYGZsh{} Declaração de uma variável do tipo float}
\PYG{n}{y} \PYG{o}{=} \PYG{l+m+mf}{3.14}

\PYG{c+c1}{\PYGZsh{} Atribuição de um valor a uma variável do tipo float}
\PYG{n}{y} \PYG{o}{=} \PYG{l+m+mf}{2.5}

\PYG{c+c1}{\PYGZsh{} Subtração de dois floats}
\PYG{n+nb}{print}\PYG{p}{(}\PYG{n}{y} \PYG{o}{\PYGZhy{}} \PYG{l+m+mf}{1.0}\PYG{p}{)}
\end{sphinxVerbatim}

\end{sphinxuseclass}\end{sphinxVerbatimInput}
\begin{sphinxVerbatimOutput}

\begin{sphinxuseclass}{cell_output}
\begin{sphinxVerbatim}[commandchars=\\\{\}]
1.5
\end{sphinxVerbatim}

\end{sphinxuseclass}\end{sphinxVerbatimOutput}

\end{sphinxuseclass}

\subsection{Strings}
\label{\detokenize{chapters/ch2/ch2:strings}}
\sphinxAtStartPar
O tipo de dado string epresenta sequências de caracteres, como “Olá, pessoal!”, “123”, “abc”, etc. Essas estruturas são comumente utilizadas para representar texto e nomes.

\sphinxAtStartPar
Aqui estão alguns exemplos de uso de strings:

\begin{sphinxuseclass}{cell}\begin{sphinxVerbatimInput}

\begin{sphinxuseclass}{cell_input}
\begin{sphinxVerbatim}[commandchars=\\\{\}]
\PYG{c+c1}{\PYGZsh{} Declaração de uma variável do tipo string}
\PYG{n}{z} \PYG{o}{=} \PYG{l+s+s2}{\PYGZdq{}}\PYG{l+s+s2}{Olá, pessoal!}\PYG{l+s+s2}{\PYGZdq{}}

\PYG{c+c1}{\PYGZsh{} Atribuição de um valor a uma variável do tipo string}
\PYG{n}{z} \PYG{o}{=} \PYG{l+s+s2}{\PYGZdq{}}\PYG{l+s+s2}{123}\PYG{l+s+s2}{\PYGZdq{}}

\PYG{c+c1}{\PYGZsh{} Concatenação de duas strings}
\PYG{n+nb}{print}\PYG{p}{(}\PYG{n}{z} \PYG{o}{+} \PYG{l+s+s2}{\PYGZdq{}}\PYG{l+s+s2}{, como vai?}\PYG{l+s+s2}{\PYGZdq{}}\PYG{p}{)}
\end{sphinxVerbatim}

\end{sphinxuseclass}\end{sphinxVerbatimInput}
\begin{sphinxVerbatimOutput}

\begin{sphinxuseclass}{cell_output}
\begin{sphinxVerbatim}[commandchars=\\\{\}]
123, como vai?
\end{sphinxVerbatim}

\end{sphinxuseclass}\end{sphinxVerbatimOutput}

\end{sphinxuseclass}
\sphinxAtStartPar
Em Python, tanto as aspas simples (\sphinxcode{\sphinxupquote{'}}) quanto as aspas duplas (\sphinxcode{\sphinxupquote{"}}) podem ser usadas para definir literais de strings. Não há diferença funcional entre elas; você pode usar qualquer uma delas de forma intercambiável para criar strings. A escolha entre as aspas simples e duplas é principalmente uma questão de estilo e preferência pessoal. Por exemplo:

\begin{sphinxVerbatim}[commandchars=\\\{\}]
\PYG{n}{aspas\PYGZus{}simples} \PYG{o}{=} \PYG{l+s+s1}{\PYGZsq{}}\PYG{l+s+s1}{Esta é uma string com aspas simples.}\PYG{l+s+s1}{\PYGZsq{}}
\PYG{n}{aspas\PYGZus{}duplas} \PYG{o}{=} \PYG{l+s+s2}{\PYGZdq{}}\PYG{l+s+s2}{Esta é uma string com aspas duplas.}\PYG{l+s+s2}{\PYGZdq{}}
\end{sphinxVerbatim}

\sphinxAtStartPar
Você pode usar um tipo de aspas para definir uma string que contenha o outro tipo de aspas sem problemas. Por exemplo:

\begin{sphinxVerbatim}[commandchars=\\\{\}]
\PYG{n}{com\PYGZus{}outras\PYGZus{}aspas} \PYG{o}{=} \PYG{l+s+s2}{\PYGZdq{}}\PYG{l+s+s2}{Esta string contém }\PYG{l+s+s2}{\PYGZsq{}}\PYG{l+s+s2}{aspas simples}\PYG{l+s+s2}{\PYGZsq{}}\PYG{l+s+s2}{ dentro dela.}\PYG{l+s+s2}{\PYGZdq{}}
\end{sphinxVerbatim}

\sphinxAtStartPar
Ou:

\begin{sphinxVerbatim}[commandchars=\\\{\}]
\PYG{n}{com\PYGZus{}outras\PYGZus{}aspas} \PYG{o}{=} \PYG{l+s+s1}{\PYGZsq{}}\PYG{l+s+s1}{Esta string contém }\PYG{l+s+s1}{\PYGZdq{}}\PYG{l+s+s1}{aspas duplas}\PYG{l+s+s1}{\PYGZdq{}}\PYG{l+s+s1}{ dentro dela.}\PYG{l+s+s1}{\PYGZsq{}}
\end{sphinxVerbatim}

\sphinxAtStartPar
Ambos os exemplos são válidos, e o Python permite que você use aspas simples dentro de uma string delimitada por aspas duplas e vice\sphinxhyphen{}versa.


\subsection{Booleanos}
\label{\detokenize{chapters/ch2/ch2:booleanos}}
\sphinxAtStartPar
Os tipos booleanos em Python representam valores lógicos, ou seja, podem ser True ou False. Esses valores são comumente utilizados para representar condições, resultados de testes e expressões lógicas.

\sphinxAtStartPar
Aqui estão alguns exemplos de uso de booleanos:

\begin{sphinxuseclass}{cell}\begin{sphinxVerbatimInput}

\begin{sphinxuseclass}{cell_input}
\begin{sphinxVerbatim}[commandchars=\\\{\}]
\PYG{c+c1}{\PYGZsh{} Declaração de uma variável do tipo booleano}
\PYG{n}{a} \PYG{o}{=} \PYG{k+kc}{True}

\PYG{c+c1}{\PYGZsh{} Atribuição de um valor a uma variável do tipo booleano}
\PYG{n}{a} \PYG{o}{=} \PYG{k+kc}{False}

\PYG{c+c1}{\PYGZsh{} Comparação de dois valores}
\PYG{n+nb}{print}\PYG{p}{(}\PYG{l+m+mi}{10} \PYG{o}{\PYGZgt{}} \PYG{l+m+mi}{20}\PYG{p}{)}
\end{sphinxVerbatim}

\end{sphinxuseclass}\end{sphinxVerbatimInput}
\begin{sphinxVerbatimOutput}

\begin{sphinxuseclass}{cell_output}
\begin{sphinxVerbatim}[commandchars=\\\{\}]
False
\end{sphinxVerbatim}

\end{sphinxuseclass}\end{sphinxVerbatimOutput}

\end{sphinxuseclass}

\section{Tipos de dados compostos}
\label{\detokenize{chapters/ch2/ch2:tipos-de-dados-compostos}}

\subsection{Listas}
\label{\detokenize{chapters/ch2/ch2:listas}}
\sphinxAtStartPar
As Listas são um tipo de dados mutável, o que significa que podem ser alteradas após serem criadas. Elas são declaradas utilizando colchetes, e os valores são separados por vírgulas.
Listas podem conter valores de qualquer tipo, incluindo inteiros, números de ponto flutuante, strings, booleanos e outros tipos de dados.

\sphinxAtStartPar
Aqui estão alguns exemplos de uso de listas:

\begin{sphinxVerbatim}[commandchars=\\\{\}]
\PYG{c+c1}{\PYGZsh{} Declaração de uma lista com três valores inteiros}
\PYG{n}{lista1} \PYG{o}{=} \PYG{p}{[}\PYG{l+m+mi}{1}\PYG{p}{,} \PYG{l+m+mi}{2}\PYG{p}{,} \PYG{l+m+mi}{3}\PYG{p}{]}

\PYG{c+c1}{\PYGZsh{} Declaração de uma lista com três valores reais}
\PYG{n}{lista2} \PYG{o}{=} \PYG{p}{[}\PYG{l+m+mf}{1.0}\PYG{p}{,} \PYG{l+m+mf}{2.5}\PYG{p}{,} \PYG{l+m+mf}{3.14}\PYG{p}{]}

\PYG{c+c1}{\PYGZsh{} Declaração de uma lista com três valores strings}
\PYG{n}{lista3} \PYG{o}{=} \PYG{p}{[}\PYG{l+s+s2}{\PYGZdq{}}\PYG{l+s+s2}{Estou}\PYG{l+s+s2}{\PYGZdq{}}\PYG{p}{,} \PYG{l+s+s2}{\PYGZdq{}}\PYG{l+s+s2}{programando}\PYG{l+s+s2}{\PYGZdq{}}\PYG{p}{,} \PYG{l+s+s2}{\PYGZdq{}}\PYG{l+s+s2}{!}\PYG{l+s+s2}{\PYGZdq{}}\PYG{p}{]}

\PYG{c+c1}{\PYGZsh{} Declaração de uma lista com três valores booleanos}
\PYG{n}{lista4} \PYG{o}{=} \PYG{p}{[}\PYG{k+kc}{True}\PYG{p}{,} \PYG{k+kc}{False}\PYG{p}{,} \PYG{k+kc}{True}\PYG{p}{]}
\end{sphinxVerbatim}

\sphinxAtStartPar
As listas podem ser utilizadas para representar uma ampla variedade de dados. Por exemplo, podemos usar listas para representar listas de compras, listas de tarefas, etc.

\sphinxAtStartPar
Aqui estão alguns exemplos específicos de uso de listas:

\sphinxAtStartPar
Para representar uma lista de compras, podemos usar uma lista:

\begin{sphinxVerbatim}[commandchars=\\\{\}]
\PYG{n}{lista\PYGZus{}de\PYGZus{}compras} \PYG{o}{=} \PYG{p}{[}\PYG{l+s+s2}{\PYGZdq{}}\PYG{l+s+s2}{pão}\PYG{l+s+s2}{\PYGZdq{}}\PYG{p}{,} \PYG{l+s+s2}{\PYGZdq{}}\PYG{l+s+s2}{leite}\PYG{l+s+s2}{\PYGZdq{}}\PYG{p}{,} \PYG{l+s+s2}{\PYGZdq{}}\PYG{l+s+s2}{ovos}\PYG{l+s+s2}{\PYGZdq{}}\PYG{p}{]}
\end{sphinxVerbatim}

\sphinxAtStartPar
Representando uma lista de tarefas, podemos usar uma lista:

\begin{sphinxVerbatim}[commandchars=\\\{\}]
\PYG{n}{lista\PYGZus{}de\PYGZus{}tarefas} \PYG{o}{=} \PYG{p}{[}\PYG{l+s+s2}{\PYGZdq{}}\PYG{l+s+s2}{lavar a louça}\PYG{l+s+s2}{\PYGZdq{}}\PYG{p}{,} \PYG{l+s+s2}{\PYGZdq{}}\PYG{l+s+s2}{varrer a casa}\PYG{l+s+s2}{\PYGZdq{}}\PYG{p}{,} \PYG{l+s+s2}{\PYGZdq{}}\PYG{l+s+s2}{estudar python}\PYG{l+s+s2}{\PYGZdq{}}\PYG{p}{]}
\end{sphinxVerbatim}

\sphinxAtStartPar
Para representar uma lista de números, podemos usar uma lista:

\begin{sphinxVerbatim}[commandchars=\\\{\}]
\PYG{n}{lista\PYGZus{}de\PYGZus{}numeros} \PYG{o}{=} \PYG{p}{[}\PYG{l+m+mi}{1}\PYG{p}{,} \PYG{l+m+mi}{2}\PYG{p}{,} \PYG{l+m+mi}{3}\PYG{p}{,} \PYG{l+m+mi}{10}\PYG{p}{,} \PYG{l+m+mi}{15}\PYG{p}{]}
\end{sphinxVerbatim}

\sphinxAtStartPar
Uma mesma lista também pode receber tipos diversos em sua atribuição. É comum a necessidade dessa mistura de tipos para solução de alguns tipos de problemas.

\begin{sphinxVerbatim}[commandchars=\\\{\}]
\PYG{n}{lista\PYGZus{}mista\PYGZus{}tipos} \PYG{o}{=} \PYG{p}{[}\PYG{l+m+mi}{1}\PYG{p}{,} \PYG{l+m+mf}{2.5}\PYG{p}{,} \PYG{l+s+s2}{\PYGZdq{}}\PYG{l+s+s2}{Estou}\PYG{l+s+s2}{\PYGZdq{}}\PYG{p}{,} \PYG{k+kc}{True}\PYG{p}{,} \PYG{p}{[}\PYG{l+m+mi}{1}\PYG{p}{,} \PYG{l+m+mi}{2}\PYG{p}{,} \PYG{l+m+mi}{3}\PYG{p}{]}\PYG{p}{]}
\end{sphinxVerbatim}

\sphinxAtStartPar
Como pode ser visto, a lista \sphinxcode{\sphinxupquote{lista\_mista\_tipos}} contém um inteiro, um número de ponto flutuante, uma string, um booleano e uma lista aninhada.

\sphinxAtStartPar
Vamos agora conhecer os principais métodos utilizados quando trabalhamos com listas. Mas antes, o que são métodos em Python?

\sphinxAtStartPar
Em termos simples, métodos em Python são como instruções especiais que dizemos a uma lista para ela realizar tarefas específicas. Cada método tem uma função específica, como adicionar, remover ou organizar elementos na lista.

\begin{sphinxuseclass}{cell}\begin{sphinxVerbatimInput}

\begin{sphinxuseclass}{cell_input}
\begin{sphinxVerbatim}[commandchars=\\\{\}]
\PYG{c+c1}{\PYGZsh{} Criando a Lista inicial}
\PYG{n}{lista\PYGZus{}mista} \PYG{o}{=} \PYG{p}{[}\PYG{l+m+mi}{1}\PYG{p}{,} \PYG{l+m+mf}{2.5}\PYG{p}{,} \PYG{l+s+s2}{\PYGZdq{}}\PYG{l+s+s2}{Estou}\PYG{l+s+s2}{\PYGZdq{}}\PYG{p}{,} \PYG{k+kc}{True}\PYG{p}{]}

\PYG{c+c1}{\PYGZsh{} Adiciona um elemento ao final da lista}
\PYG{n}{lista\PYGZus{}mista}\PYG{o}{.}\PYG{n}{append}\PYG{p}{(}\PYG{l+m+mi}{4}\PYG{p}{)}
\PYG{n+nb}{print}\PYG{p}{(}\PYG{n}{lista\PYGZus{}mista}\PYG{p}{)}

\PYG{c+c1}{\PYGZsh{} Adiciona vários elementos ao final da lista}
\PYG{n}{lista\PYGZus{}mista}\PYG{o}{.}\PYG{n}{extend}\PYG{p}{(}\PYG{p}{[}\PYG{l+m+mi}{5}\PYG{p}{,} \PYG{l+m+mi}{6}\PYG{p}{,} \PYG{l+m+mi}{7}\PYG{p}{]}\PYG{p}{)}
\PYG{n+nb}{print}\PYG{p}{(}\PYG{n}{lista\PYGZus{}mista}\PYG{p}{)}

\PYG{c+c1}{\PYGZsh{} Insere um elemento em uma posição específica da lista}
\PYG{n}{lista\PYGZus{}mista}\PYG{o}{.}\PYG{n}{insert}\PYG{p}{(}\PYG{l+m+mi}{2}\PYG{p}{,} \PYG{l+s+s2}{\PYGZdq{}}\PYG{l+s+s2}{Olá}\PYG{l+s+s2}{\PYGZdq{}}\PYG{p}{)}
\PYG{n+nb}{print}\PYG{p}{(}\PYG{n}{lista\PYGZus{}mista}\PYG{p}{)}

\PYG{c+c1}{\PYGZsh{} Remove um elemento da lista}
\PYG{n}{lista\PYGZus{}mista}\PYG{o}{.}\PYG{n}{remove}\PYG{p}{(}\PYG{l+s+s2}{\PYGZdq{}}\PYG{l+s+s2}{Estou}\PYG{l+s+s2}{\PYGZdq{}}\PYG{p}{)}
\PYG{n+nb}{print}\PYG{p}{(}\PYG{n}{lista\PYGZus{}mista}\PYG{p}{)}

\PYG{c+c1}{\PYGZsh{} Remove o último elemento da lista}
\PYG{n}{lista\PYGZus{}mista}\PYG{o}{.}\PYG{n}{pop}\PYG{p}{(}\PYG{p}{)}
\PYG{n+nb}{print}\PYG{p}{(}\PYG{n}{lista\PYGZus{}mista}\PYG{p}{)}

\PYG{c+c1}{\PYGZsh{} Conta quantas vezes um elemento aparece na lista}
\PYG{n}{ocorrencias} \PYG{o}{=} \PYG{n}{lista\PYGZus{}mista}\PYG{o}{.}\PYG{n}{count}\PYG{p}{(}\PYG{l+m+mf}{2.5}\PYG{p}{)}
\PYG{n+nb}{print}\PYG{p}{(}\PYG{n}{ocorrencias}\PYG{p}{)}

\PYG{c+c1}{\PYGZsh{} Inverte a ordem da lista}
\PYG{n}{lista\PYGZus{}mista}\PYG{o}{.}\PYG{n}{reverse}\PYG{p}{(}\PYG{p}{)}
\PYG{n+nb}{print}\PYG{p}{(}\PYG{l+s+s2}{\PYGZdq{}}\PYG{l+s+s2}{Lista invertida:}\PYG{l+s+s2}{\PYGZdq{}}\PYG{p}{,} \PYG{n}{lista\PYGZus{}mista}\PYG{p}{)}
\end{sphinxVerbatim}

\end{sphinxuseclass}\end{sphinxVerbatimInput}
\begin{sphinxVerbatimOutput}

\begin{sphinxuseclass}{cell_output}
\begin{sphinxVerbatim}[commandchars=\\\{\}]
[1, 2.5, \PYGZsq{}Estou\PYGZsq{}, True, 4]
[1, 2.5, \PYGZsq{}Estou\PYGZsq{}, True, 4, 5, 6, 7]
[1, 2.5, \PYGZsq{}Olá\PYGZsq{}, \PYGZsq{}Estou\PYGZsq{}, True, 4, 5, 6, 7]
[1, 2.5, \PYGZsq{}Olá\PYGZsq{}, True, 4, 5, 6, 7]
[1, 2.5, \PYGZsq{}Olá\PYGZsq{}, True, 4, 5, 6]
1
Lista invertida: [6, 5, 4, True, \PYGZsq{}Olá\PYGZsq{}, 2.5, 1]
\end{sphinxVerbatim}

\end{sphinxuseclass}\end{sphinxVerbatimOutput}

\end{sphinxuseclass}
\sphinxAtStartPar
Em Python, os elementos em uma lista são organizados em uma sequência numerada chamada índice. O índice é como a posição de cada elemento na lista, começando do zero para o primeiro elemento. Por exemplo, em uma lista \sphinxcode{\sphinxupquote{{[}1, 2, 3, 4{]}}}, o número 1 está no índice 0, o 2 no índice 1, e assim por diante.

\sphinxAtStartPar
\sphinxstylestrong{Como Funciona:}
\begin{itemize}
\item {} 
\sphinxAtStartPar
\sphinxstylestrong{Indexação Positiva:} Você pode acessar elementos contando a partir do início da lista. O primeiro elemento tem índice 0, o segundo tem índice 1, e assim por diante.

\item {} 
\sphinxAtStartPar
\sphinxstylestrong{Indexação Negativa:} Também é possível contar a partir do final da lista. O último elemento tem índice \sphinxhyphen{}1, o penúltimo tem índice \sphinxhyphen{}2, e assim por diante.

\end{itemize}

\sphinxAtStartPar
\sphinxstylestrong{Como Utilizar:}
\begin{itemize}
\item {} 
\sphinxAtStartPar
Para acessar um elemento em uma lista, você utiliza o operador de colchetes \sphinxcode{\sphinxupquote{{[}{]}}} com o índice desejado. Por exemplo, \sphinxcode{\sphinxupquote{lista{[}2{]}}} retorna o terceiro elemento da lista.

\item {} 
\sphinxAtStartPar
Para modificar um elemento, você pode usar a mesma notação de índice. Por exemplo, \sphinxcode{\sphinxupquote{lista{[}1{]} = 10}} atribui o valor 10 ao segundo elemento da lista.

\end{itemize}

\sphinxAtStartPar
Aqui está um exemplo prático:

\begin{sphinxuseclass}{cell}\begin{sphinxVerbatimInput}

\begin{sphinxuseclass}{cell_input}
\begin{sphinxVerbatim}[commandchars=\\\{\}]
\PYG{c+c1}{\PYGZsh{} Lista de exemplo}
\PYG{n}{numeros} \PYG{o}{=} \PYG{p}{[}\PYG{l+m+mi}{10}\PYG{p}{,} \PYG{l+m+mi}{20}\PYG{p}{,} \PYG{l+m+mi}{30}\PYG{p}{,} \PYG{l+m+mi}{40}\PYG{p}{,} \PYG{l+m+mi}{50}\PYG{p}{]}

\PYG{c+c1}{\PYGZsh{} Acessando elementos}
\PYG{n+nb}{print}\PYG{p}{(}\PYG{l+s+s2}{\PYGZdq{}}\PYG{l+s+s2}{Primeiro elemento:}\PYG{l+s+s2}{\PYGZdq{}}\PYG{p}{,} \PYG{n}{numeros}\PYG{p}{[}\PYG{l+m+mi}{0}\PYG{p}{]}\PYG{p}{)}
\PYG{n+nb}{print}\PYG{p}{(}\PYG{l+s+s2}{\PYGZdq{}}\PYG{l+s+s2}{Último elemento:}\PYG{l+s+s2}{\PYGZdq{}}\PYG{p}{,} \PYG{n}{numeros}\PYG{p}{[}\PYG{o}{\PYGZhy{}}\PYG{l+m+mi}{1}\PYG{p}{]}\PYG{p}{)}

\PYG{c+c1}{\PYGZsh{} Modificando elementos}
\PYG{n}{numeros}\PYG{p}{[}\PYG{l+m+mi}{2}\PYG{p}{]} \PYG{o}{=} \PYG{l+m+mi}{35}
\PYG{n+nb}{print}\PYG{p}{(}\PYG{l+s+s2}{\PYGZdq{}}\PYG{l+s+s2}{Lista modificada:}\PYG{l+s+s2}{\PYGZdq{}}\PYG{p}{,} \PYG{n}{numeros}\PYG{p}{)}
\end{sphinxVerbatim}

\end{sphinxuseclass}\end{sphinxVerbatimInput}
\begin{sphinxVerbatimOutput}

\begin{sphinxuseclass}{cell_output}
\begin{sphinxVerbatim}[commandchars=\\\{\}]
Primeiro elemento: 10
Último elemento: 50
Lista modificada: [10, 20, 35, 40, 50]
\end{sphinxVerbatim}

\end{sphinxuseclass}\end{sphinxVerbatimOutput}

\end{sphinxuseclass}
\sphinxAtStartPar
Lembre\sphinxhyphen{}se de que os índices devem estar dentro da faixa válida da lista para evitar erros.


\subsection{Tuplas}
\label{\detokenize{chapters/ch2/ch2:tuplas}}
\sphinxAtStartPar
As tuplas são uma sequência de valores imutáveis. Isso significa que, uma vez criadas, as tuplas não podem ser alteradas.
As tuplas são declaradas usando parênteses, com os valores separados por vírgulas. Por exemplo, a seguinte declaração cria uma tupla com três valores:

\begin{sphinxVerbatim}[commandchars=\\\{\}]
\PYG{n}{tupla} \PYG{o}{=} \PYG{p}{(}\PYG{l+m+mi}{1}\PYG{p}{,} \PYG{l+m+mi}{2}\PYG{p}{,} \PYG{l+m+mi}{3}\PYG{p}{)}
\end{sphinxVerbatim}

\sphinxAtStartPar
As tuplas podem conter valores de qualquer tipo, incluindo inteiros, reais, strings, booleanos, etc.

\sphinxAtStartPar
Aqui estão alguns exemplos de uso:

\begin{sphinxVerbatim}[commandchars=\\\{\}]
\PYG{c+c1}{\PYGZsh{} Declaração de uma tupla com três valores inteiros}
\PYG{n}{inteiros} \PYG{o}{=} \PYG{p}{(}\PYG{l+m+mi}{1}\PYG{p}{,} \PYG{l+m+mi}{2}\PYG{p}{,} \PYG{l+m+mi}{3}\PYG{p}{)}

\PYG{c+c1}{\PYGZsh{} Declaração de uma tupla com três valores reais}
\PYG{n}{reais} \PYG{o}{=} \PYG{p}{(}\PYG{l+m+mf}{1.0}\PYG{p}{,} \PYG{l+m+mf}{2.5}\PYG{p}{,} \PYG{l+m+mf}{3.14}\PYG{p}{)}

\PYG{c+c1}{\PYGZsh{} Declaração de uma tupla com três valores strings}
\PYG{n}{strings} \PYG{o}{=} \PYG{p}{(}\PYG{l+s+s2}{\PYGZdq{}}\PYG{l+s+s2}{Olá}\PYG{l+s+s2}{\PYGZdq{}}\PYG{p}{,} \PYG{l+s+s2}{\PYGZdq{}}\PYG{l+s+s2}{mundo}\PYG{l+s+s2}{\PYGZdq{}}\PYG{p}{,} \PYG{l+s+s2}{\PYGZdq{}}\PYG{l+s+s2}{!}\PYG{l+s+s2}{\PYGZdq{}}\PYG{p}{)}

\PYG{c+c1}{\PYGZsh{} Declaração de uma tupla com três valores booleanos}
\PYG{n}{booleanos} \PYG{o}{=} \PYG{p}{(}\PYG{k+kc}{True}\PYG{p}{,} \PYG{k+kc}{False}\PYG{p}{,} \PYG{k+kc}{True}\PYG{p}{)}

\PYG{c+c1}{\PYGZsh{} Declaração de uma tupla com valores de tipos diferentes}
\PYG{n}{tupla\PYGZus{}mista} \PYG{o}{=} \PYG{p}{(}\PYG{k+kc}{True}\PYG{p}{,} \PYG{l+m+mi}{19}\PYG{p}{,} \PYG{l+m+mf}{7.5}\PYG{p}{,} \PYG{l+s+s2}{\PYGZdq{}}\PYG{l+s+s2}{a}\PYG{l+s+s2}{\PYGZdq{}}\PYG{p}{)}
\end{sphinxVerbatim}

\sphinxAtStartPar
As tuplas são um tipo de dados útil para representar dados que não precisam ser alterados. Por exemplo, podemos usar tuplas para representar coordenadas, dados de identificação, etc.

\sphinxAtStartPar
Aqui estão alguns exemplos específicos de uso de tuplas em Python:

\sphinxAtStartPar
Para representar as coordenadas de um ponto no plano cartesiano, podemos usar uma tupla:

\begin{sphinxVerbatim}[commandchars=\\\{\}]
\PYG{n}{ponto} \PYG{o}{=} \PYG{p}{(}\PYG{l+m+mf}{1.0}\PYG{p}{,} \PYG{l+m+mf}{2.0}\PYG{p}{)}
\end{sphinxVerbatim}

\sphinxAtStartPar
Representaando o número de identificação de um funcionário, podemos usar uma tupla:

\begin{sphinxVerbatim}[commandchars=\\\{\}]
\PYG{n}{identificacao} \PYG{o}{=} \PYG{p}{(}\PYG{l+m+mi}{1234567890}\PYG{p}{,} \PYG{l+s+s2}{\PYGZdq{}}\PYG{l+s+s2}{João da Silva}\PYG{l+s+s2}{\PYGZdq{}}\PYG{p}{)}
\end{sphinxVerbatim}

\sphinxAtStartPar
Para representar os dados de um produto, podemos usar uma tupla:

\begin{sphinxVerbatim}[commandchars=\\\{\}]
\PYG{n}{produto} \PYG{o}{=} \PYG{p}{(}\PYG{l+s+s2}{\PYGZdq{}}\PYG{l+s+s2}{Camisa}\PYG{l+s+s2}{\PYGZdq{}}\PYG{p}{,} \PYG{l+s+s2}{\PYGZdq{}}\PYG{l+s+s2}{P}\PYG{l+s+s2}{\PYGZdq{}}\PYG{p}{,} \PYG{l+m+mf}{100.0}\PYG{p}{)}
\end{sphinxVerbatim}

\sphinxAtStartPar
\sphinxstylestrong{Índices em Tuplas}

\sphinxAtStartPar
Assim como em listas, as tuplas também utilizam índices para acessar seus elementos. Os índices em tuplas começam do 0 para o primeiro elemento e seguem uma sequência numérica.

\sphinxAtStartPar
\sphinxstylestrong{Exemplo de Acesso por Índice:}

\begin{sphinxuseclass}{cell}\begin{sphinxVerbatimInput}

\begin{sphinxuseclass}{cell_input}
\begin{sphinxVerbatim}[commandchars=\\\{\}]
\PYG{c+c1}{\PYGZsh{} Declaração de uma tupla}
\PYG{n}{tupla} \PYG{o}{=} \PYG{p}{(}\PYG{l+m+mi}{10}\PYG{p}{,} \PYG{l+m+mi}{20}\PYG{p}{,} \PYG{l+m+mi}{30}\PYG{p}{,} \PYG{l+m+mi}{40}\PYG{p}{,} \PYG{l+m+mi}{50}\PYG{p}{)}

\PYG{c+c1}{\PYGZsh{} Acessando elementos por índice}
\PYG{n}{primeiro\PYGZus{}elemento} \PYG{o}{=} \PYG{n}{tupla}\PYG{p}{[}\PYG{l+m+mi}{0}\PYG{p}{]}
\PYG{n}{terceiro\PYGZus{}elemento} \PYG{o}{=} \PYG{n}{tupla}\PYG{p}{[}\PYG{l+m+mi}{2}\PYG{p}{]}

\PYG{n+nb}{print}\PYG{p}{(}\PYG{l+s+s2}{\PYGZdq{}}\PYG{l+s+s2}{Primeiro elemento:}\PYG{l+s+s2}{\PYGZdq{}}\PYG{p}{,} \PYG{n}{primeiro\PYGZus{}elemento}\PYG{p}{)}
\PYG{n+nb}{print}\PYG{p}{(}\PYG{l+s+s2}{\PYGZdq{}}\PYG{l+s+s2}{Terceiro elemento:}\PYG{l+s+s2}{\PYGZdq{}}\PYG{p}{,} \PYG{n}{terceiro\PYGZus{}elemento}\PYG{p}{)}
\end{sphinxVerbatim}

\end{sphinxuseclass}\end{sphinxVerbatimInput}
\begin{sphinxVerbatimOutput}

\begin{sphinxuseclass}{cell_output}
\begin{sphinxVerbatim}[commandchars=\\\{\}]
Primeiro elemento: 10
Terceiro elemento: 30
\end{sphinxVerbatim}

\end{sphinxuseclass}\end{sphinxVerbatimOutput}

\end{sphinxuseclass}
\sphinxAtStartPar
\sphinxstylestrong{Observações:}
Os índices negativos funcionam da mesma forma que em listas, onde \sphinxhyphen{}1 refere\sphinxhyphen{}se ao último elemento, \sphinxhyphen{}2 ao penúltimo, e assim por diante.

\begin{sphinxuseclass}{cell}\begin{sphinxVerbatimInput}

\begin{sphinxuseclass}{cell_input}
\begin{sphinxVerbatim}[commandchars=\\\{\}]
\PYG{c+c1}{\PYGZsh{} Acessando o último elemento por índice negativo}
\PYG{n}{ultimo\PYGZus{}elemento} \PYG{o}{=} \PYG{n}{tupla}\PYG{p}{[}\PYG{o}{\PYGZhy{}}\PYG{l+m+mi}{1}\PYG{p}{]}
\PYG{n+nb}{print}\PYG{p}{(}\PYG{l+s+s2}{\PYGZdq{}}\PYG{l+s+s2}{Último elemento:}\PYG{l+s+s2}{\PYGZdq{}}\PYG{p}{,} \PYG{n}{ultimo\PYGZus{}elemento}\PYG{p}{)}
\end{sphinxVerbatim}

\end{sphinxuseclass}\end{sphinxVerbatimInput}
\begin{sphinxVerbatimOutput}

\begin{sphinxuseclass}{cell_output}
\begin{sphinxVerbatim}[commandchars=\\\{\}]
Último elemento: 50
\end{sphinxVerbatim}

\end{sphinxuseclass}\end{sphinxVerbatimOutput}

\end{sphinxuseclass}
\sphinxAtStartPar
\sphinxstylestrong{Índices em Tuplas Mistas}

\begin{sphinxuseclass}{cell}\begin{sphinxVerbatimInput}

\begin{sphinxuseclass}{cell_input}
\begin{sphinxVerbatim}[commandchars=\\\{\}]
\PYG{c+c1}{\PYGZsh{} Declaração de uma tupla com valores de tipos diferentes}
\PYG{n}{tupla\PYGZus{}mista} \PYG{o}{=} \PYG{p}{(}\PYG{k+kc}{True}\PYG{p}{,} \PYG{l+m+mi}{19}\PYG{p}{,} \PYG{l+m+mf}{7.5}\PYG{p}{,} \PYG{l+s+s2}{\PYGZdq{}}\PYG{l+s+s2}{a}\PYG{l+s+s2}{\PYGZdq{}}\PYG{p}{)}

\PYG{c+c1}{\PYGZsh{} Acessando elementos por índice}
\PYG{n}{primeiro\PYGZus{}elemento\PYGZus{}misto} \PYG{o}{=} \PYG{n}{tupla\PYGZus{}mista}\PYG{p}{[}\PYG{l+m+mi}{0}\PYG{p}{]}
\PYG{n}{ultimo\PYGZus{}elemento\PYGZus{}misto} \PYG{o}{=} \PYG{n}{tupla\PYGZus{}mista}\PYG{p}{[}\PYG{o}{\PYGZhy{}}\PYG{l+m+mi}{1}\PYG{p}{]}

\PYG{n+nb}{print}\PYG{p}{(}\PYG{l+s+s2}{\PYGZdq{}}\PYG{l+s+s2}{Primeiro elemento misto:}\PYG{l+s+s2}{\PYGZdq{}}\PYG{p}{,} \PYG{n}{primeiro\PYGZus{}elemento\PYGZus{}misto}\PYG{p}{)}
\PYG{n+nb}{print}\PYG{p}{(}\PYG{l+s+s2}{\PYGZdq{}}\PYG{l+s+s2}{Último elemento misto:}\PYG{l+s+s2}{\PYGZdq{}}\PYG{p}{,} \PYG{n}{ultimo\PYGZus{}elemento\PYGZus{}misto}\PYG{p}{)}
\end{sphinxVerbatim}

\end{sphinxuseclass}\end{sphinxVerbatimInput}
\begin{sphinxVerbatimOutput}

\begin{sphinxuseclass}{cell_output}
\begin{sphinxVerbatim}[commandchars=\\\{\}]
Primeiro elemento misto: True
Último elemento misto: a
\end{sphinxVerbatim}

\end{sphinxuseclass}\end{sphinxVerbatimOutput}

\end{sphinxuseclass}
\sphinxAtStartPar
Os índices em tuplas permitem acessar e trabalhar com os elementos individualmente, facilitando a manipulação dessas estruturas de dados imutáveis em Python.

\sphinxAtStartPar
Assim como as listas, as tuplas em Python possuem alguns métodos especiais que podem ser utilizados para realizar operações específicas. No entanto, devido à imutabilidade das tuplas (não é possível modificar uma tupla após sua criação), esses métodos são mais limitados em comparação com os disponíveis para listas.

\sphinxAtStartPar
Aqui estão alguns dos métodos especiais comumente utilizados em tuplas:
\begin{itemize}
\item {} 
\sphinxAtStartPar
\sphinxstylestrong{count(valor):}
Retorna o número de ocorrências do valor especificado na tupla.

\sphinxAtStartPar
Exemplo:

\end{itemize}

\begin{sphinxuseclass}{cell}\begin{sphinxVerbatimInput}

\begin{sphinxuseclass}{cell_input}
\begin{sphinxVerbatim}[commandchars=\\\{\}]
   \PYG{n}{minha\PYGZus{}tupla} \PYG{o}{=} \PYG{p}{(}\PYG{l+m+mi}{1}\PYG{p}{,} \PYG{l+m+mi}{2}\PYG{p}{,} \PYG{l+m+mi}{2}\PYG{p}{,} \PYG{l+m+mi}{3}\PYG{p}{,} \PYG{l+m+mi}{4}\PYG{p}{,} \PYG{l+m+mi}{2}\PYG{p}{)}
   \PYG{n}{numero\PYGZus{}de\PYGZus{}dois} \PYG{o}{=} \PYG{n}{minha\PYGZus{}tupla}\PYG{o}{.}\PYG{n}{count}\PYG{p}{(}\PYG{l+m+mi}{2}\PYG{p}{)}
   \PYG{n+nb}{print}\PYG{p}{(}\PYG{n}{numero\PYGZus{}de\PYGZus{}dois}\PYG{p}{)}
\end{sphinxVerbatim}

\end{sphinxuseclass}\end{sphinxVerbatimInput}
\begin{sphinxVerbatimOutput}

\begin{sphinxuseclass}{cell_output}
\begin{sphinxVerbatim}[commandchars=\\\{\}]
3
\end{sphinxVerbatim}

\end{sphinxuseclass}\end{sphinxVerbatimOutput}

\end{sphinxuseclass}\begin{itemize}
\item {} 
\sphinxAtStartPar
\sphinxstylestrong{index(valor{[}, start{[}, stop{]}{]}):}
Retorna o índice da primeira ocorrência do valor especificado. Você pode opcionalmente fornecer os argumentos \sphinxcode{\sphinxupquote{start}} e \sphinxcode{\sphinxupquote{stop}} para limitar a busca a uma sub\sphinxhyphen{}tupla.

\sphinxAtStartPar
Exemplo:

\end{itemize}

\begin{sphinxuseclass}{cell}\begin{sphinxVerbatimInput}

\begin{sphinxuseclass}{cell_input}
\begin{sphinxVerbatim}[commandchars=\\\{\}]
\PYG{n}{minha\PYGZus{}tupla} \PYG{o}{=} \PYG{p}{(}\PYG{l+m+mi}{1}\PYG{p}{,} \PYG{l+m+mi}{2}\PYG{p}{,} \PYG{l+m+mi}{3}\PYG{p}{,} \PYG{l+m+mi}{4}\PYG{p}{,} \PYG{l+m+mi}{5}\PYG{p}{)}
\PYG{n}{indice\PYGZus{}do\PYGZus{}tres} \PYG{o}{=} \PYG{n}{minha\PYGZus{}tupla}\PYG{o}{.}\PYG{n}{index}\PYG{p}{(}\PYG{l+m+mi}{3}\PYG{p}{)}
\PYG{n+nb}{print}\PYG{p}{(}\PYG{n}{indice\PYGZus{}do\PYGZus{}tres}\PYG{p}{)}
\end{sphinxVerbatim}

\end{sphinxuseclass}\end{sphinxVerbatimInput}
\begin{sphinxVerbatimOutput}

\begin{sphinxuseclass}{cell_output}
\begin{sphinxVerbatim}[commandchars=\\\{\}]
2
\end{sphinxVerbatim}

\end{sphinxuseclass}\end{sphinxVerbatimOutput}

\end{sphinxuseclass}
\sphinxAtStartPar
Lembre\sphinxhyphen{}se de que, devido à imutabilidade das tuplas, métodos que alteram o conteúdo (como \sphinxcode{\sphinxupquote{append}}, \sphinxcode{\sphinxupquote{extend}}, \sphinxcode{\sphinxupquote{remove}}, \sphinxcode{\sphinxupquote{pop}}, \sphinxcode{\sphinxupquote{insert}}, etc.) não estão disponíveis para tuplas.


\subsection{Dicionários}
\label{\detokenize{chapters/ch2/ch2:dicionarios}}
\sphinxAtStartPar
Os dicionários servem para armazenar informações por meio de pares de chave e valor. Cada elemento do dicionário consiste em uma chave única associada a um valor correspondente.

\sphinxAtStartPar
Para criar um dicionário, utilizamos chaves \sphinxcode{\sphinxupquote{\{\}}} e separamos cada par chave\sphinxhyphen{}valor por vírgulas. Exemplo:

\begin{sphinxVerbatim}[commandchars=\\\{\}]
\PYG{n}{dicionario} \PYG{o}{=} \PYG{p}{\PYGZob{}}\PYG{l+s+s2}{\PYGZdq{}}\PYG{l+s+s2}{nome}\PYG{l+s+s2}{\PYGZdq{}}\PYG{p}{:} \PYG{l+s+s2}{\PYGZdq{}}\PYG{l+s+s2}{João}\PYG{l+s+s2}{\PYGZdq{}}\PYG{p}{,} \PYG{l+s+s2}{\PYGZdq{}}\PYG{l+s+s2}{idade}\PYG{l+s+s2}{\PYGZdq{}}\PYG{p}{:} \PYG{l+m+mi}{30}\PYG{p}{,} \PYG{l+s+s2}{\PYGZdq{}}\PYG{l+s+s2}{cidade}\PYG{l+s+s2}{\PYGZdq{}}\PYG{p}{:} \PYG{l+s+s2}{\PYGZdq{}}\PYG{l+s+s2}{Natal}\PYG{l+s+s2}{\PYGZdq{}}\PYG{p}{\PYGZcb{}}
\end{sphinxVerbatim}
\begin{itemize}
\item {} 
\sphinxAtStartPar
\sphinxstylestrong{Chaves:} São rótulos exclusivos que acessam os valores associados.

\item {} 
\sphinxAtStartPar
\sphinxstylestrong{Tipos de Chaves:} Podem ser de qualquer tipo de dado imutável, como strings, números inteiros ou reais.

\item {} 
\sphinxAtStartPar
\sphinxstylestrong{Tipos de Valores:} Podem ser de qualquer tipo de dado, inclusive listas ou tuplas.

\end{itemize}

\sphinxAtStartPar
Os dicionários são úteis quando queremos associar informações relacionadas entre si. Por exemplo, no dicionário acima, “nome” é a chave associada ao valor “João”. Essa estrutura flexível e poderosa é amplamente utilizada em Python para representar dados estruturados de forma eficiente.

\sphinxAtStartPar
\sphinxstylestrong{Exemplos de Uso:}

\begin{sphinxVerbatim}[commandchars=\\\{\}]
\PYG{c+c1}{\PYGZsh{} Acesso a um valor do dicionário}
\PYG{n}{nome} \PYG{o}{=} \PYG{n}{dicionario}\PYG{p}{[}\PYG{l+s+s2}{\PYGZdq{}}\PYG{l+s+s2}{nome}\PYG{l+s+s2}{\PYGZdq{}}\PYG{p}{]}
\PYG{n+nb}{print}\PYG{p}{(}\PYG{l+s+s2}{\PYGZdq{}}\PYG{l+s+s2}{Nome:}\PYG{l+s+s2}{\PYGZdq{}}\PYG{p}{,} \PYG{n}{nome}\PYG{p}{)}

\PYG{c+c1}{\PYGZsh{} Alteração de um valor do dicionário}
\PYG{n}{dicionario}\PYG{p}{[}\PYG{l+s+s2}{\PYGZdq{}}\PYG{l+s+s2}{idade}\PYG{l+s+s2}{\PYGZdq{}}\PYG{p}{]} \PYG{o}{=} \PYG{l+m+mi}{31}
\PYG{n+nb}{print}\PYG{p}{(}\PYG{l+s+s2}{\PYGZdq{}}\PYG{l+s+s2}{Idade atualizada:}\PYG{l+s+s2}{\PYGZdq{}}\PYG{p}{,} \PYG{n}{dicionario}\PYG{p}{[}\PYG{l+s+s2}{\PYGZdq{}}\PYG{l+s+s2}{idade}\PYG{l+s+s2}{\PYGZdq{}}\PYG{p}{]}\PYG{p}{)}

\PYG{c+c1}{\PYGZsh{} Remoção de um valor do dicionário}
\PYG{k}{del} \PYG{n}{dicionario}\PYG{p}{[}\PYG{l+s+s2}{\PYGZdq{}}\PYG{l+s+s2}{cidade}\PYG{l+s+s2}{\PYGZdq{}}\PYG{p}{]}
\PYG{n+nb}{print}\PYG{p}{(}\PYG{l+s+s2}{\PYGZdq{}}\PYG{l+s+s2}{Dicionário após remoção:}\PYG{l+s+s2}{\PYGZdq{}}\PYG{p}{,} \PYG{n}{dicionario}\PYG{p}{)}
\end{sphinxVerbatim}

\sphinxAtStartPar
\sphinxstylestrong{Acesso a Elementos Individualmente:}

\sphinxAtStartPar
Podemos acessar cada valor individualmente no dicionário utilizando suas chaves:

\begin{sphinxVerbatim}[commandchars=\\\{\}]
\PYG{c+c1}{\PYGZsh{} Acesso a elementos individualmente}
\PYG{n}{telefone} \PYG{o}{=} \PYG{n}{contato}\PYG{p}{[}\PYG{l+s+s2}{\PYGZdq{}}\PYG{l+s+s2}{telefone}\PYG{l+s+s2}{\PYGZdq{}}\PYG{p}{]}
\PYG{n+nb}{print}\PYG{p}{(}\PYG{l+s+s2}{\PYGZdq{}}\PYG{l+s+s2}{Telefone de contato:}\PYG{l+s+s2}{\PYGZdq{}}\PYG{p}{,} \PYG{n}{telefone}\PYG{p}{)}

\PYG{c+c1}{\PYGZsh{} Acesso a configuração de idioma}
\PYG{n}{idioma\PYGZus{}config} \PYG{o}{=} \PYG{n}{configuracoes}\PYG{p}{[}\PYG{l+s+s2}{\PYGZdq{}}\PYG{l+s+s2}{idioma}\PYG{l+s+s2}{\PYGZdq{}}\PYG{p}{]}
\PYG{n+nb}{print}\PYG{p}{(}\PYG{l+s+s2}{\PYGZdq{}}\PYG{l+s+s2}{Configuração de idioma:}\PYG{l+s+s2}{\PYGZdq{}}\PYG{p}{,} \PYG{n}{idioma\PYGZus{}config}\PYG{p}{)}

\PYG{c+c1}{\PYGZsh{} Acesso à cor do produto}
\PYG{n}{cor\PYGZus{}produto} \PYG{o}{=} \PYG{n}{produto}\PYG{p}{[}\PYG{l+s+s2}{\PYGZdq{}}\PYG{l+s+s2}{cor}\PYG{l+s+s2}{\PYGZdq{}}\PYG{p}{]}
\PYG{n+nb}{print}\PYG{p}{(}\PYG{l+s+s2}{\PYGZdq{}}\PYG{l+s+s2}{Cor do produto:}\PYG{l+s+s2}{\PYGZdq{}}\PYG{p}{,} \PYG{n}{cor\PYGZus{}produto}\PYG{p}{)}
\end{sphinxVerbatim}

\sphinxAtStartPar
A capacidade de acessar elementos individualmente nos dicionários permite recuperar informações específicas de maneira direta e eficiente. Essa característica torna os dicionários uma estrutura de dados poderosa para representar e manipular dados em Python.

\sphinxAtStartPar
Os dicionários oferecem uma variedade de métodos para manipular e acessar os dados armazenados. Aqui estão alguns dos principais métodos de dicionários:

\sphinxAtStartPar
\sphinxstylestrong{\sphinxcode{\sphinxupquote{clear()}}:}
Remove todos os itens do dicionário.

\begin{sphinxuseclass}{cell}\begin{sphinxVerbatimInput}

\begin{sphinxuseclass}{cell_input}
\begin{sphinxVerbatim}[commandchars=\\\{\}]
\PYG{n}{meu\PYGZus{}dicionario} \PYG{o}{=} \PYG{p}{\PYGZob{}}\PYG{l+s+s2}{\PYGZdq{}}\PYG{l+s+s2}{nome}\PYG{l+s+s2}{\PYGZdq{}}\PYG{p}{:} \PYG{l+s+s2}{\PYGZdq{}}\PYG{l+s+s2}{João}\PYG{l+s+s2}{\PYGZdq{}}\PYG{p}{,} \PYG{l+s+s2}{\PYGZdq{}}\PYG{l+s+s2}{idade}\PYG{l+s+s2}{\PYGZdq{}}\PYG{p}{:} \PYG{l+m+mi}{25}\PYG{p}{\PYGZcb{}}
\PYG{n}{meu\PYGZus{}dicionario}\PYG{o}{.}\PYG{n}{clear}\PYG{p}{(}\PYG{p}{)}
\PYG{n+nb}{print}\PYG{p}{(}\PYG{n}{meu\PYGZus{}dicionario}\PYG{p}{)}
\end{sphinxVerbatim}

\end{sphinxuseclass}\end{sphinxVerbatimInput}
\begin{sphinxVerbatimOutput}

\begin{sphinxuseclass}{cell_output}
\begin{sphinxVerbatim}[commandchars=\\\{\}]
\PYGZob{}\PYGZcb{}
\end{sphinxVerbatim}

\end{sphinxuseclass}\end{sphinxVerbatimOutput}

\end{sphinxuseclass}
\sphinxAtStartPar
\sphinxstylestrong{\sphinxcode{\sphinxupquote{copy()}}:}
Retorna uma cópia do dicionário.

\begin{sphinxuseclass}{cell}\begin{sphinxVerbatimInput}

\begin{sphinxuseclass}{cell_input}
\begin{sphinxVerbatim}[commandchars=\\\{\}]
\PYG{n}{meu\PYGZus{}dicionario} \PYG{o}{=} \PYG{p}{\PYGZob{}}\PYG{l+s+s2}{\PYGZdq{}}\PYG{l+s+s2}{nome}\PYG{l+s+s2}{\PYGZdq{}}\PYG{p}{:} \PYG{l+s+s2}{\PYGZdq{}}\PYG{l+s+s2}{Maria}\PYG{l+s+s2}{\PYGZdq{}}\PYG{p}{,} \PYG{l+s+s2}{\PYGZdq{}}\PYG{l+s+s2}{idade}\PYG{l+s+s2}{\PYGZdq{}}\PYG{p}{:} \PYG{l+m+mi}{30}\PYG{p}{\PYGZcb{}}
\PYG{n}{copia\PYGZus{}dicionario} \PYG{o}{=} \PYG{n}{meu\PYGZus{}dicionario}\PYG{o}{.}\PYG{n}{copy}\PYG{p}{(}\PYG{p}{)}
\PYG{n+nb}{print}\PYG{p}{(}\PYG{n}{copia\PYGZus{}dicionario}\PYG{p}{)}
\end{sphinxVerbatim}

\end{sphinxuseclass}\end{sphinxVerbatimInput}
\begin{sphinxVerbatimOutput}

\begin{sphinxuseclass}{cell_output}
\begin{sphinxVerbatim}[commandchars=\\\{\}]
\PYGZob{}\PYGZsq{}nome\PYGZsq{}: \PYGZsq{}Maria\PYGZsq{}, \PYGZsq{}idade\PYGZsq{}: 30\PYGZcb{}
\end{sphinxVerbatim}

\end{sphinxuseclass}\end{sphinxVerbatimOutput}

\end{sphinxuseclass}
\sphinxAtStartPar
\sphinxstylestrong{\sphinxcode{\sphinxupquote{get(chave{[}, valor\_padrão{]})}}:}
Retorna o valor associado à chave especificada. Se a chave não existir, retorna um valor padrão (ou \sphinxcode{\sphinxupquote{None}} se não fornecido).

\begin{sphinxuseclass}{cell}\begin{sphinxVerbatimInput}

\begin{sphinxuseclass}{cell_input}
\begin{sphinxVerbatim}[commandchars=\\\{\}]
\PYG{n}{meu\PYGZus{}dicionario} \PYG{o}{=} \PYG{p}{\PYGZob{}}\PYG{l+s+s2}{\PYGZdq{}}\PYG{l+s+s2}{nome}\PYG{l+s+s2}{\PYGZdq{}}\PYG{p}{:} \PYG{l+s+s2}{\PYGZdq{}}\PYG{l+s+s2}{Carlos}\PYG{l+s+s2}{\PYGZdq{}}\PYG{p}{,} \PYG{l+s+s2}{\PYGZdq{}}\PYG{l+s+s2}{idade}\PYG{l+s+s2}{\PYGZdq{}}\PYG{p}{:} \PYG{l+m+mi}{28}\PYG{p}{\PYGZcb{}}
\PYG{n}{idade} \PYG{o}{=} \PYG{n}{meu\PYGZus{}dicionario}\PYG{o}{.}\PYG{n}{get}\PYG{p}{(}\PYG{l+s+s2}{\PYGZdq{}}\PYG{l+s+s2}{idade}\PYG{l+s+s2}{\PYGZdq{}}\PYG{p}{)}
\PYG{n+nb}{print}\PYG{p}{(}\PYG{n}{idade}\PYG{p}{)}
\end{sphinxVerbatim}

\end{sphinxuseclass}\end{sphinxVerbatimInput}
\begin{sphinxVerbatimOutput}

\begin{sphinxuseclass}{cell_output}
\begin{sphinxVerbatim}[commandchars=\\\{\}]
28
\end{sphinxVerbatim}

\end{sphinxuseclass}\end{sphinxVerbatimOutput}

\end{sphinxuseclass}
\sphinxAtStartPar
\sphinxstylestrong{\sphinxcode{\sphinxupquote{items()}}:}
Retorna uma lista de tuplas contendo pares chave\sphinxhyphen{}valor.

\begin{sphinxuseclass}{cell}\begin{sphinxVerbatimInput}

\begin{sphinxuseclass}{cell_input}
\begin{sphinxVerbatim}[commandchars=\\\{\}]
\PYG{n}{meu\PYGZus{}dicionario} \PYG{o}{=} \PYG{p}{\PYGZob{}}\PYG{l+s+s2}{\PYGZdq{}}\PYG{l+s+s2}{nome}\PYG{l+s+s2}{\PYGZdq{}}\PYG{p}{:} \PYG{l+s+s2}{\PYGZdq{}}\PYG{l+s+s2}{Ana}\PYG{l+s+s2}{\PYGZdq{}}\PYG{p}{,} \PYG{l+s+s2}{\PYGZdq{}}\PYG{l+s+s2}{idade}\PYG{l+s+s2}{\PYGZdq{}}\PYG{p}{:} \PYG{l+m+mi}{35}\PYG{p}{\PYGZcb{}}
\PYG{n}{itens} \PYG{o}{=} \PYG{n}{meu\PYGZus{}dicionario}\PYG{o}{.}\PYG{n}{items}\PYG{p}{(}\PYG{p}{)}
\PYG{n+nb}{print}\PYG{p}{(}\PYG{n}{itens}\PYG{p}{)}
\end{sphinxVerbatim}

\end{sphinxuseclass}\end{sphinxVerbatimInput}
\begin{sphinxVerbatimOutput}

\begin{sphinxuseclass}{cell_output}
\begin{sphinxVerbatim}[commandchars=\\\{\}]
dict\PYGZus{}items([(\PYGZsq{}nome\PYGZsq{}, \PYGZsq{}Ana\PYGZsq{}), (\PYGZsq{}idade\PYGZsq{}, 35)])
\end{sphinxVerbatim}

\begin{sphinxVerbatim}[commandchars=\\\{\}]

\end{sphinxVerbatim}

\end{sphinxuseclass}\end{sphinxVerbatimOutput}

\end{sphinxuseclass}
\sphinxAtStartPar
\sphinxstylestrong{\sphinxcode{\sphinxupquote{keys()}}:}
Retorna uma lista contendo todas as chaves do dicionário.

\begin{sphinxuseclass}{cell}\begin{sphinxVerbatimInput}

\begin{sphinxuseclass}{cell_input}
\begin{sphinxVerbatim}[commandchars=\\\{\}]
\PYG{n}{meu\PYGZus{}dicionario} \PYG{o}{=} \PYG{p}{\PYGZob{}}\PYG{l+s+s2}{\PYGZdq{}}\PYG{l+s+s2}{nome}\PYG{l+s+s2}{\PYGZdq{}}\PYG{p}{:} \PYG{l+s+s2}{\PYGZdq{}}\PYG{l+s+s2}{Lucas}\PYG{l+s+s2}{\PYGZdq{}}\PYG{p}{,} \PYG{l+s+s2}{\PYGZdq{}}\PYG{l+s+s2}{idade}\PYG{l+s+s2}{\PYGZdq{}}\PYG{p}{:} \PYG{l+m+mi}{22}\PYG{p}{\PYGZcb{}}
\PYG{n}{chaves} \PYG{o}{=} \PYG{n}{meu\PYGZus{}dicionario}\PYG{o}{.}\PYG{n}{keys}\PYG{p}{(}\PYG{p}{)}
\PYG{n+nb}{print}\PYG{p}{(}\PYG{n}{chaves}\PYG{p}{)}
\end{sphinxVerbatim}

\end{sphinxuseclass}\end{sphinxVerbatimInput}
\begin{sphinxVerbatimOutput}

\begin{sphinxuseclass}{cell_output}
\begin{sphinxVerbatim}[commandchars=\\\{\}]
dict\PYGZus{}keys([\PYGZsq{}nome\PYGZsq{}, \PYGZsq{}idade\PYGZsq{}])
\end{sphinxVerbatim}

\end{sphinxuseclass}\end{sphinxVerbatimOutput}

\end{sphinxuseclass}
\sphinxAtStartPar
\sphinxstylestrong{\sphinxcode{\sphinxupquote{values()}}:}
Retorna uma lista contendo todos os valores do dicionário.

\begin{sphinxuseclass}{cell}\begin{sphinxVerbatimInput}

\begin{sphinxuseclass}{cell_input}
\begin{sphinxVerbatim}[commandchars=\\\{\}]
\PYG{n}{meu\PYGZus{}dicionario} \PYG{o}{=} \PYG{p}{\PYGZob{}}\PYG{l+s+s2}{\PYGZdq{}}\PYG{l+s+s2}{nome}\PYG{l+s+s2}{\PYGZdq{}}\PYG{p}{:} \PYG{l+s+s2}{\PYGZdq{}}\PYG{l+s+s2}{Julia}\PYG{l+s+s2}{\PYGZdq{}}\PYG{p}{,} \PYG{l+s+s2}{\PYGZdq{}}\PYG{l+s+s2}{idade}\PYG{l+s+s2}{\PYGZdq{}}\PYG{p}{:} \PYG{l+m+mi}{27}\PYG{p}{\PYGZcb{}}
\PYG{n}{valores} \PYG{o}{=} \PYG{n}{meu\PYGZus{}dicionario}\PYG{o}{.}\PYG{n}{values}\PYG{p}{(}\PYG{p}{)}
\PYG{n+nb}{print}\PYG{p}{(}\PYG{n}{valores}\PYG{p}{)}
\end{sphinxVerbatim}

\end{sphinxuseclass}\end{sphinxVerbatimInput}
\begin{sphinxVerbatimOutput}

\begin{sphinxuseclass}{cell_output}
\begin{sphinxVerbatim}[commandchars=\\\{\}]
dict\PYGZus{}values([\PYGZsq{}Julia\PYGZsq{}, 27])
\end{sphinxVerbatim}

\begin{sphinxVerbatim}[commandchars=\\\{\}]

\end{sphinxVerbatim}

\end{sphinxuseclass}\end{sphinxVerbatimOutput}

\end{sphinxuseclass}
\sphinxAtStartPar
\sphinxstylestrong{\sphinxcode{\sphinxupquote{pop(chave{[}, valor\_padrão{]})}}:}
Remove e retorna o valor associado à chave especificada. Se a chave não existir, retorna um valor padrão (ou gera um erro se não fornecido).

\begin{sphinxuseclass}{cell}\begin{sphinxVerbatimInput}

\begin{sphinxuseclass}{cell_input}
\begin{sphinxVerbatim}[commandchars=\\\{\}]
\PYG{n}{meu\PYGZus{}dicionario} \PYG{o}{=} \PYG{p}{\PYGZob{}}\PYG{l+s+s2}{\PYGZdq{}}\PYG{l+s+s2}{nome}\PYG{l+s+s2}{\PYGZdq{}}\PYG{p}{:} \PYG{l+s+s2}{\PYGZdq{}}\PYG{l+s+s2}{Pedro}\PYG{l+s+s2}{\PYGZdq{}}\PYG{p}{,} \PYG{l+s+s2}{\PYGZdq{}}\PYG{l+s+s2}{idade}\PYG{l+s+s2}{\PYGZdq{}}\PYG{p}{:} \PYG{l+m+mi}{32}\PYG{p}{\PYGZcb{}}
\PYG{n}{idade} \PYG{o}{=} \PYG{n}{meu\PYGZus{}dicionario}\PYG{o}{.}\PYG{n}{pop}\PYG{p}{(}\PYG{l+s+s2}{\PYGZdq{}}\PYG{l+s+s2}{idade}\PYG{l+s+s2}{\PYGZdq{}}\PYG{p}{)}
\PYG{n+nb}{print}\PYG{p}{(}\PYG{n}{idade}\PYG{p}{)}
\end{sphinxVerbatim}

\end{sphinxuseclass}\end{sphinxVerbatimInput}
\begin{sphinxVerbatimOutput}

\begin{sphinxuseclass}{cell_output}
\begin{sphinxVerbatim}[commandchars=\\\{\}]
32
\end{sphinxVerbatim}

\end{sphinxuseclass}\end{sphinxVerbatimOutput}

\end{sphinxuseclass}
\sphinxAtStartPar
\sphinxstylestrong{\sphinxcode{\sphinxupquote{popitem()}}:}
Remove e retorna o último par chave\sphinxhyphen{}valor do dicionário como uma tupla.

\begin{sphinxuseclass}{cell}\begin{sphinxVerbatimInput}

\begin{sphinxuseclass}{cell_input}
\begin{sphinxVerbatim}[commandchars=\\\{\}]
\PYG{n}{meu\PYGZus{}dicionario} \PYG{o}{=} \PYG{p}{\PYGZob{}}\PYG{l+s+s2}{\PYGZdq{}}\PYG{l+s+s2}{nome}\PYG{l+s+s2}{\PYGZdq{}}\PYG{p}{:} \PYG{l+s+s2}{\PYGZdq{}}\PYG{l+s+s2}{Fernanda}\PYG{l+s+s2}{\PYGZdq{}}\PYG{p}{,} \PYG{l+s+s2}{\PYGZdq{}}\PYG{l+s+s2}{idade}\PYG{l+s+s2}{\PYGZdq{}}\PYG{p}{:} \PYG{l+m+mi}{29}\PYG{p}{\PYGZcb{}}
\PYG{n}{ultimo\PYGZus{}item} \PYG{o}{=} \PYG{n}{meu\PYGZus{}dicionario}\PYG{o}{.}\PYG{n}{popitem}\PYG{p}{(}\PYG{p}{)}
\PYG{n+nb}{print}\PYG{p}{(}\PYG{n}{ultimo\PYGZus{}item}\PYG{p}{)}
\end{sphinxVerbatim}

\end{sphinxuseclass}\end{sphinxVerbatimInput}
\begin{sphinxVerbatimOutput}

\begin{sphinxuseclass}{cell_output}
\begin{sphinxVerbatim}[commandchars=\\\{\}]
(\PYGZsq{}idade\PYGZsq{}, 29)
\end{sphinxVerbatim}

\end{sphinxuseclass}\end{sphinxVerbatimOutput}

\end{sphinxuseclass}
\sphinxAtStartPar
\sphinxstylestrong{\sphinxcode{\sphinxupquote{update(dicionario)}}:}
Atualiza o dicionário com pares chave\sphinxhyphen{}valor de outro dicionário ou iterável.

\begin{sphinxuseclass}{cell}\begin{sphinxVerbatimInput}

\begin{sphinxuseclass}{cell_input}
\begin{sphinxVerbatim}[commandchars=\\\{\}]
\PYG{n}{meu\PYGZus{}dicionario} \PYG{o}{=} \PYG{p}{\PYGZob{}}\PYG{l+s+s2}{\PYGZdq{}}\PYG{l+s+s2}{nome}\PYG{l+s+s2}{\PYGZdq{}}\PYG{p}{:} \PYG{l+s+s2}{\PYGZdq{}}\PYG{l+s+s2}{Rafael}\PYG{l+s+s2}{\PYGZdq{}}\PYG{p}{,} \PYG{l+s+s2}{\PYGZdq{}}\PYG{l+s+s2}{idade}\PYG{l+s+s2}{\PYGZdq{}}\PYG{p}{:} \PYG{l+m+mi}{26}\PYG{p}{\PYGZcb{}}
\PYG{n}{outro\PYGZus{}dicionario} \PYG{o}{=} \PYG{p}{\PYGZob{}}\PYG{l+s+s2}{\PYGZdq{}}\PYG{l+s+s2}{cidade}\PYG{l+s+s2}{\PYGZdq{}}\PYG{p}{:} \PYG{l+s+s2}{\PYGZdq{}}\PYG{l+s+s2}{São Paulo}\PYG{l+s+s2}{\PYGZdq{}}\PYG{p}{\PYGZcb{}}
\PYG{n}{meu\PYGZus{}dicionario}\PYG{o}{.}\PYG{n}{update}\PYG{p}{(}\PYG{n}{outro\PYGZus{}dicionario}\PYG{p}{)}
\PYG{n+nb}{print}\PYG{p}{(}\PYG{n}{meu\PYGZus{}dicionario}\PYG{p}{)}
\end{sphinxVerbatim}

\end{sphinxuseclass}\end{sphinxVerbatimInput}
\begin{sphinxVerbatimOutput}

\begin{sphinxuseclass}{cell_output}
\begin{sphinxVerbatim}[commandchars=\\\{\}]
\PYGZob{}\PYGZsq{}nome\PYGZsq{}: \PYGZsq{}Rafael\PYGZsq{}, \PYGZsq{}idade\PYGZsq{}: 26, \PYGZsq{}cidade\PYGZsq{}: \PYGZsq{}São Paulo\PYGZsq{}\PYGZcb{}
\end{sphinxVerbatim}

\end{sphinxuseclass}\end{sphinxVerbatimOutput}

\end{sphinxuseclass}
\sphinxAtStartPar
Estes são apenas alguns dos métodos disponíveis para dicionários em Python. A escolha do método dependerá da operação específica que você deseja realizar.


\subsection{Conjuntos}
\label{\detokenize{chapters/ch2/ch2:conjuntos}}
\sphinxAtStartPar
Os conjuntos são uma estrutura de dados em Python que permite armazenar uma coleção de elementos únicos. Isso significa que, cada elemento de um conjunto deve ser diferente de todos os outros elementos do conjunto.

\sphinxAtStartPar
Os conjuntos são declarados usando chaves, com os elementos separados por vírgulas. Por exemplo, a seguinte declaração cria um conjunto com três elementos:

\begin{sphinxVerbatim}[commandchars=\\\{\}]
\PYG{n}{conjunto} \PYG{o}{=} \PYG{p}{\PYGZob{}}\PYG{l+m+mi}{1}\PYG{p}{,} \PYG{l+m+mi}{2}\PYG{p}{,} \PYG{l+m+mi}{3}\PYG{p}{\PYGZcb{}}
\end{sphinxVerbatim}

\sphinxAtStartPar
Os elementos dos conjuntos podem ser de qualquer tipo de dados, incluindo inteiros, reais, strings, booleanos, etc.
Aqui estão alguns exemplos de uso de conjuntos em Python:

\begin{sphinxuseclass}{cell}\begin{sphinxVerbatimInput}

\begin{sphinxuseclass}{cell_input}
\begin{sphinxVerbatim}[commandchars=\\\{\}]
\PYG{c+c1}{\PYGZsh{} Declaração de um conjunto com três elementos}
\PYG{n}{conjunto} \PYG{o}{=} \PYG{p}{\PYGZob{}}\PYG{l+m+mi}{1}\PYG{p}{,} \PYG{l+m+mi}{2}\PYG{p}{,} \PYG{l+m+mi}{3}\PYG{p}{\PYGZcb{}}

\PYG{c+c1}{\PYGZsh{} Adição de um elemento}
\PYG{n}{conjunto}\PYG{o}{.}\PYG{n}{add}\PYG{p}{(}\PYG{l+m+mi}{4}\PYG{p}{)}

\PYG{c+c1}{\PYGZsh{} Remoção de um elemento}
\PYG{n}{conjunto}\PYG{o}{.}\PYG{n}{remove}\PYG{p}{(}\PYG{l+m+mi}{2}\PYG{p}{)}

\PYG{n+nb}{print}\PYG{p}{(}\PYG{n}{conjunto}\PYG{p}{)}
\end{sphinxVerbatim}

\end{sphinxuseclass}\end{sphinxVerbatimInput}
\begin{sphinxVerbatimOutput}

\begin{sphinxuseclass}{cell_output}
\begin{sphinxVerbatim}[commandchars=\\\{\}]
\PYGZob{}1, 3, 4\PYGZcb{}
\end{sphinxVerbatim}

\end{sphinxuseclass}\end{sphinxVerbatimOutput}

\end{sphinxuseclass}
\begin{sphinxuseclass}{cell}\begin{sphinxVerbatimInput}

\begin{sphinxuseclass}{cell_input}
\begin{sphinxVerbatim}[commandchars=\\\{\}]
\PYG{c+c1}{\PYGZsh{} União de conjuntos}
\PYG{n}{conjunto\PYGZus{}1} \PYG{o}{=} \PYG{p}{\PYGZob{}}\PYG{l+m+mi}{1}\PYG{p}{,} \PYG{l+m+mi}{2}\PYG{p}{,} \PYG{l+m+mi}{3}\PYG{p}{\PYGZcb{}}
\PYG{n}{conjunto\PYGZus{}2} \PYG{o}{=} \PYG{p}{\PYGZob{}}\PYG{l+m+mi}{4}\PYG{p}{,} \PYG{l+m+mi}{5}\PYG{p}{,} \PYG{l+m+mi}{6}\PYG{p}{\PYGZcb{}}
\PYG{n}{conjunto\PYGZus{}uniao} \PYG{o}{=} \PYG{n}{conjunto\PYGZus{}1} \PYG{o}{|} \PYG{n}{conjunto\PYGZus{}2}
\PYG{n+nb}{print}\PYG{p}{(}\PYG{n}{conjunto\PYGZus{}uniao}\PYG{p}{)}

\PYG{c+c1}{\PYGZsh{} Interseção de conjuntos}
\PYG{n}{conjunto\PYGZus{}1} \PYG{o}{=} \PYG{p}{\PYGZob{}}\PYG{l+m+mi}{1}\PYG{p}{,} \PYG{l+m+mi}{2}\PYG{p}{,} \PYG{l+m+mi}{3}\PYG{p}{\PYGZcb{}}
\PYG{n}{conjunto\PYGZus{}2} \PYG{o}{=} \PYG{p}{\PYGZob{}}\PYG{l+m+mi}{2}\PYG{p}{,} \PYG{l+m+mi}{3}\PYG{p}{,} \PYG{l+m+mi}{4}\PYG{p}{\PYGZcb{}}
\PYG{n}{conjunto\PYGZus{}intersecao} \PYG{o}{=} \PYG{n}{conjunto\PYGZus{}1} \PYG{o}{\PYGZam{}} \PYG{n}{conjunto\PYGZus{}2}
\PYG{n+nb}{print}\PYG{p}{(}\PYG{n}{conjunto\PYGZus{}intersecao}\PYG{p}{)}

\PYG{c+c1}{\PYGZsh{} Diferença de conjuntos}
\PYG{n}{conjunto\PYGZus{}1} \PYG{o}{=} \PYG{p}{\PYGZob{}}\PYG{l+m+mi}{1}\PYG{p}{,} \PYG{l+m+mi}{2}\PYG{p}{,} \PYG{l+m+mi}{3}\PYG{p}{\PYGZcb{}}
\PYG{n}{conjunto\PYGZus{}2} \PYG{o}{=} \PYG{p}{\PYGZob{}}\PYG{l+m+mi}{2}\PYG{p}{,} \PYG{l+m+mi}{3}\PYG{p}{,} \PYG{l+m+mi}{4}\PYG{p}{\PYGZcb{}}
\PYG{n}{conjunto\PYGZus{}diferenca} \PYG{o}{=} \PYG{n}{conjunto\PYGZus{}1} \PYG{o}{\PYGZhy{}} \PYG{n}{conjunto\PYGZus{}2}
\PYG{n+nb}{print}\PYG{p}{(}\PYG{n}{conjunto\PYGZus{}diferenca}\PYG{p}{)}
\end{sphinxVerbatim}

\end{sphinxuseclass}\end{sphinxVerbatimInput}
\begin{sphinxVerbatimOutput}

\begin{sphinxuseclass}{cell_output}
\begin{sphinxVerbatim}[commandchars=\\\{\}]
\PYGZob{}1, 2, 3, 4, 5, 6\PYGZcb{}
\PYGZob{}2, 3\PYGZcb{}
\PYGZob{}1\PYGZcb{}
\end{sphinxVerbatim}

\end{sphinxuseclass}\end{sphinxVerbatimOutput}

\end{sphinxuseclass}
\sphinxAtStartPar
Os conjuntos são um tipo de dados que pode ser usado para representar uma ampla gama de dados. Por exemplo, podemos usar conjuntos para representar coleções de números, letras, etc.

\sphinxAtStartPar
Aqui estão alguns exemplos específicos de uso de conjuntos em Python. Para representar uma coleção de números primos, podemos usar um conjunto:

\begin{sphinxVerbatim}[commandchars=\\\{\}]
\PYG{n}{primos} \PYG{o}{=} \PYG{p}{\PYGZob{}}\PYG{l+m+mi}{2}\PYG{p}{,} \PYG{l+m+mi}{3}\PYG{p}{,} \PYG{l+m+mi}{5}\PYG{p}{,} \PYG{l+m+mi}{7}\PYG{p}{,} \PYG{l+m+mi}{11}\PYG{p}{,} \PYG{l+m+mi}{13}\PYG{p}{,} \PYG{l+m+mi}{17}\PYG{p}{,} \PYG{l+m+mi}{19}\PYG{p}{\PYGZcb{}}
\end{sphinxVerbatim}

\sphinxAtStartPar
Para representar uma coleção de letras do alfabeto, podemos usar um conjunto:

\begin{sphinxVerbatim}[commandchars=\\\{\}]
\PYG{n}{letras} \PYG{o}{=} \PYG{p}{\PYGZob{}}\PYG{l+s+s2}{\PYGZdq{}}\PYG{l+s+s2}{a}\PYG{l+s+s2}{\PYGZdq{}}\PYG{p}{,} \PYG{l+s+s2}{\PYGZdq{}}\PYG{l+s+s2}{b}\PYG{l+s+s2}{\PYGZdq{}}\PYG{p}{,} \PYG{l+s+s2}{\PYGZdq{}}\PYG{l+s+s2}{c}\PYG{l+s+s2}{\PYGZdq{}}\PYG{p}{,} \PYG{l+s+s2}{\PYGZdq{}}\PYG{l+s+s2}{d}\PYG{l+s+s2}{\PYGZdq{}}\PYG{p}{,} \PYG{l+s+s2}{\PYGZdq{}}\PYG{l+s+s2}{e}\PYG{l+s+s2}{\PYGZdq{}}\PYG{p}{,} \PYG{l+s+s2}{\PYGZdq{}}\PYG{l+s+s2}{f}\PYG{l+s+s2}{\PYGZdq{}}\PYG{p}{,} \PYG{l+s+s2}{\PYGZdq{}}\PYG{l+s+s2}{g}\PYG{l+s+s2}{\PYGZdq{}}\PYG{p}{,} \PYG{l+s+s2}{\PYGZdq{}}\PYG{l+s+s2}{h}\PYG{l+s+s2}{\PYGZdq{}}\PYG{p}{,} \PYG{l+s+s2}{\PYGZdq{}}\PYG{l+s+s2}{i}\PYG{l+s+s2}{\PYGZdq{}}\PYG{p}{,} \PYG{l+s+s2}{\PYGZdq{}}\PYG{l+s+s2}{j}\PYG{l+s+s2}{\PYGZdq{}}\PYG{p}{,} \PYG{l+s+s2}{\PYGZdq{}}\PYG{l+s+s2}{k}\PYG{l+s+s2}{\PYGZdq{}}\PYG{p}{,} \PYG{l+s+s2}{\PYGZdq{}}\PYG{l+s+s2}{l}\PYG{l+s+s2}{\PYGZdq{}}\PYG{p}{,} \PYG{l+s+s2}{\PYGZdq{}}\PYG{l+s+s2}{m}\PYG{l+s+s2}{\PYGZdq{}}\PYG{p}{,} \PYG{l+s+s2}{\PYGZdq{}}\PYG{l+s+s2}{n}\PYG{l+s+s2}{\PYGZdq{}}\PYG{p}{,} \PYG{l+s+s2}{\PYGZdq{}}\PYG{l+s+s2}{o}\PYG{l+s+s2}{\PYGZdq{}}\PYG{p}{,} \PYG{l+s+s2}{\PYGZdq{}}\PYG{l+s+s2}{p}\PYG{l+s+s2}{\PYGZdq{}}\PYG{p}{,} \PYG{l+s+s2}{\PYGZdq{}}\PYG{l+s+s2}{q}\PYG{l+s+s2}{\PYGZdq{}}\PYG{p}{,} \PYG{l+s+s2}{\PYGZdq{}}\PYG{l+s+s2}{r}\PYG{l+s+s2}{\PYGZdq{}}\PYG{p}{,} \PYG{l+s+s2}{\PYGZdq{}}\PYG{l+s+s2}{s}\PYG{l+s+s2}{\PYGZdq{}}\PYG{p}{,} \PYG{l+s+s2}{\PYGZdq{}}\PYG{l+s+s2}{t}\PYG{l+s+s2}{\PYGZdq{}}\PYG{p}{,} \PYG{l+s+s2}{\PYGZdq{}}\PYG{l+s+s2}{u}\PYG{l+s+s2}{\PYGZdq{}}\PYG{p}{,} \PYG{l+s+s2}{\PYGZdq{}}\PYG{l+s+s2}{v}\PYG{l+s+s2}{\PYGZdq{}}\PYG{p}{,} \PYG{l+s+s2}{\PYGZdq{}}\PYG{l+s+s2}{w}\PYG{l+s+s2}{\PYGZdq{}}\PYG{p}{,} \PYG{l+s+s2}{\PYGZdq{}}\PYG{l+s+s2}{x}\PYG{l+s+s2}{\PYGZdq{}}\PYG{p}{,} \PYG{l+s+s2}{\PYGZdq{}}\PYG{l+s+s2}{y}\PYG{l+s+s2}{\PYGZdq{}}\PYG{p}{,} \PYG{l+s+s2}{\PYGZdq{}}\PYG{l+s+s2}{z}\PYG{l+s+s2}{\PYGZdq{}}\PYG{p}{\PYGZcb{}}
\end{sphinxVerbatim}

\sphinxAtStartPar
Para representar uma coleção de cores, podemos usar um conjunto:

\begin{sphinxVerbatim}[commandchars=\\\{\}]
\PYG{n}{cores} \PYG{o}{=} \PYG{p}{\PYGZob{}}\PYG{l+s+s2}{\PYGZdq{}}\PYG{l+s+s2}{vermelho}\PYG{l+s+s2}{\PYGZdq{}}\PYG{p}{,} \PYG{l+s+s2}{\PYGZdq{}}\PYG{l+s+s2}{verde}\PYG{l+s+s2}{\PYGZdq{}}\PYG{p}{,} \PYG{l+s+s2}{\PYGZdq{}}\PYG{l+s+s2}{amarelo}\PYG{l+s+s2}{\PYGZdq{}}\PYG{p}{,} \PYG{l+s+s2}{\PYGZdq{}}\PYG{l+s+s2}{azul}\PYG{l+s+s2}{\PYGZdq{}}\PYG{p}{,} \PYG{l+s+s2}{\PYGZdq{}}\PYG{l+s+s2}{roxo}\PYG{l+s+s2}{\PYGZdq{}}\PYG{p}{,} \PYG{l+s+s2}{\PYGZdq{}}\PYG{l+s+s2}{preto}\PYG{l+s+s2}{\PYGZdq{}}\PYG{p}{,} \PYG{l+s+s2}{\PYGZdq{}}\PYG{l+s+s2}{branco}\PYG{l+s+s2}{\PYGZdq{}}\PYG{p}{\PYGZcb{}}
\end{sphinxVerbatim}

\sphinxAtStartPar
\sphinxstylestrong{Exemplos de nomes didáticos para declarações de conjuntos:}

\begin{sphinxVerbatim}[commandchars=\\\{\}]
\PYG{c+c1}{\PYGZsh{} Declaração de um conjunto com três elementos}
\PYG{n}{conjunto\PYGZus{}de\PYGZus{}numeros} \PYG{o}{=} \PYG{p}{\PYGZob{}}\PYG{l+m+mi}{1}\PYG{p}{,} \PYG{l+m+mi}{2}\PYG{p}{,} \PYG{l+m+mi}{3}\PYG{p}{\PYGZcb{}}

\PYG{c+c1}{\PYGZsh{} Declaração de um conjunto com as cores do arco\PYGZhy{}íris}
\PYG{n}{conjunto\PYGZus{}de\PYGZus{}cores} \PYG{o}{=} \PYG{p}{\PYGZob{}}\PYG{l+s+s2}{\PYGZdq{}}\PYG{l+s+s2}{vermelho}\PYG{l+s+s2}{\PYGZdq{}}\PYG{p}{,} \PYG{l+s+s2}{\PYGZdq{}}\PYG{l+s+s2}{laranja}\PYG{l+s+s2}{\PYGZdq{}}\PYG{p}{,} \PYG{l+s+s2}{\PYGZdq{}}\PYG{l+s+s2}{amarelo}\PYG{l+s+s2}{\PYGZdq{}}\PYG{p}{,} \PYG{l+s+s2}{\PYGZdq{}}\PYG{l+s+s2}{verde}\PYG{l+s+s2}{\PYGZdq{}}\PYG{p}{,} \PYG{l+s+s2}{\PYGZdq{}}\PYG{l+s+s2}{azul}\PYG{l+s+s2}{\PYGZdq{}}\PYG{p}{,} \PYG{l+s+s2}{\PYGZdq{}}\PYG{l+s+s2}{anil}\PYG{l+s+s2}{\PYGZdq{}}\PYG{p}{,} \PYG{l+s+s2}{\PYGZdq{}}\PYG{l+s+s2}{violeta}\PYG{l+s+s2}{\PYGZdq{}}\PYG{p}{\PYGZcb{}}
\end{sphinxVerbatim}

\sphinxAtStartPar
A escolha do nome ideal para um conjunto depende do contexto em que ele será usado.


\section{Exercícios}
\label{\detokenize{chapters/ch2/ch2:exercicios}}\begin{enumerate}
\sphinxsetlistlabels{\arabic}{enumi}{enumii}{}{.}%
\item {} 
\sphinxAtStartPar
Declare duas variáveis, uma do tipo int e
outra do tipo float. Em seguida, atribuir a elas valores e imprimir o resultado da soma, da subtração, da multiplicação e da divisão entre as duas variáveis.

\item {} 
\sphinxAtStartPar
Declarar uma variável do tipo str e atribuir a ela uma string. Em seguida, imprimir o comprimento da string, a primeira letra da string, a última letra da string e a string invertida.

\item {} 
\sphinxAtStartPar
Declare uma variável do tipo bool e atribuir a ela um valor. Em seguida, imprimir o resultado da operação de negação sobre a variável.

\item {} 
\sphinxAtStartPar
Faça um programa que calcule o Índice de Massa Corporal (IMC) de uma pessoa. O IMC é calculado dividindo\sphinxhyphen{}se o peso da pessoa pela sua altura ao quadrado. O IMC é uma medida da relação entre o peso e a altura de uma pessoa.
O programa deve imprimir o IMC da pessoa, classificando\sphinxhyphen{}o de acordo com a tabela abaixo:

\end{enumerate}


\begin{savenotes}\sphinxattablestart
\centering
\begin{tabulary}{\linewidth}[t]{|T|T|}
\hline
\sphinxstyletheadfamily 
\sphinxAtStartPar
IMC
&\sphinxstyletheadfamily 
\sphinxAtStartPar
Classificação
\\
\hline
\sphinxAtStartPar
< 18,5
&
\sphinxAtStartPar
Abaixo do peso
\\
\hline
\sphinxAtStartPar
18,5 \sphinxhyphen{} 24,9
&
\sphinxAtStartPar
Saudável
\\
\hline
\sphinxAtStartPar
25,0 \sphinxhyphen{} 29,9
&
\sphinxAtStartPar
Sobrepeso
\\
\hline
\sphinxAtStartPar
30,0 \sphinxhyphen{} 34,9
&
\sphinxAtStartPar
Obesidade grau I
\\
\hline
\sphinxAtStartPar
35,0 \sphinxhyphen{} 39,9
&
\sphinxAtStartPar
Obesidade grau II
\\
\hline
\end{tabulary}
\par
\sphinxattableend\end{savenotes}
\begin{quote}

\sphinxAtStartPar
= 40,0 | Obesidade grau III
\end{quote}
\begin{enumerate}
\sphinxsetlistlabels{\arabic}{enumi}{enumii}{}{.}%
\setcounter{enumi}{4}
\item {} 
\sphinxAtStartPar
Crie uma lista chamada \sphinxcode{\sphinxupquote{frutas}} com as seguintes frutas: maçã, banana, laranja, pera e melancia. Em seguida, imprima a lista.

\item {} 
\sphinxAtStartPar
Criar uma tupla chamada \sphinxcode{\sphinxupquote{coordenadas}} com as coordenadas (2, 3). Em seguida, imprima as coordenadas.

\item {} 
\sphinxAtStartPar
Crie um dicionário chamado contato. Peça ao usuário para fornecer os dados para as chaves “nome”, “telefone” e “endereço”. Em seguida, imprima o conteúdo do dicionário.

\end{enumerate}

\sphinxAtStartPar
Exemplo de execução:

\begin{sphinxVerbatim}[commandchars=\\\{\}]
\PYG{n}{Digite} \PYG{n}{o} \PYG{n}{nome} \PYG{n}{do} \PYG{n}{contato}\PYG{p}{:} \PYG{n}{Python} \PYG{n}{da} \PYG{n}{Silva}
\PYG{n}{Digite} \PYG{n}{o} \PYG{n}{telefone} \PYG{n}{do} \PYG{n}{contato}\PYG{p}{:} \PYG{p}{(}\PYG{l+m+mi}{84}\PYG{p}{)} \PYG{l+m+mi}{9999}\PYG{o}{\PYGZhy{}}\PYG{l+m+mi}{9999}
\PYG{n}{Digite} \PYG{n}{o} \PYG{n}{endereço} \PYG{n}{do} \PYG{n}{contato}\PYG{p}{:} \PYG{n}{Rua} \PYG{n}{da} \PYG{n}{Programação}\PYG{p}{,} \PYG{l+m+mi}{123}

\PYG{n}{Nome}\PYG{p}{:} \PYG{n}{Python} \PYG{n}{da} \PYG{n}{Silva}\PYG{p}{,} \PYG{n}{Telefone}\PYG{p}{:} \PYG{p}{(}\PYG{l+m+mi}{84}\PYG{p}{)} \PYG{l+m+mi}{9999}\PYG{o}{\PYGZhy{}}\PYG{l+m+mi}{9999}\PYG{p}{,} \PYG{n}{Endereço}\PYG{p}{:} \PYG{n}{Rua} \PYG{n}{da} \PYG{n}{Programação}\PYG{p}{,} \PYG{l+m+mf}{123.}
\end{sphinxVerbatim}
\begin{enumerate}
\sphinxsetlistlabels{\arabic}{enumi}{enumii}{}{.}%
\setcounter{enumi}{7}
\item {} 
\sphinxAtStartPar
Crie um conjunto chamado \sphinxcode{\sphinxupquote{numeros}} com os números 1, 2, 3, 4 e 5. Imprima o conjunto.

\item {} 
\sphinxAtStartPar
Criar dois conjuntos, \sphinxcode{\sphinxupquote{conjunto\_a}} e \sphinxcode{\sphinxupquote{conjunto\_b}}, com alguns números. Realizar as seguintes operações e imprimir os resultados:
\begin{itemize}
\item {} 
\sphinxAtStartPar
União: Combinar os elementos únicos de ambos os conjuntos.

\item {} 
\sphinxAtStartPar
Interseção: Encontrar os elementos que estão presentes em ambos os conjuntos.

\item {} 
\sphinxAtStartPar
Diferença: Identificar os elementos que estão em \sphinxcode{\sphinxupquote{conjunto\_a}} mas não em \sphinxcode{\sphinxupquote{conjunto\_b}}.

\end{itemize}

\item {} 
\sphinxAtStartPar
Criar uma string original, por exemplo, “Python é uma linguagem de programação poderosa!”. Realizar as seguintes operações e imprimir os resultados:

\end{enumerate}
\begin{itemize}
\item {} 
\sphinxAtStartPar
Transformar em maiúsculas: Converter toda a string para letras maiúsculas.

\item {} 
\sphinxAtStartPar
Transformar em minúsculas: Converter toda a string para letras minúsculas.

\item {} 
\sphinxAtStartPar
Substituir parte da string: Substituir uma parte específica da string por outra (por exemplo, substituir “poderosa” por “versátil”).

\end{itemize}

\sphinxstepscope


\chapter{Capítulo 3: Operadores e expressões}
\label{\detokenize{chapters/ch3/ch3:capitulo-3-operadores-e-expressoes}}\label{\detokenize{chapters/ch3/ch3::doc}}
\sphinxAtStartPar
Compreender os operadores e expressões em Python é essencial para capacitar programadores na manipulação eficiente de dados e na execução de operações fundamentais. Esses operadores, que incluem aritméticos, relacionais, lógicos e de atribuição, desempenham papéis específicos, permitindo desde cálculos matemáticos até avaliações condicionais. O domínio desses conceitos não apenas facilita a escrita de códigos mais eficazes, mas também é vital para a resolução de problemas complexos e a compreensão de códigos existentes. Em resumo, o conhecimento aprofundado dos operadores e expressões em Python é uma habilidade fundamental para o desenvolvimento eficiente de algoritmos e lógica de programação.


\section{Operadores Aritméticos}
\label{\detokenize{chapters/ch3/ch3:operadores-aritmeticos}}\begin{itemize}
\item {} 
\sphinxAtStartPar
\sphinxcode{\sphinxupquote{+}} (adição): Soma dois valores.

\item {} 
\sphinxAtStartPar
\sphinxcode{\sphinxupquote{\sphinxhyphen{}}} (subtração): Subtrai o operando direito do operando esquerdo.

\item {} 
\sphinxAtStartPar
\sphinxcode{\sphinxupquote{*}} (multiplicação): Multiplica dois valores.

\item {} 
\sphinxAtStartPar
\sphinxcode{\sphinxupquote{/}} (divisão): Divide o operando esquerdo pelo operando direito.

\item {} 
\sphinxAtStartPar
\sphinxcode{\sphinxupquote{\%}} (módulo): Retorna o resto da divisão do operando esquerdo pelo operando direito.

\item {} 
\sphinxAtStartPar
\sphinxcode{\sphinxupquote{//}} (divisão de piso): Retorna a parte inteira da divisão do operando esquerdo pelo operando direito.

\item {} 
\sphinxAtStartPar
\sphinxcode{\sphinxupquote{**}} (potência): Eleva o operando esquerdo à potência do operando direito.

\end{itemize}

\begin{sphinxuseclass}{cell}\begin{sphinxVerbatimInput}

\begin{sphinxuseclass}{cell_input}
\begin{sphinxVerbatim}[commandchars=\\\{\}]
\PYG{c+c1}{\PYGZsh{} Operadores Aritméticos em Python}

\PYG{c+c1}{\PYGZsh{} Adição (+): Soma dois valores.}
\PYG{n}{soma} \PYG{o}{=} \PYG{l+m+mi}{5} \PYG{o}{+} \PYG{l+m+mi}{3}
\PYG{n+nb}{print}\PYG{p}{(}\PYG{l+s+s2}{\PYGZdq{}}\PYG{l+s+s2}{Adição:}\PYG{l+s+s2}{\PYGZdq{}}\PYG{p}{,} \PYG{n}{soma}\PYG{p}{)}

\PYG{c+c1}{\PYGZsh{} Subtração (\PYGZhy{}): Subtrai o operando direito do operando esquerdo.}
\PYG{n}{subtracao} \PYG{o}{=} \PYG{l+m+mi}{7} \PYG{o}{\PYGZhy{}} \PYG{l+m+mi}{4}
\PYG{n+nb}{print}\PYG{p}{(}\PYG{l+s+s2}{\PYGZdq{}}\PYG{l+s+s2}{Subtração:}\PYG{l+s+s2}{\PYGZdq{}}\PYG{p}{,} \PYG{n}{subtracao}\PYG{p}{)}

\PYG{c+c1}{\PYGZsh{} Multiplicação (*): Multiplica dois valores.}
\PYG{n}{multiplicacao} \PYG{o}{=} \PYG{l+m+mi}{6} \PYG{o}{*} \PYG{l+m+mi}{2}
\PYG{n+nb}{print}\PYG{p}{(}\PYG{l+s+s2}{\PYGZdq{}}\PYG{l+s+s2}{Multiplicação:}\PYG{l+s+s2}{\PYGZdq{}}\PYG{p}{,} \PYG{n}{multiplicacao}\PYG{p}{)}

\PYG{c+c1}{\PYGZsh{} Divisão (/): Divide o operando esquerdo pelo operando direito.}
\PYG{n}{divisao} \PYG{o}{=} \PYG{l+m+mi}{10} \PYG{o}{/} \PYG{l+m+mi}{3}
\PYG{n+nb}{print}\PYG{p}{(}\PYG{l+s+s2}{\PYGZdq{}}\PYG{l+s+s2}{Divisão:}\PYG{l+s+s2}{\PYGZdq{}}\PYG{p}{,} \PYG{n}{divisao}\PYG{p}{)}

\PYG{c+c1}{\PYGZsh{} Módulo (\PYGZpc{}): Retorna o resto da divisão do operando esquerdo pelo operando direito.}
\PYG{n}{modulo} \PYG{o}{=} \PYG{l+m+mi}{15} \PYG{o}{\PYGZpc{}} \PYG{l+m+mi}{4}
\PYG{n+nb}{print}\PYG{p}{(}\PYG{l+s+s2}{\PYGZdq{}}\PYG{l+s+s2}{Módulo:}\PYG{l+s+s2}{\PYGZdq{}}\PYG{p}{,} \PYG{n}{modulo}\PYG{p}{)}

\PYG{c+c1}{\PYGZsh{} Divisão de Piso (//): Retorna a parte inteira da divisão do operando esquerdo pelo operando direito.}
\PYG{n}{divisao\PYGZus{}piso} \PYG{o}{=} \PYG{l+m+mi}{20} \PYG{o}{/}\PYG{o}{/} \PYG{l+m+mi}{3}
\PYG{n+nb}{print}\PYG{p}{(}\PYG{l+s+s2}{\PYGZdq{}}\PYG{l+s+s2}{Divisão de Piso:}\PYG{l+s+s2}{\PYGZdq{}}\PYG{p}{,} \PYG{n}{divisao\PYGZus{}piso}\PYG{p}{)}

\PYG{c+c1}{\PYGZsh{} Potência (**): Eleva o operando esquerdo à potência do operando direito.}
\PYG{n}{potencia} \PYG{o}{=} \PYG{l+m+mi}{2} \PYG{o}{*}\PYG{o}{*} \PYG{l+m+mi}{4}
\PYG{n+nb}{print}\PYG{p}{(}\PYG{l+s+s2}{\PYGZdq{}}\PYG{l+s+s2}{Potência:}\PYG{l+s+s2}{\PYGZdq{}}\PYG{p}{,} \PYG{n}{potencia}\PYG{p}{)}
\end{sphinxVerbatim}

\end{sphinxuseclass}\end{sphinxVerbatimInput}
\begin{sphinxVerbatimOutput}

\begin{sphinxuseclass}{cell_output}
\begin{sphinxVerbatim}[commandchars=\\\{\}]
Adição: 8
Subtração: 3
Multiplicação: 12
Divisão: 3.3333333333333335
Módulo: 3
Divisão de Piso: 6
Potência: 16
\end{sphinxVerbatim}

\end{sphinxuseclass}\end{sphinxVerbatimOutput}

\end{sphinxuseclass}

\section{Operadores Relacionais}
\label{\detokenize{chapters/ch3/ch3:operadores-relacionais}}\begin{itemize}
\item {} 
\sphinxAtStartPar
\sphinxcode{\sphinxupquote{==}} (igual a): Retorna True se os operandos forem iguais.

\item {} 
\sphinxAtStartPar
\sphinxcode{\sphinxupquote{!=}} (diferente de): Retorna True se os operandos não forem iguais.

\item {} 
\sphinxAtStartPar
\sphinxcode{\sphinxupquote{<}} (menor que): Retorna True se o operando esquerdo for menor que o operando direito.

\item {} 
\sphinxAtStartPar
\sphinxcode{\sphinxupquote{>}} (maior que): Retorna True se o operando esquerdo for maior que o operando direito.

\item {} 
\sphinxAtStartPar
\sphinxcode{\sphinxupquote{<=}} (menor ou igual a): Retorna True se o operando esquerdo for menor ou igual ao operando direito.

\item {} 
\sphinxAtStartPar
\sphinxcode{\sphinxupquote{>=}} (maior ou igual a): Retorna True se o operando esquerdo for maior ou igual ao operando direito.

\end{itemize}

\begin{sphinxuseclass}{cell}\begin{sphinxVerbatimInput}

\begin{sphinxuseclass}{cell_input}
\begin{sphinxVerbatim}[commandchars=\\\{\}]
\PYG{c+c1}{\PYGZsh{} Operadores Relacionais em Python}

\PYG{c+c1}{\PYGZsh{} Igual a (==): Retorna True se os operandos forem iguais.}
\PYG{n}{igual\PYGZus{}a} \PYG{o}{=} \PYG{l+m+mi}{5} \PYG{o}{==} \PYG{l+m+mi}{5}
\PYG{n+nb}{print}\PYG{p}{(}\PYG{l+s+s2}{\PYGZdq{}}\PYG{l+s+s2}{Igual a:}\PYG{l+s+s2}{\PYGZdq{}}\PYG{p}{,} \PYG{n}{igual\PYGZus{}a}\PYG{p}{)}

\PYG{c+c1}{\PYGZsh{} Diferente de (!=): Retorna True se os operandos não forem iguais.}
\PYG{n}{diferente\PYGZus{}de} \PYG{o}{=} \PYG{l+m+mi}{6} \PYG{o}{!=} \PYG{l+m+mi}{5}
\PYG{n+nb}{print}\PYG{p}{(}\PYG{l+s+s2}{\PYGZdq{}}\PYG{l+s+s2}{Diferente de:}\PYG{l+s+s2}{\PYGZdq{}}\PYG{p}{,} \PYG{n}{diferente\PYGZus{}de}\PYG{p}{)}

\PYG{c+c1}{\PYGZsh{} Menor que (\PYGZlt{}): Retorna True se o operando esquerdo for menor que o operando direito.}
\PYG{n}{menor\PYGZus{}que} \PYG{o}{=} \PYG{l+m+mi}{4} \PYG{o}{\PYGZlt{}} \PYG{l+m+mi}{7}
\PYG{n+nb}{print}\PYG{p}{(}\PYG{l+s+s2}{\PYGZdq{}}\PYG{l+s+s2}{Menor que:}\PYG{l+s+s2}{\PYGZdq{}}\PYG{p}{,} \PYG{n}{menor\PYGZus{}que}\PYG{p}{)}

\PYG{c+c1}{\PYGZsh{} Maior que (\PYGZgt{}): Retorna True se o operando esquerdo for maior que o operando direito.}
\PYG{n}{maior\PYGZus{}que} \PYG{o}{=} \PYG{l+m+mi}{8} \PYG{o}{\PYGZgt{}} \PYG{l+m+mi}{5}
\PYG{n+nb}{print}\PYG{p}{(}\PYG{l+s+s2}{\PYGZdq{}}\PYG{l+s+s2}{Maior que:}\PYG{l+s+s2}{\PYGZdq{}}\PYG{p}{,} \PYG{n}{maior\PYGZus{}que}\PYG{p}{)}

\PYG{c+c1}{\PYGZsh{} Menor ou igual a (\PYGZlt{}=): Retorna True se o operando esquerdo for menor ou igual ao operando direito.}
\PYG{n}{menor\PYGZus{}ou\PYGZus{}igual\PYGZus{}a} \PYG{o}{=} \PYG{l+m+mi}{5} \PYG{o}{\PYGZlt{}}\PYG{o}{=} \PYG{l+m+mi}{5}
\PYG{n+nb}{print}\PYG{p}{(}\PYG{l+s+s2}{\PYGZdq{}}\PYG{l+s+s2}{Menor ou igual a:}\PYG{l+s+s2}{\PYGZdq{}}\PYG{p}{,} \PYG{n}{menor\PYGZus{}ou\PYGZus{}igual\PYGZus{}a}\PYG{p}{)}

\PYG{c+c1}{\PYGZsh{} Maior ou igual a (\PYGZgt{}=): Retorna True se o operando esquerdo for maior ou igual ao operando direito.}
\PYG{n}{maior\PYGZus{}ou\PYGZus{}igual\PYGZus{}a} \PYG{o}{=} \PYG{l+m+mi}{7} \PYG{o}{\PYGZgt{}}\PYG{o}{=} \PYG{l+m+mi}{6}
\PYG{n+nb}{print}\PYG{p}{(}\PYG{l+s+s2}{\PYGZdq{}}\PYG{l+s+s2}{Maior ou igual a:}\PYG{l+s+s2}{\PYGZdq{}}\PYG{p}{,} \PYG{n}{maior\PYGZus{}ou\PYGZus{}igual\PYGZus{}a}\PYG{p}{)}
\end{sphinxVerbatim}

\end{sphinxuseclass}\end{sphinxVerbatimInput}
\begin{sphinxVerbatimOutput}

\begin{sphinxuseclass}{cell_output}
\begin{sphinxVerbatim}[commandchars=\\\{\}]
Igual a: True
Diferente de: True
Menor que: True
Maior que: True
Menor ou igual a: True
Maior ou igual a: True
\end{sphinxVerbatim}

\end{sphinxuseclass}\end{sphinxVerbatimOutput}

\end{sphinxuseclass}

\section{Operadores Lógicos}
\label{\detokenize{chapters/ch3/ch3:operadores-logicos}}\begin{itemize}
\item {} 
\sphinxAtStartPar
\sphinxcode{\sphinxupquote{and}} (e lógico): Retorna True se ambos os operandos forem True.

\item {} 
\sphinxAtStartPar
\sphinxcode{\sphinxupquote{or}} (ou lógico): Retorna True se pelo menos um dos operandos for True.

\item {} 
\sphinxAtStartPar
\sphinxcode{\sphinxupquote{not}} (negação lógica): Inverte o valor do operando.

\end{itemize}

\begin{sphinxuseclass}{cell}\begin{sphinxVerbatimInput}

\begin{sphinxuseclass}{cell_input}
\begin{sphinxVerbatim}[commandchars=\\\{\}]
\PYG{c+c1}{\PYGZsh{} Operadores Lógicos em Python}

\PYG{c+c1}{\PYGZsh{} E lógico (and): Retorna True se ambos os operandos forem True.}
\PYG{n}{resultado\PYGZus{}and\PYGZus{}1} \PYG{o}{=} \PYG{k+kc}{True} \PYG{o+ow}{and} \PYG{k+kc}{True}
\PYG{n}{resultado\PYGZus{}and\PYGZus{}2} \PYG{o}{=} \PYG{k+kc}{True} \PYG{o+ow}{and} \PYG{k+kc}{False}
\PYG{n}{resultado\PYGZus{}and\PYGZus{}3} \PYG{o}{=} \PYG{k+kc}{False} \PYG{o+ow}{and} \PYG{k+kc}{False}

\PYG{c+c1}{\PYGZsh{} Ou lógico (or): Retorna True se pelo menos um dos operandos for True.}
\PYG{n}{resultado\PYGZus{}or\PYGZus{}1} \PYG{o}{=} \PYG{k+kc}{True} \PYG{o+ow}{or} \PYG{k+kc}{True}
\PYG{n}{resultado\PYGZus{}or\PYGZus{}2} \PYG{o}{=} \PYG{k+kc}{True} \PYG{o+ow}{or} \PYG{k+kc}{False}
\PYG{n}{resultado\PYGZus{}or\PYGZus{}3} \PYG{o}{=} \PYG{k+kc}{False} \PYG{o+ow}{or} \PYG{k+kc}{False}

\PYG{c+c1}{\PYGZsh{} Negação lógica (not): Inverte o valor do operando.}
\PYG{n}{resultado\PYGZus{}not\PYGZus{}1} \PYG{o}{=} \PYG{o+ow}{not} \PYG{k+kc}{True}
\PYG{n}{resultado\PYGZus{}not\PYGZus{}2} \PYG{o}{=} \PYG{o+ow}{not} \PYG{k+kc}{False}

\PYG{c+c1}{\PYGZsh{} Exibição dos resultados}
\PYG{n+nb}{print}\PYG{p}{(}\PYG{l+s+s2}{\PYGZdq{}}\PYG{l+s+s2}{E lógico (and):}\PYG{l+s+s2}{\PYGZdq{}}\PYG{p}{,} \PYG{n}{resultado\PYGZus{}and\PYGZus{}1}\PYG{p}{,} \PYG{n}{resultado\PYGZus{}and\PYGZus{}2}\PYG{p}{,} \PYG{n}{resultado\PYGZus{}and\PYGZus{}3}\PYG{p}{)}
\PYG{n+nb}{print}\PYG{p}{(}\PYG{l+s+s2}{\PYGZdq{}}\PYG{l+s+s2}{Ou lógico (or):}\PYG{l+s+s2}{\PYGZdq{}}\PYG{p}{,} \PYG{n}{resultado\PYGZus{}or\PYGZus{}1}\PYG{p}{,} \PYG{n}{resultado\PYGZus{}or\PYGZus{}2}\PYG{p}{,} \PYG{n}{resultado\PYGZus{}or\PYGZus{}3}\PYG{p}{)}
\PYG{n+nb}{print}\PYG{p}{(}\PYG{l+s+s2}{\PYGZdq{}}\PYG{l+s+s2}{Negação lógica (not):}\PYG{l+s+s2}{\PYGZdq{}}\PYG{p}{,} \PYG{n}{resultado\PYGZus{}not\PYGZus{}1}\PYG{p}{,} \PYG{n}{resultado\PYGZus{}not\PYGZus{}2}\PYG{p}{)}
\end{sphinxVerbatim}

\end{sphinxuseclass}\end{sphinxVerbatimInput}
\begin{sphinxVerbatimOutput}

\begin{sphinxuseclass}{cell_output}
\begin{sphinxVerbatim}[commandchars=\\\{\}]
E lógico (and): True False False
Ou lógico (or): True True False
Negação lógica (not): False True
\end{sphinxVerbatim}

\end{sphinxuseclass}\end{sphinxVerbatimOutput}

\end{sphinxuseclass}

\section{Operadores de Atribuição}
\label{\detokenize{chapters/ch3/ch3:operadores-de-atribuicao}}\begin{itemize}
\item {} 
\sphinxAtStartPar
\sphinxcode{\sphinxupquote{=}} (atribuição): Atribui o valor do operando direito ao operando esquerdo.

\item {} 
\sphinxAtStartPar
\sphinxcode{\sphinxupquote{+=}}, \sphinxcode{\sphinxupquote{\sphinxhyphen{}=}}, \sphinxcode{\sphinxupquote{*=}}, \sphinxcode{\sphinxupquote{/=}}, \sphinxcode{\sphinxupquote{\%=}} (atribuições com operações aritméticas): Realiza a operação aritmética indicada e atribui o resultado à variável à esquerda.

\end{itemize}

\begin{sphinxuseclass}{cell}\begin{sphinxVerbatimInput}

\begin{sphinxuseclass}{cell_input}
\begin{sphinxVerbatim}[commandchars=\\\{\}]
\PYG{c+c1}{\PYGZsh{} Operadores de Atribuição em Python}

\PYG{c+c1}{\PYGZsh{} Atribuição (=): Atribui o valor do operando direito ao operando esquerdo.}
\PYG{n}{x} \PYG{o}{=} \PYG{l+m+mi}{5}
\PYG{n}{y} \PYG{o}{=} \PYG{n}{x}

\PYG{c+c1}{\PYGZsh{} Atribuições com operações aritméticas: Realiza a operação aritmética e atribui o resultado à variável à esquerda.}
\PYG{n}{a} \PYG{o}{=} \PYG{l+m+mi}{10}
\PYG{n}{a} \PYG{o}{+}\PYG{o}{=} \PYG{l+m+mi}{2}  \PYG{c+c1}{\PYGZsh{} Equivalente a: a = a + 2}

\PYG{n}{b} \PYG{o}{=} \PYG{l+m+mi}{7}
\PYG{n}{b} \PYG{o}{\PYGZhy{}}\PYG{o}{=} \PYG{l+m+mi}{3}  \PYG{c+c1}{\PYGZsh{} Equivalente a: b = b \PYGZhy{} 3}

\PYG{n}{c} \PYG{o}{=} \PYG{l+m+mi}{3}
\PYG{n}{c} \PYG{o}{*}\PYG{o}{=} \PYG{l+m+mi}{5}  \PYG{c+c1}{\PYGZsh{} Equivalente a: c = c * 5}

\PYG{n}{d} \PYG{o}{=} \PYG{l+m+mi}{8}
\PYG{n}{d} \PYG{o}{/}\PYG{o}{=} \PYG{l+m+mi}{2}  \PYG{c+c1}{\PYGZsh{} Equivalente a: d = d / 2}

\PYG{n}{e} \PYG{o}{=} \PYG{l+m+mi}{15}
\PYG{n}{e} \PYG{o}{\PYGZpc{}}\PYG{o}{=} \PYG{l+m+mi}{4}  \PYG{c+c1}{\PYGZsh{} Equivalente a: e = e \PYGZpc{} 4}

\PYG{c+c1}{\PYGZsh{} Exibição dos valores após as operações}
\PYG{n+nb}{print}\PYG{p}{(}\PYG{l+s+s2}{\PYGZdq{}}\PYG{l+s+s2}{Atribuição (=):}\PYG{l+s+s2}{\PYGZdq{}}\PYG{p}{,} \PYG{n}{x}\PYG{p}{,} \PYG{n}{y}\PYG{p}{)}
\PYG{n+nb}{print}\PYG{p}{(}\PYG{l+s+s2}{\PYGZdq{}}\PYG{l+s+s2}{Atribuição com soma (+=):}\PYG{l+s+s2}{\PYGZdq{}}\PYG{p}{,} \PYG{n}{a}\PYG{p}{)}
\PYG{n+nb}{print}\PYG{p}{(}\PYG{l+s+s2}{\PYGZdq{}}\PYG{l+s+s2}{Atribuição com subtração (\PYGZhy{}=):}\PYG{l+s+s2}{\PYGZdq{}}\PYG{p}{,} \PYG{n}{b}\PYG{p}{)}
\PYG{n+nb}{print}\PYG{p}{(}\PYG{l+s+s2}{\PYGZdq{}}\PYG{l+s+s2}{Atribuição com multiplicação (*=):}\PYG{l+s+s2}{\PYGZdq{}}\PYG{p}{,} \PYG{n}{c}\PYG{p}{)}
\PYG{n+nb}{print}\PYG{p}{(}\PYG{l+s+s2}{\PYGZdq{}}\PYG{l+s+s2}{Atribuição com divisão (/=):}\PYG{l+s+s2}{\PYGZdq{}}\PYG{p}{,} \PYG{n}{d}\PYG{p}{)}
\PYG{n+nb}{print}\PYG{p}{(}\PYG{l+s+s2}{\PYGZdq{}}\PYG{l+s+s2}{Atribuição com módulo (}\PYG{l+s+s2}{\PYGZpc{}}\PYG{l+s+s2}{=):}\PYG{l+s+s2}{\PYGZdq{}}\PYG{p}{,} \PYG{n}{e}\PYG{p}{)}
\end{sphinxVerbatim}

\end{sphinxuseclass}\end{sphinxVerbatimInput}
\begin{sphinxVerbatimOutput}

\begin{sphinxuseclass}{cell_output}
\begin{sphinxVerbatim}[commandchars=\\\{\}]
Atribuição (=): 5 5
Atribuição com soma (+=): 12
Atribuição com subtração (\PYGZhy{}=): 4
Atribuição com multiplicação (*=): 15
Atribuição com divisão (/=): 4.0
Atribuição com módulo (\PYGZpc{}=): 3
\end{sphinxVerbatim}

\end{sphinxuseclass}\end{sphinxVerbatimOutput}

\end{sphinxuseclass}

\section{Operadores de Incremento e Decremento}
\label{\detokenize{chapters/ch3/ch3:operadores-de-incremento-e-decremento}}
\sphinxAtStartPar
Em Python, ao contrário de algumas outras linguagens de programação, não há operadores de incremento (\sphinxcode{\sphinxupquote{++}}) e decremento (\sphinxcode{\sphinxupquote{\sphinxhyphen{}\sphinxhyphen{}}}) específicos. No entanto, você pode alcançar o mesmo efeito usando operadores de atribuição e aritméticos. Vamos ver como isso pode ser feito:


\subsection{Operador de Incremento}
\label{\detokenize{chapters/ch3/ch3:operador-de-incremento}}
\sphinxAtStartPar
Para incrementar o valor de uma variável em 1, você pode usar o operador de adição (\sphinxcode{\sphinxupquote{+=}}):

\begin{sphinxVerbatim}[commandchars=\\\{\}]
\PYG{c+c1}{\PYGZsh{} Exemplo de incremento}
\PYG{n}{contador} \PYG{o}{=} \PYG{l+m+mi}{5}
\PYG{n}{contador} \PYG{o}{+}\PYG{o}{=} \PYG{l+m+mi}{1}  \PYG{c+c1}{\PYGZsh{} Incrementa o valor de contador em 1}
\PYG{n+nb}{print}\PYG{p}{(}\PYG{n}{contador}\PYG{p}{)}
\end{sphinxVerbatim}

\begin{sphinxVerbatim}[commandchars=\\\{\}]
\PYG{l+m+mi}{6}
\end{sphinxVerbatim}


\subsection{Operador de Decremento}
\label{\detokenize{chapters/ch3/ch3:operador-de-decremento}}
\sphinxAtStartPar
Para decrementar o valor de uma variável em 1, você pode usar o operador de subtração (\sphinxcode{\sphinxupquote{\sphinxhyphen{}=}}):

\begin{sphinxVerbatim}[commandchars=\\\{\}]
\PYG{c+c1}{\PYGZsh{} Exemplo de decremento}
\PYG{n}{contador} \PYG{o}{=} \PYG{l+m+mi}{8}
\PYG{n}{contador} \PYG{o}{\PYGZhy{}}\PYG{o}{=} \PYG{l+m+mi}{1}  \PYG{c+c1}{\PYGZsh{} Decrementa o valor de contador em 1}
\PYG{n+nb}{print}\PYG{p}{(}\PYG{n}{contador}\PYG{p}{)}
\end{sphinxVerbatim}

\begin{sphinxVerbatim}[commandchars=\\\{\}]
\PYG{l+m+mi}{7}
\end{sphinxVerbatim}

\sphinxAtStartPar
Esses exemplos ilustram como realizar operações de incremento e decremento em Python. A sintaxe \sphinxcode{\sphinxupquote{+=}} é uma forma concisa de escrever “atribua à variável o valor atual da variável mais alguma quantidade”. O mesmo princípio se aplica ao operador \sphinxcode{\sphinxupquote{\sphinxhyphen{}=}} para decremento.


\section{Operadores Unários}
\label{\detokenize{chapters/ch3/ch3:operadores-unarios}}
\sphinxAtStartPar
Os operadores unários atuam em um único operando, ou seja, em um único valor. Eles realizam operações sobre esse valor. Vamos ver alguns exemplos:


\subsection{Operador Negativo (\sphinxstyleliteralintitle{\sphinxupquote{\sphinxhyphen{}}}):}
\label{\detokenize{chapters/ch3/ch3:operador-negativo}}
\sphinxAtStartPar
Muda o sinal do número para negativo.

\begin{sphinxVerbatim}[commandchars=\\\{\}]
\PYG{n}{x} \PYG{o}{=} \PYG{l+m+mi}{5}
\PYG{n}{resultado} \PYG{o}{=} \PYG{o}{\PYGZhy{}}\PYG{n}{x}
\end{sphinxVerbatim}

\begin{sphinxVerbatim}[commandchars=\\\{\}]
\PYG{o}{\PYGZhy{}}\PYG{l+m+mi}{5}
\end{sphinxVerbatim}


\subsection{Operador Positivo (\sphinxstyleliteralintitle{\sphinxupquote{+}}):}
\label{\detokenize{chapters/ch3/ch3:operador-positivo}}
\sphinxAtStartPar
Mantém o sinal do número (raramente usado, já que os números são positivos por padrão).

\begin{sphinxVerbatim}[commandchars=\\\{\}]
\PYG{n}{y} \PYG{o}{=} \PYG{o}{\PYGZhy{}}\PYG{l+m+mi}{3}
\PYG{n}{resultado} \PYG{o}{=} \PYG{o}{+}\PYG{n}{y}  \PYG{c+c1}{\PYGZsh{} O valor de resultado é sem mudança de sinal.}
\end{sphinxVerbatim}

\begin{sphinxVerbatim}[commandchars=\\\{\}]
\PYG{o}{\PYGZhy{}}\PYG{l+m+mi}{3}
\end{sphinxVerbatim}


\subsection{Operador de Inversão Bit a Bit (\sphinxstyleliteralintitle{\sphinxupquote{\textasciitilde{}}}):}
\label{\detokenize{chapters/ch3/ch3:operador-de-inversao-bit-a-bit}}
\sphinxAtStartPar
Inverte cada bit do número.

\begin{sphinxVerbatim}[commandchars=\\\{\}]
\PYG{n}{z} \PYG{o}{=} \PYG{l+m+mi}{7}
\PYG{n}{resultado} \PYG{o}{=} \PYG{o}{\PYGZti{}}\PYG{n}{z}
\end{sphinxVerbatim}

\begin{sphinxVerbatim}[commandchars=\\\{\}]
\PYG{o}{\PYGZhy{}}\PYG{l+m+mi}{8}
\end{sphinxVerbatim}

\sphinxAtStartPar
O operador negativo e o operador positivo lidam com números inteiros e de ponto flutuante, enquanto o operador de inversão bit a bit opera em valores inteiros.


\section{Exercícios}
\label{\detokenize{chapters/ch3/ch3:exercicios}}\begin{enumerate}
\sphinxsetlistlabels{\arabic}{enumi}{enumii}{}{.}%
\item {} 
\sphinxAtStartPar
Dada a expressão matemática: \((a = 4 \times (2 + 3))\), crie uma variável chamada \sphinxcode{\sphinxupquote{a}} e atribua a ela o resultado dessa expressão. Imprima o valor de \sphinxcode{\sphinxupquote{a}}.

\item {} 
\sphinxAtStartPar
Escreva um programa em Python que recebe dois números do usuário, realiza a soma desses números e exibe o resultado.

\item {} 
\sphinxAtStartPar
Calcule o resto da divisão de 17 por 5 e armazene o resultado em uma variável chamada \sphinxcode{\sphinxupquote{resto}}. Imprima o valor de \sphinxcode{\sphinxupquote{resto}}.

\item {} 
\sphinxAtStartPar
Crie uma expressão lógica que seja verdadeira se um número for par e maior que 10. Teste a expressão com diferentes valores e imprima os resultados.

\item {} 
\sphinxAtStartPar
Dada a variável \sphinxcode{\sphinxupquote{preco\_produto}} com o valor 150, aplique um desconto de 20\% utilizando operadores aritméticos e de atribuição. Imprima o novo valor.

\item {} 
\sphinxAtStartPar
Implemente um programa que recebe a idade de uma pessoa e verifica se ela é maior de idade (idade maior ou igual a 18). Exiba a mensagem adequada.

\item {} 
\sphinxAtStartPar
Crie uma expressão lógica que seja verdadeira apenas se um número for ímpar ou menor que 5. Teste a expressão com diferentes valores e imprima os resultados.

\item {} 
\sphinxAtStartPar
Dado o raio de um círculo, calcule a área utilizando o operador de potência (**) para elevar o raio ao quadrado. Imprima o resultado.

\item {} 
\sphinxAtStartPar
Escreva um programa que converta uma temperatura de Celsius para Fahrenheit. Utilize a fórmula \((F = \frac{9}{5}C + 32)\).

\item {} 
\sphinxAtStartPar
Dada a expressão \((x = 3)\) e \((y = 5)\), crie uma variável chamada \sphinxcode{\sphinxupquote{resultado}} que armazene o valor da expressão \((x^2 + y^2)\) e imprima o resultado.

\end{enumerate}

\sphinxstepscope


\chapter{Capítulo 4: Controle de Fluxo}
\label{\detokenize{chapters/ch4/ch4:capitulo-4-controle-de-fluxo}}\label{\detokenize{chapters/ch4/ch4::doc}}

\section{Estruturas condicionais}
\label{\detokenize{chapters/ch4/ch4:estruturas-condicionais}}
\sphinxAtStartPar
As estruturas condicionais permitem ao programador tomar decisões sobre o fluxo de execução do programa. As estruturas condicionais mais comuns em Python são:
\begin{itemize}
\item {} 
\sphinxAtStartPar
\sphinxstylestrong{\sphinxcode{\sphinxupquote{if}}:} Testa uma condição e executa um bloco de código se a condição for verdadeira.

\item {} 
\sphinxAtStartPar
\sphinxstylestrong{\sphinxcode{\sphinxupquote{elif}}:} Testa uma condição e executa um bloco de código se a condição for verdadeira, mas apenas se a condição anterior for falsa.

\item {} 
\sphinxAtStartPar
\sphinxstylestrong{\sphinxcode{\sphinxupquote{else}}:} Executa um bloco de código se nenhuma das condições anteriores for verdadeira.

\end{itemize}

\sphinxAtStartPar
As estruturas condicionais em Python, especificamente as instruções \sphinxcode{\sphinxupquote{if}}, \sphinxcode{\sphinxupquote{elif}}, e \sphinxcode{\sphinxupquote{else}}, são essenciais para tomar decisões em um programa com base em diferentes condições.


\subsection{A Instrução \sphinxstyleliteralintitle{\sphinxupquote{if}}}
\label{\detokenize{chapters/ch4/ch4:a-instrucao-if}}
\sphinxAtStartPar
A instrução \sphinxcode{\sphinxupquote{if}} é utilizada para executar um bloco de código se uma condição especificada for avaliada como verdadeira. Vejamos um exemplo prático:

\begin{sphinxVerbatim}[commandchars=\\\{\}]
\PYG{n}{idade} \PYG{o}{=} \PYG{l+m+mi}{20}

\PYG{k}{if} \PYG{n}{idade} \PYG{o}{\PYGZgt{}}\PYG{o}{=} \PYG{l+m+mi}{18}\PYG{p}{:}
    \PYG{n+nb}{print}\PYG{p}{(}\PYG{l+s+s2}{\PYGZdq{}}\PYG{l+s+s2}{Você é maior de idade.}\PYG{l+s+s2}{\PYGZdq{}}\PYG{p}{)}
\end{sphinxVerbatim}

\sphinxAtStartPar
Neste exemplo, o bloco de código dentro do \sphinxcode{\sphinxupquote{if}} só será executado se a variável \sphinxcode{\sphinxupquote{idade}} for maior ou igual a 18. Caso contrário, o bloco será ignorado.


\subsection{Instruções \sphinxstyleliteralintitle{\sphinxupquote{elif}} e \sphinxstyleliteralintitle{\sphinxupquote{else}}}
\label{\detokenize{chapters/ch4/ch4:instrucoes-elif-e-else}}
\sphinxAtStartPar
Quando lidamos com múltiplas condições, as instruções \sphinxcode{\sphinxupquote{elif}} (abreviação de “else if”) e \sphinxcode{\sphinxupquote{else}} podem ser empregadas.


\subsubsection{Exemplo com \sphinxstyleliteralintitle{\sphinxupquote{elif}}:}
\label{\detokenize{chapters/ch4/ch4:exemplo-com-elif}}
\begin{sphinxVerbatim}[commandchars=\\\{\}]
\PYG{n}{idade} \PYG{o}{=} \PYG{l+m+mi}{16}

\PYG{k}{if} \PYG{n}{idade} \PYG{o}{\PYGZlt{}} \PYG{l+m+mi}{18}\PYG{p}{:}
    \PYG{n+nb}{print}\PYG{p}{(}\PYG{l+s+s2}{\PYGZdq{}}\PYG{l+s+s2}{Você é menor de idade.}\PYG{l+s+s2}{\PYGZdq{}}\PYG{p}{)}
\PYG{k}{elif} \PYG{n}{idade} \PYG{o}{==} \PYG{l+m+mi}{18}\PYG{p}{:}
    \PYG{n+nb}{print}\PYG{p}{(}\PYG{l+s+s2}{\PYGZdq{}}\PYG{l+s+s2}{Você acabou de atingir a maioridade.}\PYG{l+s+s2}{\PYGZdq{}}\PYG{p}{)}
\PYG{k}{else}\PYG{p}{:}
    \PYG{n+nb}{print}\PYG{p}{(}\PYG{l+s+s2}{\PYGZdq{}}\PYG{l+s+s2}{Você é maior de idade.}\PYG{l+s+s2}{\PYGZdq{}}\PYG{p}{)}
\end{sphinxVerbatim}

\sphinxAtStartPar
Neste exemplo, o programa verifica a idade e imprime uma mensagem apropriada com base nas condições. Se a idade for menor que 18, imprime “Você é menor de idade”. Se a idade for exatamente 18, imprime “Você acabou de atingir a maioridade”. Caso contrário, o bloco dentro do \sphinxcode{\sphinxupquote{else}} é executado, imprimindo “Você é maior de idade”.


\subsection{Exemplo Prático: Verificação de Números Pares e Ímpares}
\label{\detokenize{chapters/ch4/ch4:exemplo-pratico-verificacao-de-numeros-pares-e-impares}}
\sphinxAtStartPar
Vamos criar um exemplo mais prático usando estruturas condicionais para verificar se um número é par ou ímpar:

\begin{sphinxVerbatim}[commandchars=\\\{\}]
\PYG{n}{numero} \PYG{o}{=} \PYG{l+m+mi}{15}

\PYG{k}{if} \PYG{n}{numero} \PYG{o}{\PYGZpc{}} \PYG{l+m+mi}{2} \PYG{o}{==} \PYG{l+m+mi}{0}\PYG{p}{:}
    \PYG{n+nb}{print}\PYG{p}{(}\PYG{l+s+sa}{f}\PYG{l+s+s2}{\PYGZdq{}}\PYG{l+s+si}{\PYGZob{}}\PYG{n}{numero}\PYG{l+s+si}{\PYGZcb{}}\PYG{l+s+s2}{ é um número par.}\PYG{l+s+s2}{\PYGZdq{}}\PYG{p}{)}
\PYG{k}{else}\PYG{p}{:}
    \PYG{n+nb}{print}\PYG{p}{(}\PYG{l+s+sa}{f}\PYG{l+s+s2}{\PYGZdq{}}\PYG{l+s+si}{\PYGZob{}}\PYG{n}{numero}\PYG{l+s+si}{\PYGZcb{}}\PYG{l+s+s2}{ é um número ímpar.}\PYG{l+s+s2}{\PYGZdq{}}\PYG{p}{)}
\end{sphinxVerbatim}

\begin{sphinxVerbatim}[commandchars=\\\{\}]
\PYG{l+m+mi}{15} \PYG{n}{é} \PYG{n}{um} \PYG{n}{número} \PYG{n}{ímpar}\PYG{o}{.}
\end{sphinxVerbatim}

\sphinxAtStartPar
Neste exemplo, o operador \sphinxcode{\sphinxupquote{\%}} calcula o resto da divisão por 2. Se o resto for zero, o número é par; caso contrário, é ímpar.

\sphinxAtStartPar
\sphinxstylestrong{Observação:} O trecho \sphinxcode{\sphinxupquote{print(f"\{numero\} é um número ímpar.")}} utiliza uma f\sphinxhyphen{}string para criar uma string formatada, onde \sphinxcode{\sphinxupquote{\{numero\}}} é substituído pelo valor atual da variável \sphinxcode{\sphinxupquote{numero}}.


\subsection{Aninhamento de Estruturas Condicionais}
\label{\detokenize{chapters/ch4/ch4:aninhamento-de-estruturas-condicionais}}
\sphinxAtStartPar
É possível aninhar instruções \sphinxcode{\sphinxupquote{if}} dentro de outras instruções \sphinxcode{\sphinxupquote{if}}, \sphinxcode{\sphinxupquote{elif}}, ou \sphinxcode{\sphinxupquote{else}}. Isso permite lidar com condições mais complexas. No entanto, deve\sphinxhyphen{}se ter cuidado para não tornar o código muito complexo.

\begin{sphinxVerbatim}[commandchars=\\\{\}]
\PYG{n}{idade} \PYG{o}{=} \PYG{l+m+mi}{25}
\PYG{n}{sexo} \PYG{o}{=} \PYG{l+s+s2}{\PYGZdq{}}\PYG{l+s+s2}{Feminino}\PYG{l+s+s2}{\PYGZdq{}}

\PYG{k}{if} \PYG{n}{idade} \PYG{o}{\PYGZgt{}}\PYG{o}{=} \PYG{l+m+mi}{18}\PYG{p}{:}
    \PYG{n+nb}{print}\PYG{p}{(}\PYG{l+s+s2}{\PYGZdq{}}\PYG{l+s+s2}{Você é maior de idade.}\PYG{l+s+s2}{\PYGZdq{}}\PYG{p}{)}
    
    \PYG{k}{if} \PYG{n}{sexo} \PYG{o}{==} \PYG{l+s+s2}{\PYGZdq{}}\PYG{l+s+s2}{Feminino}\PYG{l+s+s2}{\PYGZdq{}}\PYG{p}{:}
        \PYG{n+nb}{print}\PYG{p}{(}\PYG{l+s+s2}{\PYGZdq{}}\PYG{l+s+s2}{E também do sexo feminino.}\PYG{l+s+s2}{\PYGZdq{}}\PYG{p}{)}
\PYG{k}{else}\PYG{p}{:}
    \PYG{n+nb}{print}\PYG{p}{(}\PYG{l+s+s2}{\PYGZdq{}}\PYG{l+s+s2}{Você é menor de idade.}\PYG{l+s+s2}{\PYGZdq{}}\PYG{p}{)}
\end{sphinxVerbatim}

\sphinxAtStartPar
Analisando o exemplo:

\sphinxAtStartPar
\sphinxstylestrong{Avaliação Externa (\sphinxcode{\sphinxupquote{if idade >= 18}}):}
\begin{itemize}
\item {} 
\sphinxAtStartPar
Se a idade for maior ou igual a 18, o bloco interno é executado.

\item {} 
\sphinxAtStartPar
A mensagem “Você é maior de idade.” será impressa.

\end{itemize}

\sphinxAtStartPar
\sphinxstylestrong{Bloco Interno (\sphinxcode{\sphinxupquote{if sexo == "Feminino"}}):}
\begin{itemize}
\item {} 
\sphinxAtStartPar
Este bloco só será executado se a condição externa (idade >= 18) for verdadeira.

\item {} 
\sphinxAtStartPar
Se o sexo for “Feminino”, a mensagem “E também do sexo feminino.” será impressa.

\end{itemize}

\sphinxAtStartPar
\sphinxstylestrong{\sphinxcode{\sphinxupquote{else}} Externo:}
\begin{itemize}
\item {} 
\sphinxAtStartPar
Se a condição externa não for atendida (idade < 18), a mensagem “Você é menor de idade.” será impressa.

\end{itemize}

\sphinxAtStartPar
Agora, como segundo exemplo, considere uma situação em que você está classificando alunos com base em suas notas em uma disciplina:

\begin{sphinxVerbatim}[commandchars=\\\{\}]
\PYG{n}{nota} \PYG{o}{=} \PYG{l+m+mi}{75}

\PYG{k}{if} \PYG{n}{nota} \PYG{o}{\PYGZgt{}}\PYG{o}{=} \PYG{l+m+mi}{90}\PYG{p}{:}
    \PYG{n+nb}{print}\PYG{p}{(}\PYG{l+s+s2}{\PYGZdq{}}\PYG{l+s+s2}{Parabéns! Você obteve uma nota A.}\PYG{l+s+s2}{\PYGZdq{}}\PYG{p}{)}
\PYG{k}{elif} \PYG{n}{nota} \PYG{o}{\PYGZgt{}}\PYG{o}{=} \PYG{l+m+mi}{80}\PYG{p}{:}
    \PYG{n+nb}{print}\PYG{p}{(}\PYG{l+s+s2}{\PYGZdq{}}\PYG{l+s+s2}{Ótimo! Sua nota é B.}\PYG{l+s+s2}{\PYGZdq{}}\PYG{p}{)}
\PYG{k}{elif} \PYG{n}{nota} \PYG{o}{\PYGZgt{}}\PYG{o}{=} \PYG{l+m+mi}{70}\PYG{p}{:}
    \PYG{n+nb}{print}\PYG{p}{(}\PYG{l+s+s2}{\PYGZdq{}}\PYG{l+s+s2}{Bom trabalho! Sua nota é C.}\PYG{l+s+s2}{\PYGZdq{}}\PYG{p}{)}
\PYG{k}{else}\PYG{p}{:}
    \PYG{n+nb}{print}\PYG{p}{(}\PYG{l+s+s2}{\PYGZdq{}}\PYG{l+s+s2}{Infelizmente, você não atingiu a nota mínima. Sua nota é D.}\PYG{l+s+s2}{\PYGZdq{}}\PYG{p}{)}
\end{sphinxVerbatim}

\sphinxAtStartPar
\sphinxstylestrong{Avaliação da Nota (\sphinxcode{\sphinxupquote{if nota >= 90}}):}
\begin{itemize}
\item {} 
\sphinxAtStartPar
Se a nota for 90 ou superior, o aluno recebe uma nota A.

\end{itemize}

\sphinxAtStartPar
\sphinxstylestrong{\sphinxcode{\sphinxupquote{elif nota >= 80}}:}
\begin{itemize}
\item {} 
\sphinxAtStartPar
Se a condição anterior não for atendida, mas a nota for 80 ou superior, o aluno recebe uma nota B.

\end{itemize}

\sphinxAtStartPar
\sphinxstylestrong{\sphinxcode{\sphinxupquote{elif nota >= 70}}:}
\begin{itemize}
\item {} 
\sphinxAtStartPar
Se a condição anterior não for atendida, mas a nota for 70 ou superior, o aluno recebe uma nota C.

\end{itemize}

\sphinxAtStartPar
\sphinxstylestrong{\sphinxcode{\sphinxupquote{else}}:}
\begin{itemize}
\item {} 
\sphinxAtStartPar
Se nenhuma das condições anteriores for atendida, o aluno recebe uma nota D.

\end{itemize}

\sphinxAtStartPar
Esses exemplos mostram como você pode aninhar estruturas condicionais para lidar com várias situações e condições em seu código de maneira organizada.


\section{Estruturas de repetição}
\label{\detokenize{chapters/ch4/ch4:estruturas-de-repeticao}}
\sphinxAtStartPar
As estruturas de repetição em Python, também conhecidas como loops, são utilizadas para executar um bloco de código várias vezes. Python possui duas principais estruturas de repetição: \sphinxcode{\sphinxupquote{for}} e \sphinxcode{\sphinxupquote{while}}. Vamos explorar cada uma delas com teoria e exemplos práticos.


\subsection{Estrutura de Repetição \sphinxstyleliteralintitle{\sphinxupquote{for}}}
\label{\detokenize{chapters/ch4/ch4:estrutura-de-repeticao-for}}
\sphinxAtStartPar
A estrutura \sphinxcode{\sphinxupquote{for}} é uma ferramenta poderosa em Python, projetada para iterar sobre sequências, como listas, tuplas, strings e outros objetos iteráveis. Ela proporciona uma maneira elegante e eficiente de processar cada elemento de uma sequência, executando um bloco de código associado a cada iteração.

\sphinxAtStartPar
\sphinxstylestrong{Sintaxe}

\begin{sphinxVerbatim}[commandchars=\\\{\}]
\PYG{k}{for} \PYG{n}{variavel} \PYG{o+ow}{in} \PYG{n}{sequencia}\PYG{p}{:}
    \PYG{c+c1}{\PYGZsh{} Bloco de código a ser repetido}
\end{sphinxVerbatim}

\sphinxAtStartPar
Aqui, \sphinxcode{\sphinxupquote{variavel}} é uma variável que assume o valor de cada elemento da \sphinxcode{\sphinxupquote{sequencia}} durante cada iteração do loop. O bloco de código associado é executado para cada valor da sequência.

\sphinxAtStartPar
\sphinxstylestrong{Exemplo Prático: Iterando sobre uma Lista de Frutas}

\begin{sphinxVerbatim}[commandchars=\\\{\}]
\PYG{n}{frutas} \PYG{o}{=} \PYG{p}{[}\PYG{l+s+s2}{\PYGZdq{}}\PYG{l+s+s2}{maçã}\PYG{l+s+s2}{\PYGZdq{}}\PYG{p}{,} \PYG{l+s+s2}{\PYGZdq{}}\PYG{l+s+s2}{banana}\PYG{l+s+s2}{\PYGZdq{}}\PYG{p}{,} \PYG{l+s+s2}{\PYGZdq{}}\PYG{l+s+s2}{uva}\PYG{l+s+s2}{\PYGZdq{}}\PYG{p}{]}
\PYG{k}{for} \PYG{n}{fruta} \PYG{o+ow}{in} \PYG{n}{frutas}\PYG{p}{:}
    \PYG{n+nb}{print}\PYG{p}{(}\PYG{n}{fruta}\PYG{p}{)}
\end{sphinxVerbatim}

\sphinxAtStartPar
\sphinxstylestrong{Saída:}

\begin{sphinxVerbatim}[commandchars=\\\{\}]
\PYG{n}{maçã}
\PYG{n}{banana}
\PYG{n}{uva}
\end{sphinxVerbatim}

\sphinxAtStartPar
Neste exemplo, o loop \sphinxcode{\sphinxupquote{for}} percorre a lista \sphinxcode{\sphinxupquote{frutas}} e imprime cada elemento da lista. Isso é particularmente útil ao lidar com conjuntos de dados, como uma lista de itens a serem processados.


\subsubsection{Casos de Utilização do \sphinxstyleliteralintitle{\sphinxupquote{for}}}
\label{\detokenize{chapters/ch4/ch4:casos-de-utilizacao-do-for}}
\sphinxAtStartPar
\sphinxstylestrong{Números em um Intervalo:}

\begin{sphinxVerbatim}[commandchars=\\\{\}]
\PYG{k}{for} \PYG{n}{i} \PYG{o+ow}{in} \PYG{n+nb}{range}\PYG{p}{(}\PYG{l+m+mi}{1}\PYG{p}{,} \PYG{l+m+mi}{6}\PYG{p}{)}\PYG{p}{:}
    \PYG{n+nb}{print}\PYG{p}{(}\PYG{n}{i}\PYG{p}{)}
\end{sphinxVerbatim}

\sphinxAtStartPar
\sphinxstylestrong{Saída:}

\begin{sphinxVerbatim}[commandchars=\\\{\}]
\PYG{l+m+mi}{1}
\PYG{l+m+mi}{2}
\PYG{l+m+mi}{3}
\PYG{l+m+mi}{4}
\PYG{l+m+mi}{5}
\end{sphinxVerbatim}

\sphinxAtStartPar
O \sphinxcode{\sphinxupquote{range(1, 6)}} cria uma sequência de números de 1 a 5, e o \sphinxcode{\sphinxupquote{for}} itera sobre esses valores.

\sphinxAtStartPar
\sphinxstylestrong{Iteração sobre uma String:}

\begin{sphinxVerbatim}[commandchars=\\\{\}]
\PYG{n}{palavra} \PYG{o}{=} \PYG{l+s+s2}{\PYGZdq{}}\PYG{l+s+s2}{Python}\PYG{l+s+s2}{\PYGZdq{}}
\PYG{k}{for} \PYG{n}{letra} \PYG{o+ow}{in} \PYG{n}{palavra}\PYG{p}{:}
    \PYG{n+nb}{print}\PYG{p}{(}\PYG{n}{letra}\PYG{p}{)}
\end{sphinxVerbatim}

\sphinxAtStartPar
\sphinxstylestrong{Saída:}

\begin{sphinxVerbatim}[commandchars=\\\{\}]
\PYG{n}{P}
\PYG{n}{y}
\PYG{n}{t}
\PYG{n}{h}
\PYG{n}{o}
\PYG{n}{n}
\end{sphinxVerbatim}

\sphinxAtStartPar
O \sphinxcode{\sphinxupquote{for}} percorre cada caractere na string \sphinxcode{\sphinxupquote{palavra}} e imprime\sphinxhyphen{}os individualmente.

\sphinxAtStartPar
\sphinxstylestrong{Iterando sobre Dicionários:}

\begin{sphinxVerbatim}[commandchars=\\\{\}]
\PYG{n}{aluno\PYGZus{}notas} \PYG{o}{=} \PYG{p}{\PYGZob{}}\PYG{l+s+s2}{\PYGZdq{}}\PYG{l+s+s2}{Alice}\PYG{l+s+s2}{\PYGZdq{}}\PYG{p}{:} \PYG{l+m+mi}{90}\PYG{p}{,} \PYG{l+s+s2}{\PYGZdq{}}\PYG{l+s+s2}{Bob}\PYG{l+s+s2}{\PYGZdq{}}\PYG{p}{:} \PYG{l+m+mi}{80}\PYG{p}{,} \PYG{l+s+s2}{\PYGZdq{}}\PYG{l+s+s2}{Charlie}\PYG{l+s+s2}{\PYGZdq{}}\PYG{p}{:} \PYG{l+m+mi}{95}\PYG{p}{\PYGZcb{}}
\PYG{k}{for} \PYG{n}{aluno}\PYG{p}{,} \PYG{n}{nota} \PYG{o+ow}{in} \PYG{n}{aluno\PYGZus{}notas}\PYG{o}{.}\PYG{n}{items}\PYG{p}{(}\PYG{p}{)}\PYG{p}{:}
    \PYG{n+nb}{print}\PYG{p}{(}\PYG{l+s+sa}{f}\PYG{l+s+s2}{\PYGZdq{}}\PYG{l+s+si}{\PYGZob{}}\PYG{n}{aluno}\PYG{l+s+si}{\PYGZcb{}}\PYG{l+s+s2}{: }\PYG{l+s+si}{\PYGZob{}}\PYG{n}{nota}\PYG{l+s+si}{\PYGZcb{}}\PYG{l+s+s2}{\PYGZdq{}}\PYG{p}{)}
\end{sphinxVerbatim}

\sphinxAtStartPar
\sphinxstylestrong{Saída:}

\begin{sphinxVerbatim}[commandchars=\\\{\}]
\PYG{n}{Alice}\PYG{p}{:} \PYG{l+m+mi}{90}
\PYG{n}{Bob}\PYG{p}{:} \PYG{l+m+mi}{80}
\PYG{n}{Charlie}\PYG{p}{:} \PYG{l+m+mi}{95}
\end{sphinxVerbatim}

\sphinxAtStartPar
Aqui, o \sphinxcode{\sphinxupquote{for}} itera sobre os itens do dicionário, permitindo o acesso tanto ao nome do aluno quanto à sua nota.

\sphinxAtStartPar
Esses exemplos demonstram a versatilidade do \sphinxcode{\sphinxupquote{for}} em Python, tornando\sphinxhyphen{}o uma ferramenta valiosa para uma variedade de situações, desde a iteração básica sobre listas até a manipulação de estruturas de dados mais complexas.


\subsection{Estrutura de Repetição \sphinxstyleliteralintitle{\sphinxupquote{while}}}
\label{\detokenize{chapters/ch4/ch4:estrutura-de-repeticao-while}}
\sphinxAtStartPar
A estrutura \sphinxcode{\sphinxupquote{while}} é um componente fundamental em Python, possibilitando a execução repetida de um bloco de código enquanto uma condição especificada permanece verdadeira. Isso é particularmente útil quando o número exato de iterações não é conhecido antecipadamente.

\sphinxAtStartPar
\sphinxstylestrong{Sintaxe}

\begin{sphinxVerbatim}[commandchars=\\\{\}]
\PYG{k}{while} \PYG{n}{condicao}\PYG{p}{:}
    \PYG{c+c1}{\PYGZsh{} Bloco de código a ser repetido}
\end{sphinxVerbatim}

\sphinxAtStartPar
Neste contexto, o bloco de código é executado repetidamente enquanto a \sphinxcode{\sphinxupquote{condicao}} é verdadeira. A condição é avaliada antes de cada iteração, e o loop continua até que a condição se torne falsa.

\sphinxAtStartPar
\sphinxstylestrong{Exemplo Prático: Contagem até 5 usando \sphinxcode{\sphinxupquote{while}}}

\begin{sphinxVerbatim}[commandchars=\\\{\}]
\PYG{n}{contador} \PYG{o}{=} \PYG{l+m+mi}{1}
\PYG{k}{while} \PYG{n}{contador} \PYG{o}{\PYGZlt{}}\PYG{o}{=} \PYG{l+m+mi}{5}\PYG{p}{:}
    \PYG{n+nb}{print}\PYG{p}{(}\PYG{n}{contador}\PYG{p}{)}
    \PYG{n}{contador} \PYG{o}{+}\PYG{o}{=} \PYG{l+m+mi}{1}
\end{sphinxVerbatim}

\sphinxAtStartPar
\sphinxstylestrong{Saída:}

\begin{sphinxVerbatim}[commandchars=\\\{\}]
\PYG{l+m+mi}{1}
\PYG{l+m+mi}{2}
\PYG{l+m+mi}{3}
\PYG{l+m+mi}{4}
\PYG{l+m+mi}{5}
\end{sphinxVerbatim}

\sphinxAtStartPar
Neste exemplo, o \sphinxcode{\sphinxupquote{while}} é utilizado para contar até 5, incrementando o contador a cada iteração.


\subsubsection{Casos de Utilização do \sphinxstyleliteralintitle{\sphinxupquote{while}}}
\label{\detokenize{chapters/ch4/ch4:casos-de-utilizacao-do-while}}\begin{enumerate}
\sphinxsetlistlabels{\arabic}{enumi}{enumii}{}{.}%
\item {} 
\sphinxAtStartPar
\sphinxstylestrong{Entrada do Usuário:}

\begin{sphinxVerbatim}[commandchars=\\\{\}]
\PYG{n}{resposta} \PYG{o}{=} \PYG{l+s+s2}{\PYGZdq{}}\PYG{l+s+s2}{\PYGZdq{}}
\PYG{k}{while} \PYG{n}{resposta}\PYG{o}{.}\PYG{n}{lower}\PYG{p}{(}\PYG{p}{)} \PYG{o}{!=} \PYG{l+s+s2}{\PYGZdq{}}\PYG{l+s+s2}{sair}\PYG{l+s+s2}{\PYGZdq{}}\PYG{p}{:}
    \PYG{n}{resposta} \PYG{o}{=} \PYG{n+nb}{input}\PYG{p}{(}\PYG{l+s+s2}{\PYGZdq{}}\PYG{l+s+s2}{Digite }\PYG{l+s+s2}{\PYGZsq{}}\PYG{l+s+s2}{sair}\PYG{l+s+s2}{\PYGZsq{}}\PYG{l+s+s2}{ para encerrar: }\PYG{l+s+s2}{\PYGZdq{}}\PYG{p}{)}
\end{sphinxVerbatim}

\sphinxAtStartPar
Aqui, o loop \sphinxcode{\sphinxupquote{while}} permite que o programa continue solicitando entrada do usuário até que a resposta seja “sair”.

\item {} 
\sphinxAtStartPar
\sphinxstylestrong{Execução Baseada em Condição Dinâmica:}

\begin{sphinxVerbatim}[commandchars=\\\{\}]
\PYG{n}{limite} \PYG{o}{=} \PYG{l+m+mi}{100}
\PYG{n}{soma} \PYG{o}{=} \PYG{l+m+mi}{0}
\PYG{n}{contador} \PYG{o}{=} \PYG{l+m+mi}{1}
\PYG{k}{while} \PYG{n}{soma} \PYG{o}{\PYGZlt{}} \PYG{n}{limite}\PYG{p}{:}
    \PYG{n}{soma} \PYG{o}{+}\PYG{o}{=} \PYG{n}{contador}
    \PYG{n}{contador} \PYG{o}{+}\PYG{o}{=} \PYG{l+m+mi}{1}
\end{sphinxVerbatim}

\sphinxAtStartPar
Este exemplo ilustra como o \sphinxcode{\sphinxupquote{while}} pode ser usado para realizar iterações com base em condições que podem ser determinadas dinamicamente.

\item {} 
\sphinxAtStartPar
\sphinxstylestrong{Validação de Entrada:}

\begin{sphinxVerbatim}[commandchars=\\\{\}]
\PYG{n}{nota} \PYG{o}{=} \PYG{o}{\PYGZhy{}}\PYG{l+m+mi}{1}
\PYG{k}{while} \PYG{n}{nota} \PYG{o}{\PYGZlt{}} \PYG{l+m+mi}{0} \PYG{o+ow}{or} \PYG{n}{nota} \PYG{o}{\PYGZgt{}} \PYG{l+m+mi}{100}\PYG{p}{:}
    \PYG{n}{nota} \PYG{o}{=} \PYG{n+nb}{int}\PYG{p}{(}\PYG{n+nb}{input}\PYG{p}{(}\PYG{l+s+s2}{\PYGZdq{}}\PYG{l+s+s2}{Digite uma nota entre 0 e 100: }\PYG{l+s+s2}{\PYGZdq{}}\PYG{p}{)}\PYG{p}{)}
\end{sphinxVerbatim}

\sphinxAtStartPar
O loop \sphinxcode{\sphinxupquote{while}} garante que o programa só prossiga quando uma entrada válida é fornecida pelo usuário.

\end{enumerate}

\sphinxAtStartPar
A estrutura \sphinxcode{\sphinxupquote{while}} é uma ferramenta poderosa quando o número de iterações não é fixo antecipadamente, proporcionando flexibilidade na execução de código em situações dinâmicas.


\section{Exercícios}
\label{\detokenize{chapters/ch4/ch4:exercicios}}\begin{enumerate}
\sphinxsetlistlabels{\arabic}{enumi}{enumii}{}{.}%
\item {} 
\sphinxAtStartPar
\sphinxstylestrong{Verificação de Idade}:
Crie um programa que, ao receber a idade do usuário, imprima “Você é maior de idade” se a idade for 18 anos ou mais; caso contrário, imprima “Você é menor de idade”.

\item {} 
\sphinxAtStartPar
\sphinxstylestrong{Calculadora de Bônus}:
Faça um programa que, ao solicitar o salário de um funcionário, calcule e imprima um bônus de 10\% se o salário for inferior a R\$ 2000.

\item {} 
\sphinxAtStartPar
\sphinxstylestrong{Média de Notas}:
Desenvolva um programa que, ao receber três notas do usuário, calcule a média. Em seguida, imprima se o aluno foi aprovado (média maior ou igual a 7) ou reprovado.

\item {} 
\sphinxAtStartPar
\sphinxstylestrong{Contagem Regressiva}:
Escreva um programa que imprima uma contagem regressiva de 10 a 1 usando um loop while. Por exemplo, “Contagem regressiva: 10, 9, 8, …, 1”.

\item {} 
\sphinxAtStartPar
\sphinxstylestrong{Verificação de Palíndromo}:
Crie um programa que, ao solicitar ao usuário uma palavra, verifique se é um palíndromo (pode ser lida da mesma forma de trás para frente) e imprima o resultado.

\item {} 
\sphinxAtStartPar
\sphinxstylestrong{Tabuada Personalizada}:
Peça ao usuário para fornecer um número. O programa deve imprimir a tabuada desse número de 1 a 10. Por exemplo, se o usuário inserir 5, o programa imprimirá “5 x 1 = 5”, “5 x 2 = 10”, …, “5 x 10 = 50”.

\item {} 
\sphinxAtStartPar
\sphinxstylestrong{Validação de Senha}:
Desenvolva um programa que, ao solicitar ao usuário criar uma senha, valide se ela tem pelo menos 8 caracteres e inclui pelo menos um número. Dê feedback ao usuário sobre a validação.

\item {} 
\sphinxAtStartPar
\sphinxstylestrong{Jogo da Adivinhação}:
Crie um jogo em que o programa gere um número aleatório entre 1 e 100. O usuário deve tentar adivinhar o número, e o programa deve fornecer dicas sobre se o número é maior ou menor após cada tentativa.

\item {} 
\sphinxAtStartPar
\sphinxstylestrong{Fatorial}:
Implemente um programa que, ao solicitar ao usuário inserir um número, calcule e imprima o fatorial desse número. O fatorial de um número é o produto de todos os números inteiros de 1 até o próprio número.

\item {} 
\sphinxAtStartPar
\sphinxstylestrong{Classificação de Triângulos}:
Solicite ao usuário os comprimentos dos lados de um triângulo. O programa deve classificar o triângulo como equilátero (todos os lados iguais), isósceles (dois lados iguais) ou escaleno (nenhum lado igual).

\end{enumerate}

\sphinxstepscope


\chapter{Capítulo 5: Funções}
\label{\detokenize{chapters/ch5/ch5:capitulo-5-funcoes}}\label{\detokenize{chapters/ch5/ch5::doc}}
\sphinxAtStartPar
As funções desempenham um papel fundamental na estruturação de código em Python. Elas permitem que você divida seu programa em partes gerenciáveis e reutilizáveis, facilitando a manutenção e compreensão do código. Neste capítulo, exploraremos os conceitos essenciais relacionados a funções em Python.


\section{Definindo Funções}
\label{\detokenize{chapters/ch5/ch5:definindo-funcoes}}
\sphinxAtStartPar
Uma função é um bloco de código reutilizável que executa uma tarefa específica. Ela é definida usando a palavra\sphinxhyphen{}chave \sphinxcode{\sphinxupquote{def}}, seguida do nome da função e parênteses contendo os parâmetros. O bloco de código da função é indentado.

\sphinxAtStartPar
\sphinxstylestrong{Exemplo:}

\begin{sphinxVerbatim}[commandchars=\\\{\}]
\PYG{k}{def} \PYG{n+nf}{saudacao}\PYG{p}{(}\PYG{n}{nome\PYGZus{}pessoa}\PYG{p}{)}\PYG{p}{:}
\PYG{+w}{    }\PYG{l+s+sd}{\PYGZdq{}\PYGZdq{}\PYGZdq{}}
\PYG{l+s+sd}{    Esta função imprime uma saudação personalizada.}

\PYG{l+s+sd}{    :param nome\PYGZus{}pessoa: Uma string representando o nome da pessoa a ser saudada.}
\PYG{l+s+sd}{    \PYGZdq{}\PYGZdq{}\PYGZdq{}}
    \PYG{n+nb}{print}\PYG{p}{(}\PYG{l+s+sa}{f}\PYG{l+s+s2}{\PYGZdq{}}\PYG{l+s+s2}{Olá, }\PYG{l+s+si}{\PYGZob{}}\PYG{n}{nome\PYGZus{}pessoa}\PYG{l+s+si}{\PYGZcb{}}\PYG{l+s+s2}{!}\PYG{l+s+s2}{\PYGZdq{}}\PYG{p}{)}

\PYG{c+c1}{\PYGZsh{} Chamando a função}
\PYG{n}{saudacao}\PYG{p}{(}\PYG{l+s+s2}{\PYGZdq{}}\PYG{l+s+s2}{Alice}\PYG{l+s+s2}{\PYGZdq{}}\PYG{p}{)}
\end{sphinxVerbatim}

\begin{sphinxVerbatim}[commandchars=\\\{\}]
Olá, Alice!
\end{sphinxVerbatim}

\sphinxAtStartPar
A função \sphinxcode{\sphinxupquote{saudacao}} imprime uma saudação personalizada com base no nome fornecido como argumento. No exemplo, ao chamar a função com “Alice”, ela exibe “Olá, Alice!” na tela. Essa função realiza uma saudação personalizada.

\sphinxAtStartPar
Ao definir funções em Python, além de compreender sua estrutura básica, é fundamental adotar boas práticas para a nomenclatura de funções e parâmetros. Seguir diretrizes de nomenclatura aprimora a legibilidade do código e facilita a colaboração em projetos. Algumas regras essenciais incluem:


\subsection{Nomes de Funções:}
\label{\detokenize{chapters/ch5/ch5:nomes-de-funcoes}}\begin{itemize}
\item {} 
\sphinxAtStartPar
\sphinxstylestrong{Convenção Snake Case:} Utilize letras minúsculas e underline para separar palavras. Exemplo: \sphinxcode{\sphinxupquote{calcular\_media}}, \sphinxcode{\sphinxupquote{processar\_dados}}.

\item {} 
\sphinxAtStartPar
\sphinxstylestrong{Clareza e Descritividade:} Escolha nomes que claramente descrevam a função desempenhada, proporcionando entendimento imediato do propósito da função.

\item {} 
\sphinxAtStartPar
\sphinxstylestrong{Evite Palavras Reservadas:} Não utilize nomes que são palavras reservadas em Python, como \sphinxcode{\sphinxupquote{print}}, \sphinxcode{\sphinxupquote{if}}, \sphinxcode{\sphinxupquote{else}}, entre outras.

\item {} 
\sphinxAtStartPar
\sphinxstylestrong{Convenção de Documentação:} Siga a PEP 257 para docstrings. Forneça uma breve descrição do propósito da função, especificando tipos de parâmetros e de retorno, quando aplicável.

\end{itemize}


\subsection{Nomes de Parâmetros:}
\label{\detokenize{chapters/ch5/ch5:nomes-de-parametros}}\begin{itemize}
\item {} 
\sphinxAtStartPar
\sphinxstylestrong{Convenção Snake Case:} Mantenha a consistência na nomenclatura, utilizando letras minúsculas e underline para separar palavras. Exemplo: \sphinxcode{\sphinxupquote{nome\_completo}}, \sphinxcode{\sphinxupquote{valor\_total}}.

\item {} 
\sphinxAtStartPar
\sphinxstylestrong{Descrição Significativa:} Escolha nomes de parâmetros que indiquem claramente sua função na execução da função.

\end{itemize}


\section{Argumentos de Função}
\label{\detokenize{chapters/ch5/ch5:argumentos-de-funcao}}
\sphinxAtStartPar
Os argumentos são valores fornecidos a uma função quando ela é chamada. Uma função pode ter parâmetros, que são variáveis usadas para receber esses argumentos.

\sphinxAtStartPar
\sphinxstylestrong{Exemplo:}

\begin{sphinxVerbatim}[commandchars=\\\{\}]
\PYG{k}{def} \PYG{n+nf}{soma}\PYG{p}{(}\PYG{n}{a}\PYG{p}{,} \PYG{n}{b}\PYG{p}{)}\PYG{p}{:}
\PYG{+w}{    }\PYG{l+s+sd}{\PYGZdq{}\PYGZdq{}\PYGZdq{}Esta função retorna a soma de dois números.\PYGZdq{}\PYGZdq{}\PYGZdq{}}
    \PYG{n}{resultado} \PYG{o}{=} \PYG{n}{a} \PYG{o}{+} \PYG{n}{b}
    \PYG{k}{return} \PYG{n}{resultado}

\PYG{c+c1}{\PYGZsh{} Chamando a função com argumentos}
\PYG{n}{resultado\PYGZus{}soma} \PYG{o}{=} \PYG{n}{soma}\PYG{p}{(}\PYG{l+m+mi}{3}\PYG{p}{,} \PYG{l+m+mi}{5}\PYG{p}{)}
\PYG{n+nb}{print}\PYG{p}{(}\PYG{l+s+sa}{f}\PYG{l+s+s2}{\PYGZdq{}}\PYG{l+s+s2}{A soma é: }\PYG{l+s+si}{\PYGZob{}}\PYG{n}{resultado\PYGZus{}soma}\PYG{l+s+si}{\PYGZcb{}}\PYG{l+s+s2}{\PYGZdq{}}\PYG{p}{)}
\end{sphinxVerbatim}

\begin{sphinxVerbatim}[commandchars=\\\{\}]
\PYG{n}{A} \PYG{n}{soma} \PYG{n}{é}\PYG{p}{:} \PYG{l+m+mi}{8}
\end{sphinxVerbatim}

\sphinxAtStartPar
A função \sphinxcode{\sphinxupquote{soma}} ilustra o conceito de parâmetros posicionais, onde os argumentos são correspondidos à ordem dos parâmetros na função. No exemplo, o argumento \sphinxcode{\sphinxupquote{3}} corresponde a \sphinxcode{\sphinxupquote{a}} e o argumento \sphinxcode{\sphinxupquote{5}} corresponde a \sphinxcode{\sphinxupquote{b}}.

\sphinxAtStartPar
Além dos parâmetros posicionais, as funções em Python podem ter parâmetros chave\sphinxhyphen{}valor, também conhecidos como argumentos nomeados. Isso permite que você especifique explicitamente para qual parâmetro está passando um determinado valor, tornando a chamada da função mais explícita. Exemplo:

\begin{sphinxVerbatim}[commandchars=\\\{\}]
\PYG{k}{def} \PYG{n+nf}{saudacao}\PYG{p}{(}\PYG{n}{nome}\PYG{p}{,} \PYG{n}{mensagem}\PYG{o}{=}\PYG{l+s+s2}{\PYGZdq{}}\PYG{l+s+s2}{Olá}\PYG{l+s+s2}{\PYGZdq{}}\PYG{p}{)}\PYG{p}{:}
\PYG{+w}{    }\PYG{l+s+sd}{\PYGZdq{}\PYGZdq{}\PYGZdq{}Esta função exibe uma saudação com uma mensagem opcional.\PYGZdq{}\PYGZdq{}\PYGZdq{}}
    \PYG{n+nb}{print}\PYG{p}{(}\PYG{l+s+sa}{f}\PYG{l+s+s2}{\PYGZdq{}}\PYG{l+s+si}{\PYGZob{}}\PYG{n}{mensagem}\PYG{l+s+si}{\PYGZcb{}}\PYG{l+s+s2}{, }\PYG{l+s+si}{\PYGZob{}}\PYG{n}{nome}\PYG{l+s+si}{\PYGZcb{}}\PYG{l+s+s2}{!}\PYG{l+s+s2}{\PYGZdq{}}\PYG{p}{)}

\PYG{c+c1}{\PYGZsh{} Chamando a função com argumentos nomeados}
\PYG{n}{saudacao}\PYG{p}{(}\PYG{n}{nome}\PYG{o}{=}\PYG{l+s+s2}{\PYGZdq{}}\PYG{l+s+s2}{Alice}\PYG{l+s+s2}{\PYGZdq{}}\PYG{p}{,} \PYG{n}{mensagem}\PYG{o}{=}\PYG{l+s+s2}{\PYGZdq{}}\PYG{l+s+s2}{Bom dia}\PYG{l+s+s2}{\PYGZdq{}}\PYG{p}{)}
\end{sphinxVerbatim}

\begin{sphinxVerbatim}[commandchars=\\\{\}]
Bom dia, Alice!
\end{sphinxVerbatim}

\sphinxAtStartPar
Neste exemplo, \sphinxcode{\sphinxupquote{nome}} é um parâmetro posicional, enquanto \sphinxcode{\sphinxupquote{mensagem}} é um parâmetro chave\sphinxhyphen{}valor com um valor padrão de “Olá”. Ao chamar a função, podemos especificar \sphinxcode{\sphinxupquote{mensagem}} de forma explícita, como em \sphinxcode{\sphinxupquote{mensagem="Bom dia"}}. Isso oferece flexibilidade e clareza na passagem de argumentos para funções.


\section{Retorno de Valor}
\label{\detokenize{chapters/ch5/ch5:retorno-de-valor}}
\sphinxAtStartPar
O retorno de valor permite que uma função envie um resultado de volta ao ponto de chamada. Isso é feito usando a palavra\sphinxhyphen{}chave \sphinxcode{\sphinxupquote{return}}.

\sphinxAtStartPar
\sphinxstylestrong{Exemplo:}

\begin{sphinxVerbatim}[commandchars=\\\{\}]
\PYG{k}{def} \PYG{n+nf}{quadrado}\PYG{p}{(}\PYG{n}{x}\PYG{p}{)}\PYG{p}{:}
\PYG{+w}{    }\PYG{l+s+sd}{\PYGZdq{}\PYGZdq{}\PYGZdq{}Esta função retorna o quadrado de um número.\PYGZdq{}\PYGZdq{}\PYGZdq{}}
    \PYG{k}{return} \PYG{n}{x} \PYG{o}{*}\PYG{o}{*} \PYG{l+m+mi}{2}

\PYG{c+c1}{\PYGZsh{} Chamando a função e usando o valor retornado}
\PYG{n}{valor\PYGZus{}quadrado} \PYG{o}{=} \PYG{n}{quadrado}\PYG{p}{(}\PYG{l+m+mi}{4}\PYG{p}{)}
\PYG{n+nb}{print}\PYG{p}{(}\PYG{l+s+sa}{f}\PYG{l+s+s2}{\PYGZdq{}}\PYG{l+s+s2}{O quadrado é: }\PYG{l+s+si}{\PYGZob{}}\PYG{n}{valor\PYGZus{}quadrado}\PYG{l+s+si}{\PYGZcb{}}\PYG{l+s+s2}{\PYGZdq{}}\PYG{p}{)}
\end{sphinxVerbatim}

\begin{sphinxVerbatim}[commandchars=\\\{\}]
\PYG{n}{O} \PYG{n}{quadrado} \PYG{n}{é}\PYG{p}{:} \PYG{l+m+mi}{16}
\end{sphinxVerbatim}

\sphinxAtStartPar
O código define uma função chamada \sphinxcode{\sphinxupquote{quadrado}} que calcula o quadrado de um número. Em seguida, a função é chamada com o argumento 4, e o resultado é armazenado na variável \sphinxcode{\sphinxupquote{valor\_quadrado}}, que é então impressa, mostrando o quadrado do número 4.


\section{Funções Anônimas (Lambda)}
\label{\detokenize{chapters/ch5/ch5:funcoes-anonimas-lambda}}
\sphinxAtStartPar
As funções anônimas, ou lambdas, são funções pequenas e temporárias definidas sem um nome formal. Elas são criadas usando a palavra\sphinxhyphen{}chave \sphinxcode{\sphinxupquote{lambda}}.

\sphinxAtStartPar
\sphinxstylestrong{Exemplo:}

\begin{sphinxVerbatim}[commandchars=\\\{\}]
\PYG{n}{dobro} \PYG{o}{=} \PYG{k}{lambda} \PYG{n}{x}\PYG{p}{:} \PYG{n}{x} \PYG{o}{*} \PYG{l+m+mi}{2}
\PYG{n+nb}{print}\PYG{p}{(}\PYG{l+s+sa}{f}\PYG{l+s+s2}{\PYGZdq{}}\PYG{l+s+s2}{O dobro de 3 é: }\PYG{l+s+si}{\PYGZob{}}\PYG{n}{dobro}\PYG{p}{(}\PYG{l+m+mi}{3}\PYG{p}{)}\PYG{l+s+si}{\PYGZcb{}}\PYG{l+s+s2}{\PYGZdq{}}\PYG{p}{)}
\end{sphinxVerbatim}

\begin{sphinxVerbatim}[commandchars=\\\{\}]
\PYG{n}{O} \PYG{n}{dobro} \PYG{n}{de} \PYG{l+m+mi}{3} \PYG{n}{é}\PYG{p}{:} \PYG{l+m+mi}{6}
\end{sphinxVerbatim}

\sphinxAtStartPar
O código utilizou uma função lambda para calcular o dobro de um número. Na última linha, o código aplica essa função ao número 3 e imprime o resultado, demonstrando como calcular e exibir o dobro de um valor específico.


\section{Funções Recursivas}
\label{\detokenize{chapters/ch5/ch5:funcoes-recursivas}}
\sphinxAtStartPar
Funções recursivas são aquelas que chamam a si mesmas durante a execução. Isso é útil para resolver problemas que podem ser quebrados em casos menores do mesmo problema.

\sphinxAtStartPar
\sphinxstylestrong{Exemplo:}

\begin{sphinxVerbatim}[commandchars=\\\{\}]
\PYG{k}{def} \PYG{n+nf}{fatorial}\PYG{p}{(}\PYG{n}{n}\PYG{p}{)}\PYG{p}{:}
\PYG{+w}{    }\PYG{l+s+sd}{\PYGZdq{}\PYGZdq{}\PYGZdq{}Esta função retorna o fatorial de um número.\PYGZdq{}\PYGZdq{}\PYGZdq{}}
    \PYG{k}{if} \PYG{n}{n} \PYG{o}{==} \PYG{l+m+mi}{0} \PYG{o+ow}{or} \PYG{n}{n} \PYG{o}{==} \PYG{l+m+mi}{1}\PYG{p}{:}
        \PYG{k}{return} \PYG{l+m+mi}{1}
    \PYG{k}{else}\PYG{p}{:}
        \PYG{k}{return} \PYG{n}{n} \PYG{o}{*} \PYG{n}{fatorial}\PYG{p}{(}\PYG{n}{n} \PYG{o}{\PYGZhy{}} \PYG{l+m+mi}{1}\PYG{p}{)}

\PYG{c+c1}{\PYGZsh{} Chamando a função recursiva}
\PYG{n}{resultado\PYGZus{}fatorial} \PYG{o}{=} \PYG{n}{fatorial}\PYG{p}{(}\PYG{l+m+mi}{5}\PYG{p}{)}
\PYG{n+nb}{print}\PYG{p}{(}\PYG{l+s+sa}{f}\PYG{l+s+s2}{\PYGZdq{}}\PYG{l+s+s2}{O fatorial é: }\PYG{l+s+si}{\PYGZob{}}\PYG{n}{resultado\PYGZus{}fatorial}\PYG{l+s+si}{\PYGZcb{}}\PYG{l+s+s2}{\PYGZdq{}}\PYG{p}{)}
\end{sphinxVerbatim}

\begin{sphinxVerbatim}[commandchars=\\\{\}]
\PYG{n}{O} \PYG{n}{fatorial} \PYG{n}{é}\PYG{p}{:} \PYG{l+m+mi}{120}
\end{sphinxVerbatim}

\sphinxAtStartPar
A função \sphinxcode{\sphinxupquote{fatorial}} é uma função recursiva que calcula o fatorial de um número. Se o número for 0 ou 1, ela retorna 1; caso contrário, ela multiplica o número pelo fatorial do número anterior, chamando a si mesma de forma recursiva. No exemplo, a função é chamada com o argumento 5, resultando no cálculo do fatorial, que é impresso como “O fatorial é: 120”.


\section{Funções \sphinxstyleemphasis{built\sphinxhyphen{}in}}
\label{\detokenize{chapters/ch5/ch5:funcoes-built-in}}
\sphinxAtStartPar
As funções \sphinxstyleemphasis{built\sphinxhyphen{}in} do Python são funções que estão na própria linguagem e estão sempre disponíveis para uso, sem a necessidade de importar módulos específicos. Essas funções fornecem funcionalidades essenciais que são amplamente utilizadas. Algumas delas incluem:
\begin{itemize}
\item {} 
\sphinxAtStartPar
\sphinxstylestrong{\sphinxcode{\sphinxupquote{print()}}:} Imprime mensagens ou valores na saída padrão.

\begin{sphinxVerbatim}[commandchars=\\\{\}]
\PYG{n+nb}{print}\PYG{p}{(}\PYG{l+s+s2}{\PYGZdq{}}\PYG{l+s+s2}{Olá, mundo!}\PYG{l+s+s2}{\PYGZdq{}}\PYG{p}{)}
\end{sphinxVerbatim}

\begin{sphinxVerbatim}[commandchars=\\\{\}]
Olá, mundo!
\end{sphinxVerbatim}

\item {} 
\sphinxAtStartPar
\sphinxstylestrong{\sphinxcode{\sphinxupquote{len()}}:} Retorna o número de elementos em uma sequência (como uma lista, tupla ou string).

\begin{sphinxVerbatim}[commandchars=\\\{\}]
\PYG{n}{tamanho} \PYG{o}{=} \PYG{n+nb}{len}\PYG{p}{(}\PYG{l+s+s2}{\PYGZdq{}}\PYG{l+s+s2}{Python}\PYG{l+s+s2}{\PYGZdq{}}\PYG{p}{)}
\PYG{n+nb}{print}\PYG{p}{(}\PYG{l+s+sa}{f}\PYG{l+s+s2}{\PYGZdq{}}\PYG{l+s+s2}{O tamanho da string é: }\PYG{l+s+si}{\PYGZob{}}\PYG{n}{tamanho}\PYG{l+s+si}{\PYGZcb{}}\PYG{l+s+s2}{\PYGZdq{}}\PYG{p}{)}
\end{sphinxVerbatim}

\begin{sphinxVerbatim}[commandchars=\\\{\}]
\PYG{n}{O} \PYG{n}{tamanho} \PYG{n}{da} \PYG{n}{string} \PYG{n}{é}\PYG{p}{:} \PYG{l+m+mi}{6}
\end{sphinxVerbatim}

\item {} 
\sphinxAtStartPar
\sphinxstylestrong{\sphinxcode{\sphinxupquote{type()}}:} Retorna o tipo de um objeto.

\begin{sphinxVerbatim}[commandchars=\\\{\}]
\PYG{n}{tipo} \PYG{o}{=} \PYG{n+nb}{type}\PYG{p}{(}\PYG{l+m+mi}{42}\PYG{p}{)}
\PYG{n+nb}{print}\PYG{p}{(}\PYG{l+s+sa}{f}\PYG{l+s+s2}{\PYGZdq{}}\PYG{l+s+s2}{O tipo do objeto é: }\PYG{l+s+si}{\PYGZob{}}\PYG{n}{tipo}\PYG{l+s+si}{\PYGZcb{}}\PYG{l+s+s2}{\PYGZdq{}}\PYG{p}{)}
\end{sphinxVerbatim}

\begin{sphinxVerbatim}[commandchars=\\\{\}]
O tipo do objeto é: \PYGZlt{}class \PYGZsq{}int\PYGZsq{}\PYGZgt{}
\end{sphinxVerbatim}

\end{itemize}


\subsection{Funções para Conversão de Tipos}
\label{\detokenize{chapters/ch5/ch5:funcoes-para-conversao-de-tipos}}\begin{itemize}
\item {} 
\sphinxAtStartPar
\sphinxstylestrong{\sphinxcode{\sphinxupquote{int()}}}, \sphinxstylestrong{\sphinxcode{\sphinxupquote{float()}}}, \sphinxstylestrong{\sphinxcode{\sphinxupquote{str()}}}, \sphinxstylestrong{\sphinxcode{\sphinxupquote{list()}}}, \sphinxstylestrong{\sphinxcode{\sphinxupquote{tuple()}}}, \sphinxstylestrong{\sphinxcode{\sphinxupquote{dict()}}}: Convertem valores entre diferentes tipos.

\begin{sphinxVerbatim}[commandchars=\\\{\}]
\PYG{n}{numero\PYGZus{}texto} \PYG{o}{=} \PYG{l+s+s2}{\PYGZdq{}}\PYG{l+s+s2}{123}\PYG{l+s+s2}{\PYGZdq{}}
\PYG{n}{numero\PYGZus{}inteiro} \PYG{o}{=} \PYG{n+nb}{int}\PYG{p}{(}\PYG{n}{numero\PYGZus{}texto}\PYG{p}{)}
\PYG{n+nb}{print}\PYG{p}{(}\PYG{l+s+sa}{f}\PYG{l+s+s2}{\PYGZdq{}}\PYG{l+s+s2}{O tipo agora é: }\PYG{l+s+si}{\PYGZob{}}\PYG{n+nb}{type}\PYG{p}{(}\PYG{n}{numero\PYGZus{}inteiro}\PYG{p}{)}\PYG{l+s+si}{\PYGZcb{}}\PYG{l+s+s2}{\PYGZdq{}}\PYG{p}{)}
\end{sphinxVerbatim}

\begin{sphinxVerbatim}[commandchars=\\\{\}]
O tipo agora é: \PYGZlt{}class \PYGZsq{}int\PYGZsq{}\PYGZgt{}
\end{sphinxVerbatim}

\end{itemize}


\subsection{Funções para Manipulação de Sequências}
\label{\detokenize{chapters/ch5/ch5:funcoes-para-manipulacao-de-sequencias}}\begin{itemize}
\item {} 
\sphinxAtStartPar
\sphinxstylestrong{\sphinxcode{\sphinxupquote{max()}}}, \sphinxstylestrong{\sphinxcode{\sphinxupquote{min()}}}: Encontram o valor máximo e mínimo em uma sequência.

\begin{sphinxVerbatim}[commandchars=\\\{\}]
\PYG{n}{numeros} \PYG{o}{=} \PYG{p}{[}\PYG{l+m+mi}{2}\PYG{p}{,} \PYG{l+m+mi}{8}\PYG{p}{,} \PYG{l+m+mi}{5}\PYG{p}{,} \PYG{l+m+mi}{10}\PYG{p}{,} \PYG{l+m+mi}{3}\PYG{p}{]}
\PYG{n}{maximo} \PYG{o}{=} \PYG{n+nb}{max}\PYG{p}{(}\PYG{n}{numeros}\PYG{p}{)}
\PYG{n+nb}{print}\PYG{p}{(}\PYG{l+s+sa}{f}\PYG{l+s+s2}{\PYGZdq{}}\PYG{l+s+s2}{O valor máximo é: }\PYG{l+s+si}{\PYGZob{}}\PYG{n}{maximo}\PYG{l+s+si}{\PYGZcb{}}\PYG{l+s+s2}{\PYGZdq{}}\PYG{p}{)}
\end{sphinxVerbatim}

\begin{sphinxVerbatim}[commandchars=\\\{\}]
\PYG{n}{O} \PYG{n}{valor} \PYG{n}{máximo} \PYG{n}{é}\PYG{p}{:} \PYG{l+m+mi}{10}
\end{sphinxVerbatim}

\item {} 
\sphinxAtStartPar
\sphinxstylestrong{\sphinxcode{\sphinxupquote{sum()}}}: Retorna a soma dos elementos em uma sequência numérica.

\begin{sphinxVerbatim}[commandchars=\\\{\}]
\PYG{n}{total} \PYG{o}{=} \PYG{n+nb}{sum}\PYG{p}{(}\PYG{n}{numeros}\PYG{p}{)}
\PYG{n+nb}{print}\PYG{p}{(}\PYG{l+s+sa}{f}\PYG{l+s+s2}{\PYGZdq{}}\PYG{l+s+s2}{A soma dos elementos é: }\PYG{l+s+si}{\PYGZob{}}\PYG{n}{total}\PYG{l+s+si}{\PYGZcb{}}\PYG{l+s+s2}{\PYGZdq{}}\PYG{p}{)}
\end{sphinxVerbatim}

\begin{sphinxVerbatim}[commandchars=\\\{\}]
\PYG{n}{A} \PYG{n}{soma} \PYG{n}{dos} \PYG{n}{elementos} \PYG{n}{é}\PYG{p}{:} \PYG{l+m+mi}{28}
\end{sphinxVerbatim}

\end{itemize}


\subsection{Funções para Interagir com o Usuário}
\label{\detokenize{chapters/ch5/ch5:funcoes-para-interagir-com-o-usuario}}\begin{itemize}
\item {} 
\sphinxAtStartPar
\sphinxstylestrong{\sphinxcode{\sphinxupquote{input()}}}: Solicita entrada do usuário.

\begin{sphinxVerbatim}[commandchars=\\\{\}]
\PYG{n}{nome\PYGZus{}usuario} \PYG{o}{=} \PYG{n+nb}{input}\PYG{p}{(}\PYG{l+s+s2}{\PYGZdq{}}\PYG{l+s+s2}{Digite seu nome: }\PYG{l+s+s2}{\PYGZdq{}}\PYG{p}{)}
\PYG{n+nb}{print}\PYG{p}{(}\PYG{l+s+sa}{f}\PYG{l+s+s2}{\PYGZdq{}}\PYG{l+s+s2}{Olá, }\PYG{l+s+si}{\PYGZob{}}\PYG{n}{nome\PYGZus{}usuario}\PYG{l+s+si}{\PYGZcb{}}\PYG{l+s+s2}{!}\PYG{l+s+s2}{\PYGZdq{}}\PYG{p}{)}
\end{sphinxVerbatim}

\begin{sphinxVerbatim}[commandchars=\\\{\}]
Digite seu nome: Ana
Olá, Ana!
\end{sphinxVerbatim}

\end{itemize}


\subsection{Funções Matemáticas}
\label{\detokenize{chapters/ch5/ch5:funcoes-matematicas}}\begin{itemize}
\item {} 
\sphinxAtStartPar
\sphinxstylestrong{\sphinxcode{\sphinxupquote{abs()}}}: Retorna o valor absoluto de um número.

\begin{sphinxVerbatim}[commandchars=\\\{\}]
\PYG{n}{numero\PYGZus{}negativo} \PYG{o}{=} \PYG{o}{\PYGZhy{}}\PYG{l+m+mi}{5}
\PYG{n}{absoluto} \PYG{o}{=} \PYG{n+nb}{abs}\PYG{p}{(}\PYG{n}{numero\PYGZus{}negativo}\PYG{p}{)}
\PYG{n+nb}{print}\PYG{p}{(}\PYG{l+s+sa}{f}\PYG{l+s+s2}{\PYGZdq{}}\PYG{l+s+s2}{O valor absoluto é: }\PYG{l+s+si}{\PYGZob{}}\PYG{n}{absoluto}\PYG{l+s+si}{\PYGZcb{}}\PYG{l+s+s2}{\PYGZdq{}}\PYG{p}{)}
\end{sphinxVerbatim}

\begin{sphinxVerbatim}[commandchars=\\\{\}]
\PYG{n}{O} \PYG{n}{valor} \PYG{n}{absoluto} \PYG{n}{é}\PYG{p}{:} \PYG{l+m+mi}{5}

\end{sphinxVerbatim}

\item {} 
\sphinxAtStartPar
\sphinxstylestrong{\sphinxcode{\sphinxupquote{pow()}}}, \sphinxstylestrong{\sphinxcode{\sphinxupquote{sqrt()}}}: Potenciação e raiz quadrada.

\begin{sphinxVerbatim}[commandchars=\\\{\}]
\PYG{n}{quadrado} \PYG{o}{=} \PYG{n+nb}{pow}\PYG{p}{(}\PYG{l+m+mi}{3}\PYG{p}{,} \PYG{l+m+mi}{2}\PYG{p}{)}
\PYG{n}{raiz\PYGZus{}quadrada} \PYG{o}{=} \PYG{n}{sqrt}\PYG{p}{(}\PYG{l+m+mi}{9}\PYG{p}{)}
\end{sphinxVerbatim}

\begin{sphinxVerbatim}[commandchars=\\\{\}]
\PYG{n}{O} \PYG{n}{quadrado} \PYG{n}{é}\PYG{p}{:} \PYG{l+m+mi}{9}
\end{sphinxVerbatim}

\end{itemize}

\sphinxAtStartPar
Ao compreender e utilizar essas funções incorporadas, você poderá tornar seu código mais eficiente e conciso, aproveitando as capacidades que o Python oferece por padrão.

\sphinxAtStartPar
\sphinxstylestrong{Observação}: Para utilizar a função \sphinxcode{\sphinxupquote{sqrt()}} (raiz quadrada) mencionada no exemplo, é necessário importar o módulo \sphinxcode{\sphinxupquote{math}} em seu código Python. A função \sphinxcode{\sphinxupquote{sqrt()}} faz parte do módulo \sphinxcode{\sphinxupquote{math}} e não está disponível por padrão. Portanto, antes de usar a função \sphinxcode{\sphinxupquote{sqrt()}}, inclua a seguinte linha de importação no início código:

\begin{sphinxVerbatim}[commandchars=\\\{\}]
\PYG{k+kn}{from} \PYG{n+nn}{math} \PYG{k+kn}{import} \PYG{n}{sqrt}
\end{sphinxVerbatim}


\section{Exercícios}
\label{\detokenize{chapters/ch5/ch5:exercicios}}\begin{enumerate}
\sphinxsetlistlabels{\arabic}{enumi}{enumii}{}{.}%
\item {} 
\sphinxAtStartPar
\sphinxstylestrong{Máximo da Lista}:
Crie uma lista de números inteiros. Utilize a função \sphinxcode{\sphinxupquote{max()}} para encontrar o valor máximo e imprima o resultado.

\item {} 
\sphinxAtStartPar
\sphinxstylestrong{Comprimento da String}:
Defina uma string, por exemplo, “Python”. Utilize a função \sphinxcode{\sphinxupquote{len()}} para calcular o número de caracteres na string e exiba o resultado.

\item {} 
\sphinxAtStartPar
\sphinxstylestrong{Conversão de Entrada}:
Solicite ao usuário que insira um número utilizando \sphinxcode{\sphinxupquote{input()}}. Converta o input para um número inteiro com \sphinxcode{\sphinxupquote{int()}} e imprima o resultado.

\item {} 
\sphinxAtStartPar
\sphinxstylestrong{Soma dos Primeiros Números}:
Crie uma lista com os primeiros cinco números naturais. Calcule a soma usando a função \sphinxcode{\sphinxupquote{sum()}} e exiba o resultado.

\item {} 
\sphinxAtStartPar
\sphinxstylestrong{Mínimo da Lista}:
Tenha uma lista de números. Encontre o valor mínimo utilizando \sphinxcode{\sphinxupquote{min()}} e imprima o resultado.

\item {} 
\sphinxAtStartPar
\sphinxstylestrong{Conversão de String para Inteiro}:
Dada a string “42”, converta\sphinxhyphen{}a para um número inteiro utilizando \sphinxcode{\sphinxupquote{int()}} e exiba o resultado.

\item {} 
\sphinxAtStartPar
\sphinxstylestrong{Conversão de Decimal}:
Peça ao usuário que insira um número decimal utilizando \sphinxcode{\sphinxupquote{input()}}. Converta o input para um número de ponto flutuante usando \sphinxcode{\sphinxupquote{float()}} e imprima o resultado.

\item {} 
\sphinxAtStartPar
\sphinxstylestrong{Valor Absoluto}:
Defina um número negativo. Utilize a função \sphinxcode{\sphinxupquote{abs()}} para encontrar o valor absoluto e imprima\sphinxhyphen{}o.

\item {} 
\sphinxAtStartPar
\sphinxstylestrong{Raiz Quadrada com Math}:
Importe o módulo \sphinxcode{\sphinxupquote{math}}. Calcule a raiz quadrada de 16 usando \sphinxcode{\sphinxupquote{sqrt()}} e imprima o resultado.

\item {} 
\sphinxAtStartPar
\sphinxstylestrong{Conversão de Lista para Caracteres}:
Crie uma lista com três strings. Converta essa lista em uma lista de caracteres utilizando \sphinxcode{\sphinxupquote{list()}} e exiba o resultado.

\end{enumerate}

\sphinxstepscope


\chapter{Capítulo 6: Strings}
\label{\detokenize{chapters/ch6/ch6:capitulo-6-strings}}\label{\detokenize{chapters/ch6/ch6::doc}}

\section{Introdução a Strings}
\label{\detokenize{chapters/ch6/ch6:introducao-a-strings}}
\sphinxAtStartPar
As strings em Python são sequências de caracteres, o que significa que são compostas por letras, números, símbolos ou espaços. Essas estruturas de dados permitem uma variedade de manipulações para atender às necessidades do desenvolvedor. Neste contexto, abordaremos operações fundamentais, métodos e técnicas essenciais de manipulação de strings.


\subsection{Operações com Strings}
\label{\detokenize{chapters/ch6/ch6:operacoes-com-strings}}

\subsubsection{Concatenação}
\label{\detokenize{chapters/ch6/ch6:concatenacao}}
\sphinxAtStartPar
A concatenação de strings é uma operação que combina duas ou mais strings em uma única string. Isso é feito utilizando o operador \sphinxcode{\sphinxupquote{+}}.

\sphinxAtStartPar
\sphinxstylestrong{Exemplo:}

\begin{sphinxVerbatim}[commandchars=\\\{\}]
\PYG{n}{string1} \PYG{o}{=} \PYG{l+s+s2}{\PYGZdq{}}\PYG{l+s+s2}{Olá, }\PYG{l+s+s2}{\PYGZdq{}}
\PYG{n}{string2} \PYG{o}{=} \PYG{l+s+s2}{\PYGZdq{}}\PYG{l+s+s2}{mundo!}\PYG{l+s+s2}{\PYGZdq{}}
\PYG{n}{concatenacao} \PYG{o}{=} \PYG{n}{string1} \PYG{o}{+} \PYG{n}{string2}
\PYG{n+nb}{print}\PYG{p}{(}\PYG{n}{concatenacao}\PYG{p}{)}
\end{sphinxVerbatim}

\begin{sphinxVerbatim}[commandchars=\\\{\}]
Olá, mundo!
\end{sphinxVerbatim}


\subsubsection{Repetição}
\label{\detokenize{chapters/ch6/ch6:repeticao}}
\sphinxAtStartPar
A repetição de strings envolve duplicar ou triplicar o conteúdo de uma string. Isso é alcançado pelo operador \sphinxcode{\sphinxupquote{*}}.

\sphinxAtStartPar
\sphinxstylestrong{Exemplo:}

\begin{sphinxVerbatim}[commandchars=\\\{\}]
\PYG{n}{repeticao} \PYG{o}{=} \PYG{l+s+s2}{\PYGZdq{}}\PYG{l+s+s2}{abc}\PYG{l+s+s2}{\PYGZdq{}} \PYG{o}{*} \PYG{l+m+mi}{3}
\PYG{n+nb}{print}\PYG{p}{(}\PYG{n}{repeticao}\PYG{p}{)}
\end{sphinxVerbatim}

\begin{sphinxVerbatim}[commandchars=\\\{\}]
\PYG{n}{abcabcabc}
\end{sphinxVerbatim}


\subsubsection{Indexação e Fatiamento (\sphinxstyleemphasis{Slicing})}
\label{\detokenize{chapters/ch6/ch6:indexacao-e-fatiamento-slicing}}
\sphinxAtStartPar
As strings são indexadas, o que significa que cada caractere tem uma posição única. A indexação permite acessar caracteres específicos. Além disso, o fatiamento (\sphinxstyleemphasis{slicing}) possibilita extrair partes específicas da string.

\sphinxAtStartPar
\sphinxstylestrong{Exemplo:}

\begin{sphinxVerbatim}[commandchars=\\\{\}]
\PYG{n}{fruta} \PYG{o}{=} \PYG{l+s+s2}{\PYGZdq{}}\PYG{l+s+s2}{banana}\PYG{l+s+s2}{\PYGZdq{}}
\PYG{n}{primeira\PYGZus{}letra} \PYG{o}{=} \PYG{n}{fruta}\PYG{p}{[}\PYG{l+m+mi}{0}\PYG{p}{]}
\PYG{n}{substr} \PYG{o}{=} \PYG{n}{fruta}\PYG{p}{[}\PYG{l+m+mi}{1}\PYG{p}{:}\PYG{l+m+mi}{4}\PYG{p}{]}
\PYG{n+nb}{print}\PYG{p}{(}\PYG{n}{primeira\PYGZus{}letra}\PYG{p}{)}
\PYG{n+nb}{print}\PYG{p}{(}\PYG{n}{substr}\PYG{p}{)}
\end{sphinxVerbatim}

\begin{sphinxVerbatim}[commandchars=\\\{\}]
\PYG{n}{b}
\PYG{n}{ana}
\end{sphinxVerbatim}


\subsection{Métodos de Strings}
\label{\detokenize{chapters/ch6/ch6:metodos-de-strings}}

\subsubsection{\sphinxstyleliteralintitle{\sphinxupquote{upper()}} e \sphinxstyleliteralintitle{\sphinxupquote{lower()}}}
\label{\detokenize{chapters/ch6/ch6:upper-e-lower}}
\sphinxAtStartPar
Os métodos \sphinxcode{\sphinxupquote{upper()}} e \sphinxcode{\sphinxupquote{lower()}} alteram o caso da string para maiúsculas e minúsculas, respectivamente.

\sphinxAtStartPar
\sphinxstylestrong{Exemplo:}

\begin{sphinxVerbatim}[commandchars=\\\{\}]
\PYG{n}{fruta} \PYG{o}{=} \PYG{l+s+s2}{\PYGZdq{}}\PYG{l+s+s2}{maçã}\PYG{l+s+s2}{\PYGZdq{}}
\PYG{n}{maiuscula} \PYG{o}{=} \PYG{n}{fruta}\PYG{o}{.}\PYG{n}{upper}\PYG{p}{(}\PYG{p}{)}
\PYG{n+nb}{print}\PYG{p}{(}\PYG{n}{maiuscula}\PYG{p}{)}
\end{sphinxVerbatim}

\begin{sphinxVerbatim}[commandchars=\\\{\}]
\PYG{n}{MAÇÃ}
\end{sphinxVerbatim}


\subsubsection{\sphinxstyleliteralintitle{\sphinxupquote{replace()}}}
\label{\detokenize{chapters/ch6/ch6:replace}}
\sphinxAtStartPar
O método \sphinxcode{\sphinxupquote{replace()}} substitui parte da string por outra.

\sphinxAtStartPar
\sphinxstylestrong{Exemplo:}

\begin{sphinxVerbatim}[commandchars=\\\{\}]
\PYG{n}{mensagem} \PYG{o}{=} \PYG{l+s+s2}{\PYGZdq{}}\PYG{l+s+s2}{Olá, mundo!}\PYG{l+s+s2}{\PYGZdq{}}
\PYG{n}{nova\PYGZus{}mensagem} \PYG{o}{=} \PYG{n}{mensagem}\PYG{o}{.}\PYG{n}{replace}\PYG{p}{(}\PYG{l+s+s2}{\PYGZdq{}}\PYG{l+s+s2}{mundo}\PYG{l+s+s2}{\PYGZdq{}}\PYG{p}{,} \PYG{l+s+s2}{\PYGZdq{}}\PYG{l+s+s2}{Python}\PYG{l+s+s2}{\PYGZdq{}}\PYG{p}{)}
\PYG{n+nb}{print}\PYG{p}{(}\PYG{n}{nova\PYGZus{}mensagem}\PYG{p}{)}
\end{sphinxVerbatim}

\begin{sphinxVerbatim}[commandchars=\\\{\}]
Olá, Python!
\end{sphinxVerbatim}


\subsubsection{\sphinxstyleliteralintitle{\sphinxupquote{split()}}}
\label{\detokenize{chapters/ch6/ch6:split}}
\sphinxAtStartPar
O método \sphinxcode{\sphinxupquote{split()}} divide uma string em uma lista de substrings com base em um delimitador.

\sphinxAtStartPar
\sphinxstylestrong{Exemplo:}

\begin{sphinxVerbatim}[commandchars=\\\{\}]
\PYG{n}{frase} \PYG{o}{=} \PYG{l+s+s2}{\PYGZdq{}}\PYG{l+s+s2}{Python é uma linguagem poderosa}\PYG{l+s+s2}{\PYGZdq{}}
\PYG{n}{palavras} \PYG{o}{=} \PYG{n}{frase}\PYG{o}{.}\PYG{n}{split}\PYG{p}{(}\PYG{l+s+s2}{\PYGZdq{}}\PYG{l+s+s2}{ }\PYG{l+s+s2}{\PYGZdq{}}\PYG{p}{)}
\PYG{n+nb}{print}\PYG{p}{(}\PYG{n}{palavras}\PYG{p}{)}
\end{sphinxVerbatim}

\begin{sphinxVerbatim}[commandchars=\\\{\}]
\PYG{p}{[}\PYG{l+s+s1}{\PYGZsq{}}\PYG{l+s+s1}{Python}\PYG{l+s+s1}{\PYGZsq{}}\PYG{p}{,} \PYG{l+s+s1}{\PYGZsq{}}\PYG{l+s+s1}{é}\PYG{l+s+s1}{\PYGZsq{}}\PYG{p}{,} \PYG{l+s+s1}{\PYGZsq{}}\PYG{l+s+s1}{uma}\PYG{l+s+s1}{\PYGZsq{}}\PYG{p}{,} \PYG{l+s+s1}{\PYGZsq{}}\PYG{l+s+s1}{linguagem}\PYG{l+s+s1}{\PYGZsq{}}\PYG{p}{,} \PYG{l+s+s1}{\PYGZsq{}}\PYG{l+s+s1}{poderosa}\PYG{l+s+s1}{\PYGZsq{}}\PYG{p}{]}
\end{sphinxVerbatim}


\subsection{Explorando mais Conceitos}
\label{\detokenize{chapters/ch6/ch6:explorando-mais-conceitos}}

\subsubsection{Formatação de Strings (f\sphinxhyphen{}strings)}
\label{\detokenize{chapters/ch6/ch6:formatacao-de-strings-f-strings}}
\sphinxAtStartPar
As f\sphinxhyphen{}strings são uma forma eficiente e legível de formatar strings com valores de variáveis.

\sphinxAtStartPar
\sphinxstylestrong{Exemplo:}

\begin{sphinxVerbatim}[commandchars=\\\{\}]
\PYG{n}{nome} \PYG{o}{=} \PYG{l+s+s2}{\PYGZdq{}}\PYG{l+s+s2}{Alice}\PYG{l+s+s2}{\PYGZdq{}}
\PYG{n}{idade} \PYG{o}{=} \PYG{l+m+mi}{25}
\PYG{n}{mensagem} \PYG{o}{=} \PYG{l+s+sa}{f}\PYG{l+s+s2}{\PYGZdq{}}\PYG{l+s+s2}{Olá, meu nome é }\PYG{l+s+si}{\PYGZob{}}\PYG{n}{nome}\PYG{l+s+si}{\PYGZcb{}}\PYG{l+s+s2}{ e tenho }\PYG{l+s+si}{\PYGZob{}}\PYG{n}{idade}\PYG{l+s+si}{\PYGZcb{}}\PYG{l+s+s2}{ anos.}\PYG{l+s+s2}{\PYGZdq{}}
\PYG{n+nb}{print}\PYG{p}{(}\PYG{n}{mensagem}\PYG{p}{)}
\end{sphinxVerbatim}

\begin{sphinxVerbatim}[commandchars=\\\{\}]
\PYG{n}{Olá}\PYG{p}{,} \PYG{n}{meu} \PYG{n}{nome} \PYG{n}{é} \PYG{n}{Alice} \PYG{n}{e} \PYG{n}{tenho} \PYG{l+m+mi}{25} \PYG{n}{anos}\PYG{o}{.}
\end{sphinxVerbatim}


\subsubsection{Métodos \sphinxstyleliteralintitle{\sphinxupquote{startswith()}} e \sphinxstyleliteralintitle{\sphinxupquote{endswith()}}}
\label{\detokenize{chapters/ch6/ch6:metodos-startswith-e-endswith}}
\sphinxAtStartPar
Esses métodos verificam se uma string começa ou termina com uma determinada substring.

\sphinxAtStartPar
\sphinxstylestrong{Exemplo:}

\begin{sphinxVerbatim}[commandchars=\\\{\}]
\PYG{n}{frase} \PYG{o}{=} \PYG{l+s+s2}{\PYGZdq{}}\PYG{l+s+s2}{Python é incrível!}\PYG{l+s+s2}{\PYGZdq{}}
\PYG{n+nb}{print}\PYG{p}{(}\PYG{n}{frase}\PYG{o}{.}\PYG{n}{startswith}\PYG{p}{(}\PYG{l+s+s2}{\PYGZdq{}}\PYG{l+s+s2}{Python}\PYG{l+s+s2}{\PYGZdq{}}\PYG{p}{)}\PYG{p}{)}
\PYG{n+nb}{print}\PYG{p}{(}\PYG{n}{frase}\PYG{o}{.}\PYG{n}{endswith}\PYG{p}{(}\PYG{l+s+s2}{\PYGZdq{}}\PYG{l+s+s2}{incrível}\PYG{l+s+s2}{\PYGZdq{}}\PYG{p}{)}\PYG{p}{)}
\end{sphinxVerbatim}

\begin{sphinxVerbatim}[commandchars=\\\{\}]
\PYG{k+kc}{True}
\PYG{k+kc}{False}
\end{sphinxVerbatim}


\section{Expressões Regulares em Python}
\label{\detokenize{chapters/ch6/ch6:expressoes-regulares-em-python}}
\sphinxAtStartPar
As expressões regulares, conhecidas como regex, são uma poderosa ferramenta para manipulação e busca em strings no Python, proporcionando flexibilidade e eficiência. O módulo \sphinxcode{\sphinxupquote{re}} oferece suporte para trabalhar com expressões regulares.


\subsection{Sintaxe Básica e Principais Funções}
\label{\detokenize{chapters/ch6/ch6:sintaxe-basica-e-principais-funcoes}}
\sphinxAtStartPar
Para começar, importamos o módulo \sphinxcode{\sphinxupquote{re}}. A seguir, vamos analisar o exemplo para entender como as funções \sphinxcode{\sphinxupquote{re.match}} e \sphinxcode{\sphinxupquote{re.search}} funcionam:

\sphinxAtStartPar
\sphinxstylestrong{Exemplo}

\begin{sphinxVerbatim}[commandchars=\\\{\}]
\PYG{k+kn}{import} \PYG{n+nn}{re}

\PYG{c+c1}{\PYGZsh{} Exemplo 1: Utilizando re.match para encontrar um padrão no início da string}
\PYG{n}{padrao\PYGZus{}match} \PYG{o}{=} \PYG{l+s+sa}{r}\PYG{l+s+s1}{\PYGZsq{}}\PYG{l+s+s1}{\PYGZbs{}}\PYG{l+s+s1}{d+}\PYG{l+s+s1}{\PYGZsq{}}
\PYG{n}{texto\PYGZus{}match} \PYG{o}{=} \PYG{l+s+s2}{\PYGZdq{}}\PYG{l+s+s2}{123 Python}\PYG{l+s+s2}{\PYGZdq{}}
\PYG{n}{resultado\PYGZus{}match} \PYG{o}{=} \PYG{n}{re}\PYG{o}{.}\PYG{n}{match}\PYG{p}{(}\PYG{n}{padrao\PYGZus{}match}\PYG{p}{,} \PYG{n}{texto\PYGZus{}match}\PYG{p}{)}
\PYG{n+nb}{print}\PYG{p}{(}\PYG{l+s+sa}{f}\PYG{l+s+s2}{\PYGZdq{}}\PYG{l+s+s2}{Exemplo 1 \PYGZhy{} Buscando padrão }\PYG{l+s+si}{\PYGZob{}}\PYG{n}{padrao\PYGZus{}match}\PYG{l+s+si}{\PYGZcb{}}\PYG{l+s+s2}{ no início do texto }\PYG{l+s+s2}{\PYGZsq{}}\PYG{l+s+si}{\PYGZob{}}\PYG{n}{texto\PYGZus{}match}\PYG{l+s+si}{\PYGZcb{}}\PYG{l+s+s2}{\PYGZsq{}}\PYG{l+s+s2}{:}\PYG{l+s+s2}{\PYGZdq{}}\PYG{p}{)}
\PYG{n+nb}{print}\PYG{p}{(}\PYG{l+s+s2}{\PYGZdq{}}\PYG{l+s+s2}{Resultado:}\PYG{l+s+s2}{\PYGZdq{}}\PYG{p}{,} \PYG{n}{resultado\PYGZus{}match}\PYG{o}{.}\PYG{n}{group}\PYG{p}{(}\PYG{p}{)} \PYG{k}{if} \PYG{n}{resultado\PYGZus{}match} \PYG{k}{else} \PYG{l+s+s2}{\PYGZdq{}}\PYG{l+s+s2}{Sem correspondência}\PYG{l+s+s2}{\PYGZdq{}}\PYG{p}{)}
\PYG{n+nb}{print}\PYG{p}{(}\PYG{p}{)}

\PYG{c+c1}{\PYGZsh{} Exemplo 2: Utilizando re.search para encontrar um padrão em qualquer lugar da string}
\PYG{n}{padrao\PYGZus{}search} \PYG{o}{=} \PYG{l+s+sa}{r}\PYG{l+s+s1}{\PYGZsq{}}\PYG{l+s+s1}{\PYGZbs{}}\PYG{l+s+s1}{w+}\PYG{l+s+s1}{\PYGZsq{}}
\PYG{n}{texto\PYGZus{}search} \PYG{o}{=} \PYG{l+s+s2}{\PYGZdq{}}\PYG{l+s+s2}{Python é uma linguagem poderosa}\PYG{l+s+s2}{\PYGZdq{}}
\PYG{n}{string\PYGZus{}busca} \PYG{o}{=} \PYG{l+s+s2}{\PYGZdq{}}\PYG{l+s+s2}{linguagem}\PYG{l+s+s2}{\PYGZdq{}}
\PYG{n}{resultado\PYGZus{}search} \PYG{o}{=} \PYG{n}{re}\PYG{o}{.}\PYG{n}{search}\PYG{p}{(}\PYG{n}{padrao\PYGZus{}search}\PYG{p}{,} \PYG{n}{texto\PYGZus{}search}\PYG{p}{)}
\PYG{n+nb}{print}\PYG{p}{(}\PYG{l+s+sa}{f}\PYG{l+s+s2}{\PYGZdq{}}\PYG{l+s+s2}{Exemplo 2 \PYGZhy{} Buscando padrão }\PYG{l+s+si}{\PYGZob{}}\PYG{n}{padrao\PYGZus{}search}\PYG{l+s+si}{\PYGZcb{}}\PYG{l+s+s2}{ em qualquer lugar do texto }\PYG{l+s+s2}{\PYGZsq{}}\PYG{l+s+si}{\PYGZob{}}\PYG{n}{texto\PYGZus{}search}\PYG{l+s+si}{\PYGZcb{}}\PYG{l+s+s2}{\PYGZsq{}}\PYG{l+s+s2}{:}\PYG{l+s+s2}{\PYGZdq{}}\PYG{p}{)}
\PYG{n+nb}{print}\PYG{p}{(}\PYG{l+s+sa}{f}\PYG{l+s+s2}{\PYGZdq{}}\PYG{l+s+s2}{Buscando a string }\PYG{l+s+s2}{\PYGZsq{}}\PYG{l+s+si}{\PYGZob{}}\PYG{n}{string\PYGZus{}busca}\PYG{l+s+si}{\PYGZcb{}}\PYG{l+s+s2}{\PYGZsq{}}\PYG{l+s+s2}{:}\PYG{l+s+s2}{\PYGZdq{}}\PYG{p}{)}
\PYG{n+nb}{print}\PYG{p}{(}\PYG{l+s+s2}{\PYGZdq{}}\PYG{l+s+s2}{Resultado:}\PYG{l+s+s2}{\PYGZdq{}}\PYG{p}{,} \PYG{n}{resultado\PYGZus{}search}\PYG{o}{.}\PYG{n}{group}\PYG{p}{(}\PYG{p}{)} \PYG{k}{if} \PYG{n}{resultado\PYGZus{}search} \PYG{k}{else} \PYG{l+s+s2}{\PYGZdq{}}\PYG{l+s+s2}{Sem correspondência}\PYG{l+s+s2}{\PYGZdq{}}\PYG{p}{)}
\end{sphinxVerbatim}

\begin{sphinxVerbatim}[commandchars=\\\{\}]
\PYG{n}{Exemplo} \PYG{l+m+mi}{1} \PYG{o}{\PYGZhy{}} \PYG{n}{Buscando} \PYG{n}{padrão} \PYGZbs{}\PYG{n}{d}\PYG{o}{+} \PYG{n}{no} \PYG{n}{início} \PYG{n}{do} \PYG{n}{texto} \PYG{l+s+s1}{\PYGZsq{}}\PYG{l+s+s1}{123 Python}\PYG{l+s+s1}{\PYGZsq{}}\PYG{p}{:}
\PYG{n}{Resultado}\PYG{p}{:} \PYG{l+m+mi}{123}

\PYG{n}{Exemplo} \PYG{l+m+mi}{2} \PYG{o}{\PYGZhy{}} \PYG{n}{Buscando} \PYG{n}{padrão} \PYGZbs{}\PYG{n}{w}\PYG{o}{+} \PYG{n}{em} \PYG{n}{qualquer} \PYG{n}{lugar} \PYG{n}{do} \PYG{n}{texto} \PYG{l+s+s1}{\PYGZsq{}}\PYG{l+s+s1}{Python é uma linguagem poderosa}\PYG{l+s+s1}{\PYGZsq{}}\PYG{p}{:}
\PYG{n}{Buscando} \PYG{n}{a} \PYG{n}{string} \PYG{l+s+s1}{\PYGZsq{}}\PYG{l+s+s1}{linguagem}\PYG{l+s+s1}{\PYGZsq{}}\PYG{p}{:}
\PYG{n}{Resultado}\PYG{p}{:} \PYG{n}{linguagem}
\end{sphinxVerbatim}

\sphinxAtStartPar
\sphinxstylestrong{Explicação:}
\begin{itemize}
\item {} 
\sphinxAtStartPar
\sphinxstylestrong{Exemplo 1 \sphinxhyphen{} \sphinxcode{\sphinxupquote{re.match}} com padrão \sphinxcode{\sphinxupquote{\textbackslash{}d+}}:}
\begin{itemize}
\item {} 
\sphinxAtStartPar
O padrão \sphinxcode{\sphinxupquote{\textbackslash{}d+}} busca por um ou mais dígitos no início da string.

\item {} 
\sphinxAtStartPar
A string de texto é “123 Python”.

\item {} 
\sphinxAtStartPar
\sphinxcode{\sphinxupquote{re.match}} verifica se o padrão ocorre no início da string. Neste caso, o padrão é encontrado no início com os dígitos “123”.

\item {} 
\sphinxAtStartPar
A saída será “Resultado: 123”, indicando que houve correspondência no início da string.

\end{itemize}

\item {} 
\sphinxAtStartPar
\sphinxstylestrong{Exemplo 2 \sphinxhyphen{} \sphinxcode{\sphinxupquote{re.search}} com padrão \sphinxcode{\sphinxupquote{\textbackslash{}w+}}:}
\begin{itemize}
\item {} 
\sphinxAtStartPar
O padrão \sphinxcode{\sphinxupquote{\textbackslash{}w+}} busca por uma ou mais letras, dígitos ou underscores em qualquer lugar da string.

\item {} 
\sphinxAtStartPar
A string de texto é “Python é uma linguagem poderosa”.

\item {} 
\sphinxAtStartPar
\sphinxcode{\sphinxupquote{re.search}} procura por uma correspondência em qualquer lugar da string. Neste caso, o padrão é encontrado no início com a palavra “Python”.

\item {} 
\sphinxAtStartPar
A saída será “Resultado: linguagem”, indicando que houve correspondência em qualquer lugar da string com a palavra “linguagem”.

\end{itemize}

\end{itemize}

\sphinxAtStartPar
Esses exemplos ilustram o uso das funções \sphinxcode{\sphinxupquote{re.match}} e \sphinxcode{\sphinxupquote{re.search}} para encontrar padrões específicos em strings.


\subsubsection{\sphinxstyleliteralintitle{\sphinxupquote{re.findall(pattern, string)}} e \sphinxstyleliteralintitle{\sphinxupquote{re.finditer(pattern, string):}}}
\label{\detokenize{chapters/ch6/ch6:re-findall-pattern-string-e-re-finditer-pattern-string}}\begin{itemize}
\item {} 
\sphinxAtStartPar
\sphinxcode{\sphinxupquote{re.findall}}: Encontra todas as correspondências e retorna como uma lista.

\item {} 
\sphinxAtStartPar
\sphinxcode{\sphinxupquote{re.finditer}}: Retorna um iterador que produz objetos de correspondência.

\end{itemize}

\sphinxAtStartPar
\sphinxstylestrong{Exemplo:}

\begin{sphinxVerbatim}[commandchars=\\\{\}]
\PYG{k+kn}{import} \PYG{n+nn}{re}

\PYG{n}{padrao} \PYG{o}{=} \PYG{l+s+sa}{r}\PYG{l+s+s1}{\PYGZsq{}}\PYG{l+s+s1}{\PYGZbs{}}\PYG{l+s+s1}{w+}\PYG{l+s+s1}{\PYGZsq{}}
\PYG{n}{texto} \PYG{o}{=} \PYG{l+s+s2}{\PYGZdq{}}\PYG{l+s+s2}{Python é uma linguagem poderosa}\PYG{l+s+s2}{\PYGZdq{}}

\PYG{n}{todas\PYGZus{}correspondencias} \PYG{o}{=} \PYG{n}{re}\PYG{o}{.}\PYG{n}{findall}\PYG{p}{(}\PYG{n}{padrao}\PYG{p}{,} \PYG{n}{texto}\PYG{p}{)}
\PYG{n+nb}{print}\PYG{p}{(}\PYG{l+s+s2}{\PYGZdq{}}\PYG{l+s+s2}{Todas as correspondências:}\PYG{l+s+s2}{\PYGZdq{}}\PYG{p}{,} \PYG{n}{todas\PYGZus{}correspondencias}\PYG{p}{)}

\PYG{n}{iterador\PYGZus{}correspondencias} \PYG{o}{=} \PYG{n}{re}\PYG{o}{.}\PYG{n}{finditer}\PYG{p}{(}\PYG{n}{padrao}\PYG{p}{,} \PYG{n}{texto}\PYG{p}{)}
\PYG{n+nb}{print}\PYG{p}{(}\PYG{l+s+s2}{\PYGZdq{}}\PYG{l+s+s2}{Iterador de correspondências:}\PYG{l+s+s2}{\PYGZdq{}}\PYG{p}{,} \PYG{p}{[}\PYG{n}{match}\PYG{o}{.}\PYG{n}{group}\PYG{p}{(}\PYG{p}{)} \PYG{k}{for} \PYG{n}{match} \PYG{o+ow}{in} \PYG{n}{iterador\PYGZus{}correspondencias}\PYG{p}{]}\PYG{p}{)}
\end{sphinxVerbatim}

\begin{sphinxVerbatim}[commandchars=\\\{\}]
\PYG{n}{Todas} \PYG{k}{as} \PYG{n}{correspondências}\PYG{p}{:} \PYG{p}{[}\PYG{l+s+s1}{\PYGZsq{}}\PYG{l+s+s1}{Python}\PYG{l+s+s1}{\PYGZsq{}}\PYG{p}{,} \PYG{l+s+s1}{\PYGZsq{}}\PYG{l+s+s1}{é}\PYG{l+s+s1}{\PYGZsq{}}\PYG{p}{,} \PYG{l+s+s1}{\PYGZsq{}}\PYG{l+s+s1}{uma}\PYG{l+s+s1}{\PYGZsq{}}\PYG{p}{,} \PYG{l+s+s1}{\PYGZsq{}}\PYG{l+s+s1}{linguagem}\PYG{l+s+s1}{\PYGZsq{}}\PYG{p}{,} \PYG{l+s+s1}{\PYGZsq{}}\PYG{l+s+s1}{poderosa}\PYG{l+s+s1}{\PYGZsq{}}\PYG{p}{]}
\PYG{n}{Iterador} \PYG{n}{de} \PYG{n}{correspondências}\PYG{p}{:} \PYG{p}{[}\PYG{l+s+s1}{\PYGZsq{}}\PYG{l+s+s1}{Python}\PYG{l+s+s1}{\PYGZsq{}}\PYG{p}{,} \PYG{l+s+s1}{\PYGZsq{}}\PYG{l+s+s1}{é}\PYG{l+s+s1}{\PYGZsq{}}\PYG{p}{,} \PYG{l+s+s1}{\PYGZsq{}}\PYG{l+s+s1}{uma}\PYG{l+s+s1}{\PYGZsq{}}\PYG{p}{,} \PYG{l+s+s1}{\PYGZsq{}}\PYG{l+s+s1}{linguagem}\PYG{l+s+s1}{\PYGZsq{}}\PYG{p}{,} \PYG{l+s+s1}{\PYGZsq{}}\PYG{l+s+s1}{poderosa}\PYG{l+s+s1}{\PYGZsq{}}\PYG{p}{]}
\end{sphinxVerbatim}


\subsubsection{\sphinxstyleliteralintitle{\sphinxupquote{re.sub(pattern, replacement, string):}}}
\label{\detokenize{chapters/ch6/ch6:re-sub-pattern-replacement-string}}
\sphinxAtStartPar
Substitui todas as ocorrências do padrão por uma string de substituição.

\sphinxAtStartPar
\sphinxstylestrong{Exemplo:}

\begin{sphinxVerbatim}[commandchars=\\\{\}]
\PYG{k+kn}{import} \PYG{n+nn}{re}

\PYG{n}{padrao} \PYG{o}{=} \PYG{l+s+sa}{r}\PYG{l+s+s1}{\PYGZsq{}}\PYG{l+s+s1}{\PYGZbs{}}\PYG{l+s+s1}{d+}\PYG{l+s+s1}{\PYGZsq{}}
\PYG{n}{texto} \PYG{o}{=} \PYG{l+s+s2}{\PYGZdq{}}\PYG{l+s+s2}{Python 2 e Python 3}\PYG{l+s+s2}{\PYGZdq{}}
\PYG{n}{substituido} \PYG{o}{=} \PYG{n}{re}\PYG{o}{.}\PYG{n}{sub}\PYG{p}{(}\PYG{n}{padrao}\PYG{p}{,} \PYG{l+s+s2}{\PYGZdq{}}\PYG{l+s+s2}{X}\PYG{l+s+s2}{\PYGZdq{}}\PYG{p}{,} \PYG{n}{texto}\PYG{p}{)}
\PYG{n+nb}{print}\PYG{p}{(}\PYG{l+s+s2}{\PYGZdq{}}\PYG{l+s+s2}{Texto original:}\PYG{l+s+s2}{\PYGZdq{}}\PYG{p}{,} \PYG{n}{texto}\PYG{p}{)}
\PYG{n+nb}{print}\PYG{p}{(}\PYG{l+s+s2}{\PYGZdq{}}\PYG{l+s+s2}{Texto substituído:}\PYG{l+s+s2}{\PYGZdq{}}\PYG{p}{,} \PYG{n}{substituido}\PYG{p}{)}
\end{sphinxVerbatim}

\begin{sphinxVerbatim}[commandchars=\\\{\}]
\PYG{n}{Texto} \PYG{n}{original}\PYG{p}{:} \PYG{n}{Python} \PYG{l+m+mi}{2} \PYG{n}{e} \PYG{n}{Python} \PYG{l+m+mi}{3}
\PYG{n}{Texto} \PYG{n}{substituído}\PYG{p}{:} \PYG{n}{Python} \PYG{n}{X} \PYG{n}{e} \PYG{n}{Python} \PYG{n}{X}
\end{sphinxVerbatim}

\sphinxAtStartPar
As expressões regulares podem ser aplicadas em diversas tarefas de manipulação de strings, incluindo busca, extração e transformação de padrões específicos. Para uma compreensão mais profunda, consulte a documentação oficial sobre expressões regulares em Python: \sphinxhref{https://docs.python.org/3/library/re.html}{Expressões Regulares (re)}.


\section{Exercícios}
\label{\detokenize{chapters/ch6/ch6:exercicios}}\begin{enumerate}
\sphinxsetlistlabels{\arabic}{enumi}{enumii}{}{.}%
\item {} 
\sphinxAtStartPar
\sphinxstylestrong{Concatenação e Repetição:}
Escreva um programa que solicite ao usuário dois títulos de obras filosóficas de Arthur Schopenhauer e os concatene em uma única string. Em seguida, repita a string resultante três vezes e exiba o resultado.

\sphinxAtStartPar
\sphinxstylestrong{Testes:}

\begin{sphinxVerbatim}[commandchars=\\\{\}]
\PYG{c+c1}{\PYGZsh{} Teste 1}
\PYG{n}{entrada}\PYG{p}{:} \PYG{l+s+s2}{\PYGZdq{}}\PYG{l+s+s2}{O Mundo como Vontade e Representação}\PYG{l+s+s2}{\PYGZdq{}}\PYG{p}{,} \PYG{l+s+s2}{\PYGZdq{}}\PYG{l+s+s2}{A Arte de Escrever}\PYG{l+s+s2}{\PYGZdq{}}
\PYG{n}{saída}\PYG{p}{:} \PYG{l+s+s2}{\PYGZdq{}}\PYG{l+s+s2}{O Mundo como Vontade e RepresentaçãoA Arte de EscreverO Mundo como Vontade e RepresentaçãoA Arte de EscreverO Mundo como Vontade e RepresentaçãoA Arte de Escrever}\PYG{l+s+s2}{\PYGZdq{}}

\PYG{c+c1}{\PYGZsh{} Teste 2}
\PYG{n}{entrada}\PYG{p}{:} \PYG{l+s+s2}{\PYGZdq{}}\PYG{l+s+s2}{Parerga e Paralipomena}\PYG{l+s+s2}{\PYGZdq{}}\PYG{p}{,} \PYG{l+s+s2}{\PYGZdq{}}\PYG{l+s+s2}{Aforismos para a Sabedoria de Vida}\PYG{l+s+s2}{\PYGZdq{}}
\PYG{n}{saída}\PYG{p}{:} \PYG{l+s+s2}{\PYGZdq{}}\PYG{l+s+s2}{Parerga e ParalipomenaAforismos para a Sabedoria de VidaParerga e ParalipomenaAforismos para a Sabedoria de VidaParerga e ParalipomenaAforismos para a Sabedoria de Vida}\PYG{l+s+s2}{\PYGZdq{}}
\end{sphinxVerbatim}

\item {} 
\sphinxAtStartPar
\sphinxstylestrong{Indexação e Fatiamento:}
Crie uma função que receba o título de uma obra filosófica de Immanuel Kant como entrada e retorne a primeira e última letra do título, separadas por um hífen.

\sphinxAtStartPar
\sphinxstylestrong{Testes:}

\begin{sphinxVerbatim}[commandchars=\\\{\}]
\PYG{c+c1}{\PYGZsh{} Teste 1}
\PYG{n}{entrada}\PYG{p}{:} \PYG{l+s+s2}{\PYGZdq{}}\PYG{l+s+s2}{Crítica da Razão Pura}\PYG{l+s+s2}{\PYGZdq{}}
\PYG{n}{saída}\PYG{p}{:} \PYG{l+s+s2}{\PYGZdq{}}\PYG{l+s+s2}{C\PYGZhy{}a}\PYG{l+s+s2}{\PYGZdq{}}

\PYG{c+c1}{\PYGZsh{} Teste 2}
\PYG{n}{entrada}\PYG{p}{:} \PYG{l+s+s2}{\PYGZdq{}}\PYG{l+s+s2}{Fundamentação da Metafísica dos Costumes}\PYG{l+s+s2}{\PYGZdq{}}
\PYG{n}{saída}\PYG{p}{:} \PYG{l+s+s2}{\PYGZdq{}}\PYG{l+s+s2}{F\PYGZhy{}s}\PYG{l+s+s2}{\PYGZdq{}}
\end{sphinxVerbatim}

\item {} 
\sphinxAtStartPar
\sphinxstylestrong{Métodos \sphinxcode{\sphinxupquote{upper()}} e \sphinxcode{\sphinxupquote{lower()}}:}
Peça ao usuário para fornecer um trecho de uma obra de Platão. Converta todas as letras para maiúsculas e, em seguida, para minúsculas. Exiba ambas as versões.

\sphinxAtStartPar
\sphinxstylestrong{Testes:}

\begin{sphinxVerbatim}[commandchars=\\\{\}]
\PYG{c+c1}{\PYGZsh{} Teste 1}
\PYG{n}{entrada}\PYG{p}{:} \PYG{l+s+s2}{\PYGZdq{}}\PYG{l+s+s2}{A República é uma obra fundamental de Platão.}\PYG{l+s+s2}{\PYGZdq{}}
\PYG{n}{saída}\PYG{p}{:} \PYG{l+s+s2}{\PYGZdq{}}\PYG{l+s+s2}{A REPÚBLICA É UMA OBRA FUNDAMENTAL DE PLATÃO.}\PYG{l+s+s2}{\PYGZdq{}}\PYG{p}{,} \PYG{l+s+s2}{\PYGZdq{}}\PYG{l+s+s2}{a república é uma obra fundamental de platão.}\PYG{l+s+s2}{\PYGZdq{}}

\PYG{c+c1}{\PYGZsh{} Teste 2}
\PYG{n}{entrada}\PYG{p}{:} \PYG{l+s+s2}{\PYGZdq{}}\PYG{l+s+s2}{O Mito da Caverna é uma alegoria impactante.}\PYG{l+s+s2}{\PYGZdq{}}
\PYG{n}{saída}\PYG{p}{:} \PYG{l+s+s2}{\PYGZdq{}}\PYG{l+s+s2}{O MITO DA CAVERNA É UMA ALEGORIA IMPACTANTE.}\PYG{l+s+s2}{\PYGZdq{}}\PYG{p}{,} \PYG{l+s+s2}{\PYGZdq{}}\PYG{l+s+s2}{o mito da caverna é uma alegoria impactante.}\PYG{l+s+s2}{\PYGZdq{}}
\end{sphinxVerbatim}

\item {} 
\sphinxAtStartPar
\sphinxstylestrong{Método \sphinxcode{\sphinxupquote{replace()}}:}
Crie um programa que substitua todas as ocorrências da palavra “ignorância” por “conhecimento” em uma passagem de um diálogo de Sócrates fornecido pelo usuário.

\sphinxAtStartPar
\sphinxstylestrong{Testes:}

\begin{sphinxVerbatim}[commandchars=\\\{\}]
\PYG{c+c1}{\PYGZsh{} Teste 1}
\PYG{n}{entrada}\PYG{p}{:} \PYG{l+s+s2}{\PYGZdq{}}\PYG{l+s+s2}{A verdadeira sabedoria está em reconhecer a própria ignorância.}\PYG{l+s+s2}{\PYGZdq{}}
\PYG{n}{saída}\PYG{p}{:} \PYG{l+s+s2}{\PYGZdq{}}\PYG{l+s+s2}{A verdadeira sabedoria está em reconhecer a própria conhecimento.}\PYG{l+s+s2}{\PYGZdq{}}

\PYG{c+c1}{\PYGZsh{} Teste 2}
\PYG{n}{entrada}\PYG{p}{:} \PYG{l+s+s2}{\PYGZdq{}}\PYG{l+s+s2}{Só sei que nada sei.}\PYG{l+s+s2}{\PYGZdq{}}
\PYG{n}{saída}\PYG{p}{:} \PYG{l+s+s2}{\PYGZdq{}}\PYG{l+s+s2}{Só sei que nada conhecimento.}\PYG{l+s+s2}{\PYGZdq{}}
\end{sphinxVerbatim}

\item {} 
\sphinxAtStartPar
\sphinxstylestrong{Método \sphinxcode{\sphinxupquote{split()}}:}
Escreva um programa que solicite ao usuário uma citação de uma obra de Aristóteles e divida a citação em palavras. Em seguida, exiba a contagem de palavras e as próprias palavras em uma lista.

\sphinxAtStartPar
\sphinxstylestrong{Testes:}

\begin{sphinxVerbatim}[commandchars=\\\{\}]
\PYG{c+c1}{\PYGZsh{} Teste 1}
\PYG{n}{entrada}\PYG{p}{:} \PYG{l+s+s2}{\PYGZdq{}}\PYG{l+s+s2}{A ética de Aristóteles destaca a busca pela virtude.}\PYG{l+s+s2}{\PYGZdq{}}
\PYG{n}{saída}\PYG{p}{:} \PYG{l+s+s2}{\PYGZdq{}}\PYG{l+s+s2}{Número de palavras: 8}\PYG{l+s+s2}{\PYGZdq{}}\PYG{p}{,} \PYG{l+s+s2}{\PYGZdq{}}\PYG{l+s+s2}{[}\PYG{l+s+s2}{\PYGZsq{}}\PYG{l+s+s2}{A}\PYG{l+s+s2}{\PYGZsq{}}\PYG{l+s+s2}{, }\PYG{l+s+s2}{\PYGZsq{}}\PYG{l+s+s2}{ética}\PYG{l+s+s2}{\PYGZsq{}}\PYG{l+s+s2}{, }\PYG{l+s+s2}{\PYGZsq{}}\PYG{l+s+s2}{de}\PYG{l+s+s2}{\PYGZsq{}}\PYG{l+s+s2}{, }\PYG{l+s+s2}{\PYGZsq{}}\PYG{l+s+s2}{Aristóteles}\PYG{l+s+s2}{\PYGZsq{}}\PYG{l+s+s2}{, }\PYG{l+s+s2}{\PYGZsq{}}\PYG{l+s+s2}{destaca}\PYG{l+s+s2}{\PYGZsq{}}\PYG{l+s+s2}{, }\PYG{l+s+s2}{\PYGZsq{}}\PYG{l+s+s2}{a}\PYG{l+s+s2}{\PYGZsq{}}\PYG{l+s+s2}{, }\PYG{l+s+s2}{\PYGZsq{}}\PYG{l+s+s2}{busca}\PYG{l+s+s2}{\PYGZsq{}}\PYG{l+s+s2}{, }\PYG{l+s+s2}{\PYGZsq{}}\PYG{l+s+s2}{pela}\PYG{l+s+s2}{\PYGZsq{}}\PYG{l+s+s2}{, }\PYG{l+s+s2}{\PYGZsq{}}\PYG{l+s+s2}{virtude.}\PYG{l+s+s2}{\PYGZsq{}}\PYG{l+s+s2}{]}\PYG{l+s+s2}{\PYGZdq{}}

\PYG{c+c1}{\PYGZsh{} Teste 2}
\PYG{n}{entrada}\PYG{p}{:} \PYG{l+s+s2}{\PYGZdq{}}\PYG{l+s+s2}{A poética de Aristóteles influenciou a teoria literária.}\PYG{l+s+s2}{\PYGZdq{}}
\PYG{n}{saída}\PYG{p}{:} \PYG{l+s+s2}{\PYGZdq{}}\PYG{l+s+s2}{Número de palavras: 8}\PYG{l+s+s2}{\PYGZdq{}}\PYG{p}{,} \PYG{l+s+s2}{\PYGZdq{}}\PYG{l+s+s2}{[}\PYG{l+s+s2}{\PYGZsq{}}\PYG{l+s+s2}{A}\PYG{l+s+s2}{\PYGZsq{}}\PYG{l+s+s2}{, }\PYG{l+s+s2}{\PYGZsq{}}\PYG{l+s+s2}{poética}\PYG{l+s+s2}{\PYGZsq{}}\PYG{l+s+s2}{, }\PYG{l+s+s2}{\PYGZsq{}}\PYG{l+s+s2}{de}\PYG{l+s+s2}{\PYGZsq{}}\PYG{l+s+s2}{, }\PYG{l+s+s2}{\PYGZsq{}}\PYG{l+s+s2}{Aristóteles}\PYG{l+s+s2}{\PYGZsq{}}\PYG{l+s+s2}{, }\PYG{l+s+s2}{\PYGZsq{}}\PYG{l+s+s2}{influenciou}\PYG{l+s+s2}{\PYGZsq{}}\PYG{l+s+s2}{, }\PYG{l+s+s2}{\PYGZsq{}}\PYG{l+s+s2}{a}\PYG{l+s+s2}{\PYGZsq{}}\PYG{l+s+s2}{, }\PYG{l+s+s2}{\PYGZsq{}}\PYG{l+s+s2}{teoria}\PYG{l+s+s2}{\PYGZsq{}}\PYG{l+s+s2}{, }\PYG{l+s+s2}{\PYGZsq{}}\PYG{l+s+s2}{literária.}\PYG{l+s+s2}{\PYGZsq{}}\PYG{l+s+s2}{]}\PYG{l+s+s2}{\PYGZdq{}}
\end{sphinxVerbatim}

\item {} 
\sphinxAtStartPar
\sphinxstylestrong{Formatação de Strings:}
Crie uma função que receba o título de uma obra, o ano de publicação e o tema de uma obra de René Descartes como parâmetros e retorne uma mensagem formatada usando f\sphinxhyphen{}strings.

\sphinxAtStartPar
\sphinxstylestrong{Testes:}

\begin{sphinxVerbatim}[commandchars=\\\{\}]
\PYG{c+c1}{\PYGZsh{} Teste 1}
\PYG{n}{entrada}\PYG{p}{:} \PYG{l+s+s2}{\PYGZdq{}}\PYG{l+s+s2}{Meditações Metafísicas}\PYG{l+s+s2}{\PYGZdq{}}\PYG{p}{,} \PYG{l+m+mi}{1641}\PYG{p}{,} \PYG{l+s+s2}{\PYGZdq{}}\PYG{l+s+s2}{Dúvida Metódica}\PYG{l+s+s2}{\PYGZdq{}}
\PYG{n}{saída}\PYG{p}{:} \PYG{l+s+s2}{\PYGZdq{}}\PYG{l+s+s2}{A obra }\PYG{l+s+s2}{\PYGZsq{}}\PYG{l+s+s2}{Meditações Metafísicas}\PYG{l+s+s2}{\PYGZsq{}}\PYG{l+s+s2}{, publicada em 1641, aborda o tema da Dúvida Metódica.}\PYG{l+s+s2}{\PYGZdq{}}

\PYG{c+c1}{\PYGZsh{} Teste 2}
\PYG{n}{entrada}\PYG{p}{:} \PYG{l+s+s2}{\PYGZdq{}}\PYG{l+s+s2}{Discurso do Método}\PYG{l+s+s2}{\PYGZdq{}}\PYG{p}{,} \PYG{l+m+mi}{1637}\PYG{p}{,} \PYG{l+s+s2}{\PYGZdq{}}\PYG{l+s+s2}{Racionalismo}\PYG{l+s+s2}{\PYGZdq{}}
\PYG{n}{saída}\PYG{p}{:} \PYG{l+s+s2}{\PYGZdq{}}\PYG{l+s+s2}{A obra }\PYG{l+s+s2}{\PYGZsq{}}\PYG{l+s+s2}{Discurso do Método}\PYG{l+s+s2}{\PYGZsq{}}\PYG{l+s+s2}{, publicada em 1637, explora o tema do Racionalismo.}\PYG{l+s+s2}{\PYGZdq{}}
\end{sphinxVerbatim}

\item {} 
\sphinxAtStartPar
\sphinxstylestrong{Métodos \sphinxcode{\sphinxupquote{startswith()}} e \sphinxcode{\sphinxupquote{endswith()}}:}
Escreva uma função que receba uma lista de títulos de obras de Martin Heidegger e uma letra como parâmetros. A função deve retornar uma lista com os títulos que começam com a letra fornecida.

\sphinxAtStartPar
\sphinxstylestrong{Testes:}

\begin{sphinxVerbatim}[commandchars=\\\{\}]
\PYG{c+c1}{\PYGZsh{} Teste 1}
\PYG{n}{entrada}\PYG{p}{:} \PYG{p}{[}\PYG{l+s+s2}{\PYGZdq{}}\PYG{l+s+s2}{Ser e Tempo}\PYG{l+s+s2}{\PYGZdq{}}\PYG{p}{,} \PYG{l+s+s2}{\PYGZdq{}}\PYG{l+s+s2}{Contribuições à Filosofia (Do Evento)}\PYG{l+s+s2}{\PYGZdq{}}\PYG{p}{]}\PYG{p}{,} \PYG{l+s+s2}{\PYGZdq{}}\PYG{l+s+s2}{S}\PYG{l+s+s2}{\PYGZdq{}}
\PYG{n}{saída}\PYG{p}{:} \PYG{l+s+s2}{\PYGZdq{}}\PYG{l+s+s2}{[}\PYG{l+s+s2}{\PYGZsq{}}\PYG{l+s+s2}{Ser e Tempo}\PYG{l+s+s2}{\PYGZsq{}}\PYG{l+s+s2}{]}\PYG{l+s+s2}{\PYGZdq{}}

\PYG{c+c1}{\PYGZsh{} Teste 2}
\PYG{n}{entrada}\PYG{p}{:} \PYG{p}{[}\PYG{l+s+s2}{\PYGZdq{}}\PYG{l+s+s2}{O Conceito de Tempo em Aristóteles}\PYG{l+s+s2}{\PYGZdq{}}\PYG{p}{,} \PYG{l+s+s2}{\PYGZdq{}}\PYG{l+s+s2}{Heidegger e a Questão da Ética}\PYG{l+s+s2}{\PYGZdq{}}\PYG{p}{]}\PYG{p}{,} \PYG{l+s+s2}{\PYGZdq{}}\PYG{l+s+s2}{H}\PYG{l+s+s2}{\PYGZdq{}}
\PYG{n}{saída}\PYG{p}{:} \PYG{l+s+s2}{\PYGZdq{}}\PYG{l+s+s2}{[}\PYG{l+s+s2}{\PYGZsq{}}\PYG{l+s+s2}{Heidegger e a Questão da Ética}\PYG{l+s+s2}{\PYGZsq{}}\PYG{l+s+s2}{]}\PYG{l+s+s2}{\PYGZdq{}}
\end{sphinxVerbatim}

\item {} 
\sphinxAtStartPar
\sphinxstylestrong{Expressões Regulares \sphinxhyphen{} \sphinxcode{\sphinxupquote{re.match}}:}
Crie uma função que utilize \sphinxcode{\sphinxupquote{re.match}} para verificar se um título de obra de Jean\sphinxhyphen{}Jacques Rousseau atende aos seguintes critérios: deve começar com uma letra maiúscula, conter pelo menos um número e ter no mínimo 10 caracteres.

\sphinxAtStartPar
\sphinxstylestrong{Testes:}

\begin{sphinxVerbatim}[commandchars=\\\{\}]
\PYG{c+c1}{\PYGZsh{} Teste 1}
\PYG{n}{entrada}\PYG{p}{:} \PYG{l+s+s2}{\PYGZdq{}}\PYG{l+s+s2}{Do Contrato Social}\PYG{l+s+s2}{\PYGZdq{}}
\PYG{n}{saída}\PYG{p}{:} \PYG{k+kc}{True}

\PYG{c+c1}{\PYGZsh{} Teste 2}
\PYG{n}{entrada}\PYG{p}{:} \PYG{l+s+s2}{\PYGZdq{}}\PYG{l+s+s2}{o estado de natureza}\PYG{l+s+s2}{\PYGZdq{}}
\PYG{n}{saída}\PYG{p}{:} \PYG{k+kc}{False}
\end{sphinxVerbatim}

\item {} 
\sphinxAtStartPar
\sphinxstylestrong{Expressões Regulares \sphinxhyphen{} \sphinxcode{\sphinxupquote{re.search}}:}
Implemente uma função que utilize \sphinxcode{\sphinxupquote{re.search}} para encontrar todos os anos em que Baruch Spinoza escreveu obras e retorne uma lista com esses anos.

\sphinxAtStartPar
\sphinxstylestrong{Testes:}

\begin{sphinxVerbatim}[commandchars=\\\{\}]
\PYG{c+c1}{\PYGZsh{} Teste 1}
\PYG{n}{entrada}\PYG{p}{:} \PYG{l+s+s2}{\PYGZdq{}}\PYG{l+s+s2}{Baruch Spinoza publicou }\PYG{l+s+s2}{\PYGZsq{}}\PYG{l+s+s2}{Ética}\PYG{l+s+s2}{\PYGZsq{}}\PYG{l+s+s2}{ em 1677 e }\PYG{l+s+s2}{\PYGZsq{}}\PYG{l+s+s2}{Tratado Teológico\PYGZhy{}Político}\PYG{l+s+s2}{\PYGZsq{}}\PYG{l+s+s2}{ em 1670.}\PYG{l+s+s2}{\PYGZdq{}}
\PYG{n}{saída}\PYG{p}{:} \PYG{l+s+s2}{\PYGZdq{}}\PYG{l+s+s2}{[}\PYG{l+s+s2}{\PYGZsq{}}\PYG{l+s+s2}{1677}\PYG{l+s+s2}{\PYGZsq{}}\PYG{l+s+s2}{, }\PYG{l+s+s2}{\PYGZsq{}}\PYG{l+s+s2}{1670}\PYG{l+s+s2}{\PYGZsq{}}\PYG{l+s+s2}{]}\PYG{l+s+s2}{\PYGZdq{}}

\PYG{c+c1}{\PYGZsh{} Teste 2}
\PYG{n}{entrada}\PYG{p}{:} \PYG{l+s+s2}{\PYGZdq{}}\PYG{l+s+s2}{As ideias de Baruch Spinoza influenciaram a filosofia moderna.}\PYG{l+s+s2}{\PYGZdq{}}
\PYG{n}{saída}\PYG{p}{:} \PYG{l+s+s2}{\PYGZdq{}}\PYG{l+s+s2}{[]}\PYG{l+s+s2}{\PYGZdq{}}
\end{sphinxVerbatim}

\item {} 
\sphinxAtStartPar
\sphinxstylestrong{Expressões Regulares \sphinxhyphen{} \sphinxcode{\sphinxupquote{re.sub}}:}
Escreva uma função que use \sphinxcode{\sphinxupquote{re.sub}} para substituir todas as ocorrências de palavras que terminam com “idade” por “eterno retorno” em uma passagem de uma obra de Nietzsche.

\end{enumerate}

\sphinxAtStartPar
\sphinxstylestrong{Testes:}

\begin{sphinxVerbatim}[commandchars=\\\{\}]
\PYG{c+c1}{\PYGZsh{} Teste 1}
\PYG{n}{entrada}\PYG{p}{:} \PYG{l+s+s2}{\PYGZdq{}}\PYG{l+s+s2}{A liberdade é a base do Eterno Retorno.}\PYG{l+s+s2}{\PYGZdq{}}
\PYG{n}{saída}\PYG{p}{:} \PYG{l+s+s2}{\PYGZdq{}}\PYG{l+s+s2}{A liberdade é a base do eterno retorno.}\PYG{l+s+s2}{\PYGZdq{}}

\PYG{c+c1}{\PYGZsh{} Teste 2}
\PYG{n}{entrada}\PYG{p}{:} \PYG{l+s+s2}{\PYGZdq{}}\PYG{l+s+s2}{A igualdade e fraternidade são valores do Eterno Retorno.}\PYG{l+s+s2}{\PYGZdq{}}
\PYG{n}{saída}\PYG{p}{:} \PYG{l+s+s2}{\PYGZdq{}}\PYG{l+s+s2}{A igualdade e fraternidade são valores do Eterno Retorno.}\PYG{l+s+s2}{\PYGZdq{}}
\end{sphinxVerbatim}

\sphinxstepscope


\chapter{Capítulo 7: NumPy e Matplotlib}
\label{\detokenize{chapters/ch7/ch7:capitulo-7-numpy-e-matplotlib}}\label{\detokenize{chapters/ch7/ch7::doc}}

\section{Manipulação de Arrays com o NumPy}
\label{\detokenize{chapters/ch7/ch7:manipulacao-de-arrays-com-o-numpy}}
\sphinxAtStartPar
\sphinxhref{https://numpy.org/}{NumPy} é uma biblioteca fundamental para computação numérica em Python. Ela fornece estruturas de dados eficientes para a manipulação de arrays multidimensionais, bem como funções matemáticas para operações rápidas em elementos desses arrays.

\sphinxAtStartPar
A instrução \sphinxcode{\sphinxupquote{import numpy as np}} é utilizada para importar a biblioteca NumPy e atribuir a ela um nome de apelido (alias), que neste caso é “np”. Essa prática é comum e recomendada quando se trabalha com bibliotecas extensas como o NumPy, pois facilita a referência a funções e objetos da biblioteca, tornando o código mais conciso e legível.


\subsection{Definindo Arrays}
\label{\detokenize{chapters/ch7/ch7:definindo-arrays}}
\begin{sphinxVerbatim}[commandchars=\\\{\}]
\PYG{c+c1}{\PYGZsh{} Importando a biblioteca NumPy}
\PYG{k+kn}{import} \PYG{n+nn}{numpy} \PYG{k}{as} \PYG{n+nn}{np}

\PYG{c+c1}{\PYGZsh{} Criando uma array unidimensional}
\PYG{n}{array\PYGZus{}1d} \PYG{o}{=} \PYG{n}{np}\PYG{o}{.}\PYG{n}{array}\PYG{p}{(}\PYG{p}{[}\PYG{l+m+mi}{1}\PYG{p}{,} \PYG{l+m+mi}{2}\PYG{p}{,} \PYG{l+m+mi}{3}\PYG{p}{,} \PYG{l+m+mi}{4}\PYG{p}{,} \PYG{l+m+mi}{5}\PYG{p}{]}\PYG{p}{)}

\PYG{c+c1}{\PYGZsh{} Criando uma array bidimensional}
\PYG{n}{array\PYGZus{}2d} \PYG{o}{=} \PYG{n}{np}\PYG{o}{.}\PYG{n}{array}\PYG{p}{(}\PYG{p}{[}\PYG{p}{[}\PYG{l+m+mi}{1}\PYG{p}{,} \PYG{l+m+mi}{2}\PYG{p}{,} \PYG{l+m+mi}{3}\PYG{p}{]}\PYG{p}{,} \PYG{p}{[}\PYG{l+m+mi}{4}\PYG{p}{,} \PYG{l+m+mi}{5}\PYG{p}{,} \PYG{l+m+mi}{6}\PYG{p}{]}\PYG{p}{]}\PYG{p}{)}

\PYG{c+c1}{\PYGZsh{} Arrays especiais}
\PYG{n}{zeros\PYGZus{}array} \PYG{o}{=} \PYG{n}{np}\PYG{o}{.}\PYG{n}{zeros}\PYG{p}{(}\PYG{p}{(}\PYG{l+m+mi}{3}\PYG{p}{,} \PYG{l+m+mi}{3}\PYG{p}{)}\PYG{p}{)}  \PYG{c+c1}{\PYGZsh{} Array 3x3 preenchido com zeros}
\PYG{n}{ones\PYGZus{}array} \PYG{o}{=} \PYG{n}{np}\PYG{o}{.}\PYG{n}{ones}\PYG{p}{(}\PYG{p}{(}\PYG{l+m+mi}{2}\PYG{p}{,} \PYG{l+m+mi}{4}\PYG{p}{)}\PYG{p}{)}   \PYG{c+c1}{\PYGZsh{} Array 2x4 preenchido com uns}
\PYG{n}{random\PYGZus{}array} \PYG{o}{=} \PYG{n}{np}\PYG{o}{.}\PYG{n}{random}\PYG{o}{.}\PYG{n}{rand}\PYG{p}{(}\PYG{l+m+mi}{3}\PYG{p}{,} \PYG{l+m+mi}{2}\PYG{p}{)}  \PYG{c+c1}{\PYGZsh{} Array 3x2 com valores aleatórios entre 0 e 1}

\PYG{c+c1}{\PYGZsh{} Imprimindo os arrays}
\PYG{n+nb}{print}\PYG{p}{(}\PYG{l+s+s2}{\PYGZdq{}}\PYG{l+s+s2}{Array Unidimensional:}\PYG{l+s+s2}{\PYGZdq{}}\PYG{p}{,}\PYG{n}{array\PYGZus{}1d}\PYG{p}{)}
\PYG{n+nb}{print}\PYG{p}{(}\PYG{l+s+s2}{\PYGZdq{}}\PYG{l+s+s2}{Array Bidimensional:}\PYG{l+s+s2}{\PYGZdq{}}\PYG{p}{,}\PYG{n}{array\PYGZus{}2d}\PYG{p}{)}
\PYG{n+nb}{print}\PYG{p}{(}\PYG{l+s+s2}{\PYGZdq{}}\PYG{l+s+s2}{Array Zeros:}\PYG{l+s+s2}{\PYGZdq{}}\PYG{p}{,}\PYG{n}{zeros\PYGZus{}array}\PYG{p}{)}
\PYG{n+nb}{print}\PYG{p}{(}\PYG{l+s+s2}{\PYGZdq{}}\PYG{l+s+s2}{Array Ones:}\PYG{l+s+s2}{\PYGZdq{}}\PYG{p}{,}\PYG{n}{ones\PYGZus{}array}\PYG{p}{)}
\PYG{n+nb}{print}\PYG{p}{(}\PYG{l+s+s2}{\PYGZdq{}}\PYG{l+s+s2}{Array Random:}\PYG{l+s+s2}{\PYGZdq{}}\PYG{p}{,}\PYG{n}{random\PYGZus{}array}\PYG{p}{)}
\end{sphinxVerbatim}

\begin{sphinxVerbatim}[commandchars=\\\{\}]
\PYG{n}{Array} \PYG{n}{Unidimensional}\PYG{p}{:}
 \PYG{p}{[}\PYG{l+m+mi}{1} \PYG{l+m+mi}{2} \PYG{l+m+mi}{3} \PYG{l+m+mi}{4} \PYG{l+m+mi}{5}\PYG{p}{]}
\PYG{n}{Array} \PYG{n}{Bidimensional}\PYG{p}{:}
 \PYG{p}{[}\PYG{p}{[}\PYG{l+m+mi}{1} \PYG{l+m+mi}{2} \PYG{l+m+mi}{3}\PYG{p}{]}
 \PYG{p}{[}\PYG{l+m+mi}{4} \PYG{l+m+mi}{5} \PYG{l+m+mi}{6}\PYG{p}{]}\PYG{p}{]}
\PYG{n}{Array} \PYG{n}{Zeros}\PYG{p}{:}
 \PYG{p}{[}\PYG{p}{[}\PYG{l+m+mf}{0.} \PYG{l+m+mf}{0.} \PYG{l+m+mf}{0.}\PYG{p}{]}
 \PYG{p}{[}\PYG{l+m+mf}{0.} \PYG{l+m+mf}{0.} \PYG{l+m+mf}{0.}\PYG{p}{]}
 \PYG{p}{[}\PYG{l+m+mf}{0.} \PYG{l+m+mf}{0.} \PYG{l+m+mf}{0.}\PYG{p}{]}\PYG{p}{]}
\PYG{n}{Array} \PYG{n}{Ones}\PYG{p}{:}
 \PYG{p}{[}\PYG{p}{[}\PYG{l+m+mf}{1.} \PYG{l+m+mf}{1.} \PYG{l+m+mf}{1.} \PYG{l+m+mf}{1.}\PYG{p}{]}
 \PYG{p}{[}\PYG{l+m+mf}{1.} \PYG{l+m+mf}{1.} \PYG{l+m+mf}{1.} \PYG{l+m+mf}{1.}\PYG{p}{]}\PYG{p}{]}
\PYG{n}{Array} \PYG{n}{Random}\PYG{p}{:}
 \PYG{p}{[}\PYG{p}{[}\PYG{l+m+mf}{0.97601954} \PYG{l+m+mf}{0.98766048}\PYG{p}{]}
 \PYG{p}{[}\PYG{l+m+mf}{0.28732808} \PYG{l+m+mf}{0.28372316}\PYG{p}{]}
 \PYG{p}{[}\PYG{l+m+mf}{0.58467457} \PYG{l+m+mf}{0.4823928} \PYG{p}{]}\PYG{p}{]}
\end{sphinxVerbatim}


\subsection{Operações com Arrays}
\label{\detokenize{chapters/ch7/ch7:operacoes-com-arrays}}
\begin{sphinxVerbatim}[commandchars=\\\{\}]
\PYG{c+c1}{\PYGZsh{} Soma de arrays}
\PYG{n}{soma\PYGZus{}arrays} \PYG{o}{=} \PYG{n}{array\PYGZus{}1d} \PYG{o}{+} \PYG{n}{array\PYGZus{}1d}

\PYG{c+c1}{\PYGZsh{} Multiplicação de arrays}
\PYG{n}{mult\PYGZus{}arrays} \PYG{o}{=} \PYG{n}{array\PYGZus{}1d} \PYG{o}{*} \PYG{l+m+mi}{3}

\PYG{c+c1}{\PYGZsh{} Produto escalar}
\PYG{n}{dot\PYGZus{}product} \PYG{o}{=} \PYG{n}{np}\PYG{o}{.}\PYG{n}{dot}\PYG{p}{(}\PYG{n}{array\PYGZus{}1d}\PYG{p}{,} \PYG{n}{array\PYGZus{}1d}\PYG{p}{)}

\PYG{c+c1}{\PYGZsh{} Imprimindo os arrays}
\PYG{n+nb}{print}\PYG{p}{(}\PYG{l+s+s2}{\PYGZdq{}}\PYG{l+s+s2}{Soma dos arrays:}\PYG{l+s+se}{\PYGZbs{}n}\PYG{l+s+s2}{\PYGZdq{}}\PYG{p}{,}\PYG{n}{soma\PYGZus{}arrays}\PYG{p}{)}
\PYG{n+nb}{print}\PYG{p}{(}\PYG{l+s+s2}{\PYGZdq{}}\PYG{l+s+s2}{Multiplicação dos arrays:}\PYG{l+s+se}{\PYGZbs{}n}\PYG{l+s+s2}{\PYGZdq{}}\PYG{p}{,}\PYG{n}{mult\PYGZus{}arrays}\PYG{p}{)}
\PYG{n+nb}{print}\PYG{p}{(}\PYG{l+s+s2}{\PYGZdq{}}\PYG{l+s+s2}{Produto escalar dos arrays:}\PYG{l+s+se}{\PYGZbs{}n}\PYG{l+s+s2}{\PYGZdq{}}\PYG{p}{,}\PYG{n}{dot\PYGZus{}product}\PYG{p}{)}
\end{sphinxVerbatim}

\begin{sphinxVerbatim}[commandchars=\\\{\}]
\PYG{n}{Soma} \PYG{n}{dos} \PYG{n}{arrays}\PYG{p}{:}
 \PYG{p}{[} \PYG{l+m+mi}{2}  \PYG{l+m+mi}{4}  \PYG{l+m+mi}{6}  \PYG{l+m+mi}{8} \PYG{l+m+mi}{10}\PYG{p}{]}
\PYG{n}{Multiplicação} \PYG{n}{dos} \PYG{n}{arrays}\PYG{p}{:}
 \PYG{p}{[} \PYG{l+m+mi}{3}  \PYG{l+m+mi}{6}  \PYG{l+m+mi}{9} \PYG{l+m+mi}{12} \PYG{l+m+mi}{15}\PYG{p}{]}
\PYG{n}{Produto} \PYG{n}{escalar} \PYG{n}{dos} \PYG{n}{arrays}\PYG{p}{:}
 \PYG{l+m+mi}{55}
\end{sphinxVerbatim}


\subsubsection{Criando Arrays com valores específicos}
\label{\detokenize{chapters/ch7/ch7:criando-arrays-com-valores-especificos}}
\begin{sphinxVerbatim}[commandchars=\\\{\}]
\PYG{c+c1}{\PYGZsh{} Array unidimensional com valores espaçados uniformemente}
\PYG{n}{linear\PYGZus{}array} \PYG{o}{=} \PYG{n}{np}\PYG{o}{.}\PYG{n}{linspace}\PYG{p}{(}\PYG{l+m+mi}{0}\PYG{p}{,} \PYG{l+m+mi}{1}\PYG{p}{,} \PYG{l+m+mi}{5}\PYG{p}{)}  \PYG{c+c1}{\PYGZsh{} Cria um array com 5 elementos de 0 a 1}

\PYG{c+c1}{\PYGZsh{} Array bidimensional com valores específicos}
\PYG{n}{custom\PYGZus{}array} \PYG{o}{=} \PYG{n}{np}\PYG{o}{.}\PYG{n}{array}\PYG{p}{(}\PYG{p}{[}\PYG{p}{[}\PYG{l+m+mi}{10}\PYG{p}{,} \PYG{l+m+mi}{20}\PYG{p}{,} \PYG{l+m+mi}{30}\PYG{p}{]}\PYG{p}{,} \PYG{p}{[}\PYG{l+m+mi}{40}\PYG{p}{,} \PYG{l+m+mi}{50}\PYG{p}{,} \PYG{l+m+mi}{60}\PYG{p}{]}\PYG{p}{]}\PYG{p}{)}

\PYG{c+c1}{\PYGZsh{} Imprimindo os arrays}
\PYG{n+nb}{print}\PYG{p}{(}\PYG{l+s+s2}{\PYGZdq{}}\PYG{l+s+s2}{Array unidimensional: }\PYG{l+s+se}{\PYGZbs{}n}\PYG{l+s+s2}{\PYGZdq{}}\PYG{p}{,} \PYG{n}{linear\PYGZus{}array}\PYG{p}{)}
\PYG{n+nb}{print}\PYG{p}{(}\PYG{l+s+s2}{\PYGZdq{}}\PYG{l+s+s2}{Array bidimensional:}\PYG{l+s+se}{\PYGZbs{}n}\PYG{l+s+s2}{\PYGZdq{}}\PYG{p}{,} \PYG{n}{custom\PYGZus{}array}\PYG{p}{)}
\end{sphinxVerbatim}

\begin{sphinxVerbatim}[commandchars=\\\{\}]
\PYG{n}{Array} \PYG{n}{unidimensional}\PYG{p}{:}
 \PYG{p}{[}\PYG{l+m+mf}{0.}   \PYG{l+m+mf}{0.25} \PYG{l+m+mf}{0.5}  \PYG{l+m+mf}{0.75} \PYG{l+m+mf}{1.}  \PYG{p}{]}
\PYG{n}{Array} \PYG{n}{bidimensional}\PYG{p}{:}
 \PYG{p}{[}\PYG{p}{[}\PYG{l+m+mi}{10} \PYG{l+m+mi}{20} \PYG{l+m+mi}{30}\PYG{p}{]}
 \PYG{p}{[}\PYG{l+m+mi}{40} \PYG{l+m+mi}{50} \PYG{l+m+mi}{60}\PYG{p}{]}\PYG{p}{]}
\end{sphinxVerbatim}


\subsubsection{Concatenando e Empilhando Arrays}
\label{\detokenize{chapters/ch7/ch7:concatenando-e-empilhando-arrays}}
\begin{sphinxVerbatim}[commandchars=\\\{\}]
\PYG{c+c1}{\PYGZsh{} Concatenando dois arrays unidimensionais}
\PYG{n}{array\PYGZus{}concatenada} \PYG{o}{=} \PYG{n}{np}\PYG{o}{.}\PYG{n}{concatenate}\PYG{p}{(}\PYG{p}{(}\PYG{n}{array\PYGZus{}1d}\PYG{p}{,} \PYG{n}{linear\PYGZus{}array}\PYG{p}{)}\PYG{p}{)}

\PYG{c+c1}{\PYGZsh{} Empilhando dois arrays bidimensionais verticalmente}
\PYG{n}{stacked\PYGZus{}array\PYGZus{}vertical} \PYG{o}{=} \PYG{n}{np}\PYG{o}{.}\PYG{n}{vstack}\PYG{p}{(}\PYG{p}{(}\PYG{n}{array\PYGZus{}2d}\PYG{p}{,} \PYG{n}{custom\PYGZus{}array}\PYG{p}{)}\PYG{p}{)}

\PYG{c+c1}{\PYGZsh{} Empilhando dois arrays bidimensionais horizontalmente}
\PYG{n}{stacked\PYGZus{}array\PYGZus{}horizontal} \PYG{o}{=} \PYG{n}{np}\PYG{o}{.}\PYG{n}{hstack}\PYG{p}{(}\PYG{p}{(}\PYG{n}{array\PYGZus{}2d}\PYG{p}{,} \PYG{n}{custom\PYGZus{}array}\PYG{p}{)}\PYG{p}{)}

\PYG{c+c1}{\PYGZsh{} Imprimindo os arrays concatenados e empilhados}
\PYG{n+nb}{print}\PYG{p}{(}\PYG{l+s+s2}{\PYGZdq{}}\PYG{l+s+s2}{Array concatenada: }\PYG{l+s+se}{\PYGZbs{}n}\PYG{l+s+s2}{\PYGZdq{}}\PYG{p}{,}\PYG{n}{array\PYGZus{}concatenada}\PYG{p}{)}
\PYG{n+nb}{print}\PYG{p}{(}\PYG{l+s+s2}{\PYGZdq{}}\PYG{l+s+s2}{Array empilhada verticalmente: }\PYG{l+s+se}{\PYGZbs{}n}\PYG{l+s+s2}{\PYGZdq{}}\PYG{p}{,} \PYG{n}{stacked\PYGZus{}array\PYGZus{}vertical}\PYG{p}{)}
\PYG{n+nb}{print}\PYG{p}{(}\PYG{l+s+s2}{\PYGZdq{}}\PYG{l+s+s2}{Array empilhada horizontalmente: }\PYG{l+s+se}{\PYGZbs{}n}\PYG{l+s+s2}{\PYGZdq{}}\PYG{p}{,}\PYG{n}{stacked\PYGZus{}array\PYGZus{}horizontal}\PYG{p}{)}

\end{sphinxVerbatim}

\begin{sphinxVerbatim}[commandchars=\\\{\}]
\PYG{n}{Array} \PYG{n}{concatenada}\PYG{p}{:}
 \PYG{p}{[}\PYG{l+m+mf}{1.}   \PYG{l+m+mf}{2.}   \PYG{l+m+mf}{3.}   \PYG{l+m+mf}{4.}   \PYG{l+m+mf}{5.}   \PYG{l+m+mf}{0.}   \PYG{l+m+mf}{0.25} \PYG{l+m+mf}{0.5}  \PYG{l+m+mf}{0.75} \PYG{l+m+mf}{1.}  \PYG{p}{]}
\PYG{n}{Array} \PYG{n}{empilhada} \PYG{n}{verticalmente}\PYG{p}{:}
 \PYG{p}{[}\PYG{p}{[} \PYG{l+m+mi}{1}  \PYG{l+m+mi}{2}  \PYG{l+m+mi}{3}\PYG{p}{]}
 \PYG{p}{[} \PYG{l+m+mi}{4}  \PYG{l+m+mi}{5}  \PYG{l+m+mi}{6}\PYG{p}{]}
 \PYG{p}{[}\PYG{l+m+mi}{10} \PYG{l+m+mi}{20} \PYG{l+m+mi}{30}\PYG{p}{]}
 \PYG{p}{[}\PYG{l+m+mi}{40} \PYG{l+m+mi}{50} \PYG{l+m+mi}{60}\PYG{p}{]}\PYG{p}{]}
\PYG{n}{Array} \PYG{n}{empilhada} \PYG{n}{horizontalmente}\PYG{p}{:}
 \PYG{p}{[}\PYG{p}{[} \PYG{l+m+mi}{1}  \PYG{l+m+mi}{2}  \PYG{l+m+mi}{3} \PYG{l+m+mi}{10} \PYG{l+m+mi}{20} \PYG{l+m+mi}{30}\PYG{p}{]}
 \PYG{p}{[} \PYG{l+m+mi}{4}  \PYG{l+m+mi}{5}  \PYG{l+m+mi}{6} \PYG{l+m+mi}{40} \PYG{l+m+mi}{50} \PYG{l+m+mi}{60}\PYG{p}{]}\PYG{p}{]}

\end{sphinxVerbatim}


\subsubsection{Reshape e Transposição}
\label{\detokenize{chapters/ch7/ch7:reshape-e-transposicao}}
\begin{sphinxVerbatim}[commandchars=\\\{\}]
\PYG{c+c1}{\PYGZsh{} Mudando a forma (reshape) de um array}
\PYG{n}{array\PYGZus{}novo\PYGZus{}shape} \PYG{o}{=} \PYG{n}{array\PYGZus{}1d}\PYG{o}{.}\PYG{n}{reshape}\PYG{p}{(}\PYG{p}{(}\PYG{l+m+mi}{5}\PYG{p}{,} \PYG{l+m+mi}{1}\PYG{p}{)}\PYG{p}{)}

\PYG{c+c1}{\PYGZsh{} Transpondo um array bidimensional}
\PYG{n}{array\PYGZus{}transposta} \PYG{o}{=} \PYG{n}{array\PYGZus{}2d}\PYG{o}{.}\PYG{n}{T}

\PYG{c+c1}{\PYGZsh{} Iprimindo os arrays}
\PYG{n+nb}{print}\PYG{p}{(}\PYG{l+s+s2}{\PYGZdq{}}\PYG{l+s+s2}{Array com novom shape:}\PYG{l+s+se}{\PYGZbs{}n}\PYG{l+s+s2}{\PYGZdq{}}\PYG{p}{,} \PYG{n}{array\PYGZus{}novo\PYGZus{}shape}\PYG{p}{)}
\PYG{n+nb}{print}\PYG{p}{(}\PYG{l+s+s2}{\PYGZdq{}}\PYG{l+s+s2}{Array transposta:}\PYG{l+s+se}{\PYGZbs{}n}\PYG{l+s+s2}{\PYGZdq{}}\PYG{p}{,} \PYG{n}{array\PYGZus{}transposta}\PYG{p}{)}
\end{sphinxVerbatim}

\begin{sphinxVerbatim}[commandchars=\\\{\}]
\PYG{n}{Array} \PYG{n}{com} \PYG{n}{novom} \PYG{n}{shape}\PYG{p}{:}
 \PYG{p}{[}\PYG{p}{[}\PYG{l+m+mi}{1}\PYG{p}{]}
 \PYG{p}{[}\PYG{l+m+mi}{2}\PYG{p}{]}
 \PYG{p}{[}\PYG{l+m+mi}{3}\PYG{p}{]}
 \PYG{p}{[}\PYG{l+m+mi}{4}\PYG{p}{]}
 \PYG{p}{[}\PYG{l+m+mi}{5}\PYG{p}{]}\PYG{p}{]}
\PYG{n}{Array} \PYG{n}{transposta}\PYG{p}{:}
 \PYG{p}{[}\PYG{p}{[}\PYG{l+m+mi}{1} \PYG{l+m+mi}{4}\PYG{p}{]}
 \PYG{p}{[}\PYG{l+m+mi}{2} \PYG{l+m+mi}{5}\PYG{p}{]}
 \PYG{p}{[}\PYG{l+m+mi}{3} \PYG{l+m+mi}{6}\PYG{p}{]}\PYG{p}{]}
\end{sphinxVerbatim}


\subsubsection{Operações Estatísticas}
\label{\detokenize{chapters/ch7/ch7:operacoes-estatisticas}}
\begin{sphinxVerbatim}[commandchars=\\\{\}]
\PYG{c+c1}{\PYGZsh{} Média e desvio padrão de um array}
\PYG{n}{valor\PYGZus{}medio} \PYG{o}{=} \PYG{n}{np}\PYG{o}{.}\PYG{n}{mean}\PYG{p}{(}\PYG{n}{array\PYGZus{}1d}\PYG{p}{)}
\PYG{n}{desvio\PYGZus{}padrao} \PYG{o}{=} \PYG{n}{np}\PYG{o}{.}\PYG{n}{std}\PYG{p}{(}\PYG{n}{array\PYGZus{}1d}\PYG{p}{)}

\PYG{c+c1}{\PYGZsh{} Mínimo e máximo de um array bidimensional}
\PYG{n}{min\PYGZus{}value} \PYG{o}{=} \PYG{n}{np}\PYG{o}{.}\PYG{n}{min}\PYG{p}{(}\PYG{n}{array\PYGZus{}2d}\PYG{p}{)}
\PYG{n}{max\PYGZus{}value} \PYG{o}{=} \PYG{n}{np}\PYG{o}{.}\PYG{n}{max}\PYG{p}{(}\PYG{n}{array\PYGZus{}2d}\PYG{p}{,} \PYG{n}{axis}\PYG{o}{=}\PYG{l+m+mi}{1}\PYG{p}{)}  \PYG{c+c1}{\PYGZsh{} Máximo ao longo das linhas}

\PYG{c+c1}{\PYGZsh{} Imprimindo os valores}
\PYG{n+nb}{print}\PYG{p}{(}\PYG{l+s+s2}{\PYGZdq{}}\PYG{l+s+s2}{Média:}\PYG{l+s+s2}{\PYGZdq{}}\PYG{p}{,} \PYG{n}{valor\PYGZus{}medio}\PYG{p}{)}
\PYG{n+nb}{print}\PYG{p}{(}\PYG{l+s+s2}{\PYGZdq{}}\PYG{l+s+s2}{Désvio Padrão:}\PYG{l+s+s2}{\PYGZdq{}}\PYG{p}{,} \PYG{n}{desvio\PYGZus{}padrao}\PYG{p}{)}
\PYG{n+nb}{print}\PYG{p}{(}\PYG{l+s+s2}{\PYGZdq{}}\PYG{l+s+s2}{Valor Mínimo:}\PYG{l+s+s2}{\PYGZdq{}}\PYG{p}{,} \PYG{n}{min\PYGZus{}value}\PYG{p}{)}
\PYG{n+nb}{print}\PYG{p}{(}\PYG{l+s+s2}{\PYGZdq{}}\PYG{l+s+s2}{Valor Máximo:}\PYG{l+s+s2}{\PYGZdq{}}\PYG{p}{,} \PYG{n}{max\PYGZus{}value}\PYG{p}{)}
\end{sphinxVerbatim}

\begin{sphinxVerbatim}[commandchars=\\\{\}]
\PYG{n}{Média}\PYG{p}{:} \PYG{l+m+mf}{3.0}
\PYG{n}{Désvio} \PYG{n}{Padrão}\PYG{p}{:} \PYG{l+m+mf}{1.4142135623730951}
\PYG{n}{Valor} \PYG{n}{Mínimo}\PYG{p}{:} \PYG{l+m+mi}{1}
\PYG{n}{Valor} \PYG{n}{Máximo}\PYG{p}{:} \PYG{p}{[}\PYG{l+m+mi}{3} \PYG{l+m+mi}{6}\PYG{p}{]}

\end{sphinxVerbatim}


\subsubsection{Indexação e Fatiamento}
\label{\detokenize{chapters/ch7/ch7:indexacao-e-fatiamento}}
\begin{sphinxVerbatim}[commandchars=\\\{\}]
\PYG{c+c1}{\PYGZsh{} Acessando elementos específicos}
\PYG{n}{elemento\PYGZus{}2nd} \PYG{o}{=} \PYG{n}{array\PYGZus{}1d}\PYG{p}{[}\PYG{l+m+mi}{1}\PYG{p}{]}

\PYG{c+c1}{\PYGZsh{} Fatiando(slicing) uma array}
\PYG{n}{slice\PYGZus{}array} \PYG{o}{=} \PYG{n}{array\PYGZus{}1d}\PYG{p}{[}\PYG{l+m+mi}{1}\PYG{p}{:}\PYG{l+m+mi}{4}\PYG{p}{]}

\PYG{c+c1}{\PYGZsh{} Imprimindo os elementos}
\PYG{n+nb}{print}\PYG{p}{(}\PYG{l+s+s2}{\PYGZdq{}}\PYG{l+s+s2}{Elemento 2nd:}\PYG{l+s+s2}{\PYGZdq{}}\PYG{p}{,} \PYG{n}{elemento\PYGZus{}2nd}\PYG{p}{)}
\PYG{n+nb}{print}\PYG{p}{(}\PYG{l+s+s2}{\PYGZdq{}}\PYG{l+s+s2}{Slice:}\PYG{l+s+s2}{\PYGZdq{}}\PYG{p}{,} \PYG{n}{slice\PYGZus{}array}\PYG{p}{)}
\end{sphinxVerbatim}

\begin{sphinxVerbatim}[commandchars=\\\{\}]
\PYG{n}{Elemento} \PYG{l+m+mi}{2}\PYG{n}{nd}\PYG{p}{:} \PYG{l+m+mi}{2}
\PYG{n}{Slice}\PYG{p}{:} \PYG{p}{[}\PYG{l+m+mi}{2} \PYG{l+m+mi}{3} \PYG{l+m+mi}{4}\PYG{p}{]}
\end{sphinxVerbatim}


\subsubsection{Operações Lógicas}
\label{\detokenize{chapters/ch7/ch7:operacoes-logicas}}
\begin{sphinxVerbatim}[commandchars=\\\{\}]
\PYG{c+c1}{\PYGZsh{} Filtrando elementos maiores que 3}
\PYG{n}{array\PYGZus{}filtrado} \PYG{o}{=} \PYG{n}{array\PYGZus{}1d}\PYG{p}{[}\PYG{n}{array\PYGZus{}1d} \PYG{o}{\PYGZgt{}} \PYG{l+m+mi}{3}\PYG{p}{]}

\PYG{c+c1}{\PYGZsh{} Substituindo valores baseados em uma condição}
\PYG{n}{array\PYGZus{}1d}\PYG{p}{[}\PYG{n}{array\PYGZus{}1d} \PYG{o}{\PYGZgt{}} \PYG{l+m+mi}{3}\PYG{p}{]} \PYG{o}{=} \PYG{l+m+mi}{0}

\PYG{c+c1}{\PYGZsh{} Imprimindo os arrays}
\PYG{n+nb}{print}\PYG{p}{(}\PYG{l+s+s2}{\PYGZdq{}}\PYG{l+s+s2}{Arrray Filtrado:}\PYG{l+s+s2}{\PYGZdq{}}\PYG{p}{,} \PYG{n}{array\PYGZus{}filtrado}\PYG{p}{)}
\PYG{n+nb}{print}\PYG{p}{(}\PYG{l+s+s2}{\PYGZdq{}}\PYG{l+s+s2}{Array 1D:}\PYG{l+s+s2}{\PYGZdq{}}\PYG{p}{,} \PYG{n}{array\PYGZus{}1d}\PYG{p}{)}
\end{sphinxVerbatim}

\begin{sphinxVerbatim}[commandchars=\\\{\}]
\PYG{n}{Arrray} \PYG{n}{Filtrado}\PYG{p}{:} \PYG{p}{[}\PYG{l+m+mi}{4} \PYG{l+m+mi}{5}\PYG{p}{]}
\PYG{n}{Array} \PYG{l+m+mi}{1}\PYG{n}{D}\PYG{p}{:} \PYG{p}{[}\PYG{l+m+mi}{1} \PYG{l+m+mi}{2} \PYG{l+m+mi}{3} \PYG{l+m+mi}{0} \PYG{l+m+mi}{0}\PYG{p}{]}
\end{sphinxVerbatim}


\section{Visualização de Dados com o Matplotlib}
\label{\detokenize{chapters/ch7/ch7:visualizacao-de-dados-com-o-matplotlib}}
\sphinxAtStartPar
\sphinxstylestrong{Visualização de Dados com \sphinxhref{https://matplotlib.org/}{Matplotlib}}

\sphinxAtStartPar
A visualização de dados é essencial para analisar e compreender informações, proporcionando clareza na apresentação de dados e facilitando a identificação de padrões e tendências.

\sphinxAtStartPar
O Matplotlib, uma biblioteca de visualização de dados em Python, é uma escolha popular entre cientistas de dados, engenheiros e analistas de negócios. Além de ser de código aberto e gratuito, destaca\sphinxhyphen{}se por sua facilidade de uso.

\sphinxAtStartPar
\sphinxstylestrong{Funcionalidades do Matplotlib}

\sphinxAtStartPar
A biblioteca oferece uma ampla gama de funcionalidades, incluindo gráficos de linha, barras, dispersão, área, pizza, histogramas, boxplots, violin plots, mapas de calor, nuvem de palavras, entre outros. A personalização de gráficos também é uma característica marcante.

\sphinxAtStartPar
\sphinxstylestrong{Exemplos de Visualizações com Matplotlib}
\begin{itemize}
\item {} 
\sphinxAtStartPar
Gráfico de linha: Evolução de casos de COVID\sphinxhyphen{}19 no Brasil.

\item {} 
\sphinxAtStartPar
Gráfico de barras: Distribuição de renda no Brasil.

\item {} 
\sphinxAtStartPar
Gráfico de dispersão: Correlação entre duas variáveis.

\item {} 
\sphinxAtStartPar
Histograma: Distribuição de uma variável.

\end{itemize}

\sphinxAtStartPar
\sphinxstylestrong{Integração com Outras Bibliotecas}

\sphinxAtStartPar
O Matplotlib integra\sphinxhyphen{}se de maneira eficiente com outras bibliotecas, como NumPy, proporcionando um ambiente poderoso para análise de dados. Essa integração permite, por exemplo, visualizar dados gerados por algoritmos de aprendizado de máquina.


\subsection{Gráficos Simples}
\label{\detokenize{chapters/ch7/ch7:graficos-simples}}
\sphinxAtStartPar
Para introduzir a criação de gráficos, começaremos com um exemplo simples de gráfico de linha usando a biblioteca Matplotlib.

\begin{sphinxuseclass}{cell}\begin{sphinxVerbatimInput}

\begin{sphinxuseclass}{cell_input}
\begin{sphinxVerbatim}[commandchars=\\\{\}]
\PYG{k+kn}{import} \PYG{n+nn}{matplotlib}\PYG{n+nn}{.}\PYG{n+nn}{pyplot} \PYG{k}{as} \PYG{n+nn}{plt}

\PYG{c+c1}{\PYGZsh{} Dados de exemplo}
\PYG{n}{x} \PYG{o}{=} \PYG{p}{[}\PYG{l+m+mi}{1}\PYG{p}{,} \PYG{l+m+mi}{2}\PYG{p}{,} \PYG{l+m+mi}{3}\PYG{p}{,} \PYG{l+m+mi}{4}\PYG{p}{,} \PYG{l+m+mi}{5}\PYG{p}{]}
\PYG{n}{y} \PYG{o}{=} \PYG{p}{[}\PYG{l+m+mi}{2}\PYG{p}{,} \PYG{l+m+mi}{4}\PYG{p}{,} \PYG{l+m+mi}{6}\PYG{p}{,} \PYG{l+m+mi}{8}\PYG{p}{,} \PYG{l+m+mi}{10}\PYG{p}{]}

\PYG{c+c1}{\PYGZsh{} Criando o gráfico de linha}
\PYG{n}{plt}\PYG{o}{.}\PYG{n}{plot}\PYG{p}{(}\PYG{n}{x}\PYG{p}{,} \PYG{n}{y}\PYG{p}{,} \PYG{n}{marker}\PYG{o}{=}\PYG{l+s+s1}{\PYGZsq{}}\PYG{l+s+s1}{o}\PYG{l+s+s1}{\PYGZsq{}}\PYG{p}{,} \PYG{n}{linestyle}\PYG{o}{=}\PYG{l+s+s1}{\PYGZsq{}}\PYG{l+s+s1}{\PYGZhy{}}\PYG{l+s+s1}{\PYGZsq{}}\PYG{p}{,} \PYG{n}{color}\PYG{o}{=}\PYG{l+s+s1}{\PYGZsq{}}\PYG{l+s+s1}{b}\PYG{l+s+s1}{\PYGZsq{}}\PYG{p}{,} \PYG{n}{label}\PYG{o}{=}\PYG{l+s+s1}{\PYGZsq{}}\PYG{l+s+s1}{Dados de Exemplo}\PYG{l+s+s1}{\PYGZsq{}}\PYG{p}{)}

\PYG{c+c1}{\PYGZsh{} Adicionando rótulos e título}
\PYG{n}{plt}\PYG{o}{.}\PYG{n}{xlabel}\PYG{p}{(}\PYG{l+s+s1}{\PYGZsq{}}\PYG{l+s+s1}{Eixo X}\PYG{l+s+s1}{\PYGZsq{}}\PYG{p}{)}
\PYG{n}{plt}\PYG{o}{.}\PYG{n}{ylabel}\PYG{p}{(}\PYG{l+s+s1}{\PYGZsq{}}\PYG{l+s+s1}{Eixo Y}\PYG{l+s+s1}{\PYGZsq{}}\PYG{p}{)}
\PYG{n}{plt}\PYG{o}{.}\PYG{n}{title}\PYG{p}{(}\PYG{l+s+s1}{\PYGZsq{}}\PYG{l+s+s1}{Gráfico de Linha Simples}\PYG{l+s+s1}{\PYGZsq{}}\PYG{p}{)}
\PYG{n}{plt}\PYG{o}{.}\PYG{n}{legend}\PYG{p}{(}\PYG{p}{)}

\PYG{c+c1}{\PYGZsh{} Exibindo o gráfico}
\PYG{n}{plt}\PYG{o}{.}\PYG{n}{show}\PYG{p}{(}\PYG{p}{)}
\end{sphinxVerbatim}

\end{sphinxuseclass}\end{sphinxVerbatimInput}
\begin{sphinxVerbatimOutput}

\begin{sphinxuseclass}{cell_output}
\noindent\sphinxincludegraphics{{4d3ac3ec4b8c73b18d62454d685b95f58bdbc32b84dd7a9ebaa116a8bcadd006}.png}

\end{sphinxuseclass}\end{sphinxVerbatimOutput}

\end{sphinxuseclass}
\sphinxAtStartPar
O código começa importando a biblioteca Matplotlib usando \sphinxcode{\sphinxupquote{import matplotlib.pyplot as plt}}. Isso é essencial para utilizar as funcionalidades da biblioteca, e o alias \sphinxcode{\sphinxupquote{plt}} é comumente usado para facilitar o acesso a essas funções. Em seguida, o código continua com a definição dos dados e a criação do gráfico de linha. Ao final, a função \sphinxcode{\sphinxupquote{show()}} é utilizada para exibir o gráfico.


\subsection{Gráfico de Linha: Evolução de Casos de COVID\sphinxhyphen{}19 no Brasil}
\label{\detokenize{chapters/ch7/ch7:grafico-de-linha-evolucao-de-casos-de-covid-19-no-brasil}}
\sphinxAtStartPar
Este exemplo utiliza o Matplotlib para criar um gráfico de linha representando a evolução dos casos de COVID\sphinxhyphen{}19 no Brasil. A biblioteca NumPy é utilizada para gerar dados fictícios para ilustração.

\begin{sphinxuseclass}{cell}\begin{sphinxVerbatimInput}

\begin{sphinxuseclass}{cell_input}
\begin{sphinxVerbatim}[commandchars=\\\{\}]
\PYG{k+kn}{import} \PYG{n+nn}{matplotlib}\PYG{n+nn}{.}\PYG{n+nn}{pyplot} \PYG{k}{as} \PYG{n+nn}{plt}
\PYG{k+kn}{import} \PYG{n+nn}{numpy} \PYG{k}{as} \PYG{n+nn}{np}

\PYG{c+c1}{\PYGZsh{} Dados fictícios para ilustração}
\PYG{n}{dias} \PYG{o}{=} \PYG{n}{np}\PYG{o}{.}\PYG{n}{arange}\PYG{p}{(}\PYG{l+m+mi}{1}\PYG{p}{,} \PYG{l+m+mi}{31}\PYG{p}{)}
\PYG{n}{casos\PYGZus{}diarios} \PYG{o}{=} \PYG{n}{np}\PYG{o}{.}\PYG{n}{random}\PYG{o}{.}\PYG{n}{randint}\PYG{p}{(}\PYG{l+m+mi}{100}\PYG{p}{,} \PYG{l+m+mi}{1000}\PYG{p}{,} \PYG{n}{size}\PYG{o}{=}\PYG{l+m+mi}{30}\PYG{p}{)}\PYG{o}{.}\PYG{n}{cumsum}\PYG{p}{(}\PYG{p}{)}

\PYG{c+c1}{\PYGZsh{} Criando o gráfico de linha}
\PYG{n}{plt}\PYG{o}{.}\PYG{n}{plot}\PYG{p}{(}\PYG{n}{dias}\PYG{p}{,} \PYG{n}{casos\PYGZus{}diarios}\PYG{p}{,} \PYG{n}{marker}\PYG{o}{=}\PYG{l+s+s1}{\PYGZsq{}}\PYG{l+s+s1}{o}\PYG{l+s+s1}{\PYGZsq{}}\PYG{p}{,} \PYG{n}{linestyle}\PYG{o}{=}\PYG{l+s+s1}{\PYGZsq{}}\PYG{l+s+s1}{\PYGZhy{}}\PYG{l+s+s1}{\PYGZsq{}}\PYG{p}{,} \PYG{n}{color}\PYG{o}{=}\PYG{l+s+s1}{\PYGZsq{}}\PYG{l+s+s1}{b}\PYG{l+s+s1}{\PYGZsq{}}\PYG{p}{,} \PYG{n}{label}\PYG{o}{=}\PYG{l+s+s1}{\PYGZsq{}}\PYG{l+s+s1}{Casos Diários}\PYG{l+s+s1}{\PYGZsq{}}\PYG{p}{)}

\PYG{c+c1}{\PYGZsh{} Adicionando rótulos e título}
\PYG{n}{plt}\PYG{o}{.}\PYG{n}{xlabel}\PYG{p}{(}\PYG{l+s+s1}{\PYGZsq{}}\PYG{l+s+s1}{Dias}\PYG{l+s+s1}{\PYGZsq{}}\PYG{p}{)}
\PYG{n}{plt}\PYG{o}{.}\PYG{n}{ylabel}\PYG{p}{(}\PYG{l+s+s1}{\PYGZsq{}}\PYG{l+s+s1}{Total de Casos}\PYG{l+s+s1}{\PYGZsq{}}\PYG{p}{)}
\PYG{n}{plt}\PYG{o}{.}\PYG{n}{title}\PYG{p}{(}\PYG{l+s+s1}{\PYGZsq{}}\PYG{l+s+s1}{Evolução de Casos de COVID\PYGZhy{}19 no Brasil}\PYG{l+s+s1}{\PYGZsq{}}\PYG{p}{)}
\PYG{n}{plt}\PYG{o}{.}\PYG{n}{legend}\PYG{p}{(}\PYG{p}{)}

\PYG{c+c1}{\PYGZsh{} Exibindo o gráfico}
\PYG{n}{plt}\PYG{o}{.}\PYG{n}{show}\PYG{p}{(}\PYG{p}{)}
\end{sphinxVerbatim}

\end{sphinxuseclass}\end{sphinxVerbatimInput}
\begin{sphinxVerbatimOutput}

\begin{sphinxuseclass}{cell_output}
\noindent\sphinxincludegraphics{{5f68507cc2df024a28fceb005984bb5750251f785fc3d662f075d161bbd88451}.png}

\end{sphinxuseclass}\end{sphinxVerbatimOutput}

\end{sphinxuseclass}

\subsection{Gráfico de Barras: Distribuição de Renda no Brasil}
\label{\detokenize{chapters/ch7/ch7:grafico-de-barras-distribuicao-de-renda-no-brasil}}
\sphinxAtStartPar
Neste exemplo, um gráfico de barras é utilizado para representar a distribuição de renda no Brasil. Os dados são fictícios e representam percentuais da população em diferentes faixas de renda.

\begin{sphinxuseclass}{cell}\begin{sphinxVerbatimInput}

\begin{sphinxuseclass}{cell_input}
\begin{sphinxVerbatim}[commandchars=\\\{\}]
\PYG{k+kn}{import} \PYG{n+nn}{matplotlib}\PYG{n+nn}{.}\PYG{n+nn}{pyplot} \PYG{k}{as} \PYG{n+nn}{plt}

\PYG{c+c1}{\PYGZsh{} Dados fictícios para ilustração}
\PYG{n}{faixas\PYGZus{}renda} \PYG{o}{=} \PYG{p}{[}\PYG{l+s+s1}{\PYGZsq{}}\PYG{l+s+s1}{0\PYGZhy{}999}\PYG{l+s+s1}{\PYGZsq{}}\PYG{p}{,} \PYG{l+s+s1}{\PYGZsq{}}\PYG{l+s+s1}{1000\PYGZhy{}1999}\PYG{l+s+s1}{\PYGZsq{}}\PYG{p}{,} \PYG{l+s+s1}{\PYGZsq{}}\PYG{l+s+s1}{2000\PYGZhy{}2999}\PYG{l+s+s1}{\PYGZsq{}}\PYG{p}{,} \PYG{l+s+s1}{\PYGZsq{}}\PYG{l+s+s1}{3000\PYGZhy{}3999}\PYG{l+s+s1}{\PYGZsq{}}\PYG{p}{,} \PYG{l+s+s1}{\PYGZsq{}}\PYG{l+s+s1}{4000+}\PYG{l+s+s1}{\PYGZsq{}}\PYG{p}{]}
\PYG{n}{percentuais} \PYG{o}{=} \PYG{p}{[}\PYG{l+m+mi}{20}\PYG{p}{,} \PYG{l+m+mi}{30}\PYG{p}{,} \PYG{l+m+mi}{25}\PYG{p}{,} \PYG{l+m+mi}{15}\PYG{p}{,} \PYG{l+m+mi}{10}\PYG{p}{]}

\PYG{c+c1}{\PYGZsh{} Criando o gráfico de barras}
\PYG{n}{plt}\PYG{o}{.}\PYG{n}{bar}\PYG{p}{(}\PYG{n}{faixas\PYGZus{}renda}\PYG{p}{,} \PYG{n}{percentuais}\PYG{p}{,} \PYG{n}{color}\PYG{o}{=}\PYG{l+s+s1}{\PYGZsq{}}\PYG{l+s+s1}{green}\PYG{l+s+s1}{\PYGZsq{}}\PYG{p}{)}

\PYG{c+c1}{\PYGZsh{} Adicionando rótulos e título}
\PYG{n}{plt}\PYG{o}{.}\PYG{n}{xlabel}\PYG{p}{(}\PYG{l+s+s1}{\PYGZsq{}}\PYG{l+s+s1}{Faixas de Renda}\PYG{l+s+s1}{\PYGZsq{}}\PYG{p}{)}
\PYG{n}{plt}\PYG{o}{.}\PYG{n}{ylabel}\PYG{p}{(}\PYG{l+s+s1}{\PYGZsq{}}\PYG{l+s+s1}{Percentual da População}\PYG{l+s+s1}{\PYGZsq{}}\PYG{p}{)}
\PYG{n}{plt}\PYG{o}{.}\PYG{n}{title}\PYG{p}{(}\PYG{l+s+s1}{\PYGZsq{}}\PYG{l+s+s1}{Distribuição de Renda no Brasil}\PYG{l+s+s1}{\PYGZsq{}}\PYG{p}{)}

\PYG{c+c1}{\PYGZsh{} Exibindo o gráfico}
\PYG{n}{plt}\PYG{o}{.}\PYG{n}{show}\PYG{p}{(}\PYG{p}{)}
\end{sphinxVerbatim}

\end{sphinxuseclass}\end{sphinxVerbatimInput}
\begin{sphinxVerbatimOutput}

\begin{sphinxuseclass}{cell_output}
\noindent\sphinxincludegraphics{{aefff5e195bb3a088d4d8127c410ba02a73e9657c4ae69ed81c077f24350309b}.png}

\end{sphinxuseclass}\end{sphinxVerbatimOutput}

\end{sphinxuseclass}

\subsection{Gráfico de Dispersão: Correlação entre Duas Variáveis}
\label{\detokenize{chapters/ch7/ch7:grafico-de-dispersao-correlacao-entre-duas-variaveis}}
\sphinxAtStartPar
Este exemplo utiliza um gráfico de dispersão para visualizar a correlação entre duas variáveis fictícias: horas de estudo e desempenho em exames.

\begin{sphinxuseclass}{cell}\begin{sphinxVerbatimInput}

\begin{sphinxuseclass}{cell_input}
\begin{sphinxVerbatim}[commandchars=\\\{\}]
\PYG{k+kn}{import} \PYG{n+nn}{matplotlib}\PYG{n+nn}{.}\PYG{n+nn}{pyplot} \PYG{k}{as} \PYG{n+nn}{plt}
\PYG{k+kn}{import} \PYG{n+nn}{numpy} \PYG{k}{as} \PYG{n+nn}{np}

\PYG{c+c1}{\PYGZsh{} Dados fictícios para ilustração}
\PYG{n}{horas\PYGZus{}estudo} \PYG{o}{=} \PYG{n}{np}\PYG{o}{.}\PYG{n}{random}\PYG{o}{.}\PYG{n}{uniform}\PYG{p}{(}\PYG{l+m+mi}{1}\PYG{p}{,} \PYG{l+m+mi}{10}\PYG{p}{,} \PYG{n}{size}\PYG{o}{=}\PYG{l+m+mi}{50}\PYG{p}{)}
\PYG{n}{desempenho\PYGZus{}exames} \PYG{o}{=} \PYG{l+m+mi}{70} \PYG{o}{+} \PYG{l+m+mi}{2} \PYG{o}{*} \PYG{n}{horas\PYGZus{}estudo} \PYG{o}{+} \PYG{n}{np}\PYG{o}{.}\PYG{n}{random}\PYG{o}{.}\PYG{n}{normal}\PYG{p}{(}\PYG{l+m+mi}{0}\PYG{p}{,} \PYG{l+m+mi}{2}\PYG{p}{,} \PYG{n}{size}\PYG{o}{=}\PYG{l+m+mi}{50}\PYG{p}{)}

\PYG{c+c1}{\PYGZsh{} Criando o gráfico de dispersão}
\PYG{n}{plt}\PYG{o}{.}\PYG{n}{scatter}\PYG{p}{(}\PYG{n}{horas\PYGZus{}estudo}\PYG{p}{,} \PYG{n}{desempenho\PYGZus{}exames}\PYG{p}{,} \PYG{n}{color}\PYG{o}{=}\PYG{l+s+s1}{\PYGZsq{}}\PYG{l+s+s1}{red}\PYG{l+s+s1}{\PYGZsq{}}\PYG{p}{,} \PYG{n}{alpha}\PYG{o}{=}\PYG{l+m+mf}{0.7}\PYG{p}{)}

\PYG{c+c1}{\PYGZsh{} Adicionando rótulos e título}
\PYG{n}{plt}\PYG{o}{.}\PYG{n}{xlabel}\PYG{p}{(}\PYG{l+s+s1}{\PYGZsq{}}\PYG{l+s+s1}{Horas de Estudo}\PYG{l+s+s1}{\PYGZsq{}}\PYG{p}{)}
\PYG{n}{plt}\PYG{o}{.}\PYG{n}{ylabel}\PYG{p}{(}\PYG{l+s+s1}{\PYGZsq{}}\PYG{l+s+s1}{Desempenho em Exames}\PYG{l+s+s1}{\PYGZsq{}}\PYG{p}{)}
\PYG{n}{plt}\PYG{o}{.}\PYG{n}{title}\PYG{p}{(}\PYG{l+s+s1}{\PYGZsq{}}\PYG{l+s+s1}{Correlação entre Horas de Estudo e Desempenho em Exames}\PYG{l+s+s1}{\PYGZsq{}}\PYG{p}{)}

\PYG{c+c1}{\PYGZsh{} Exibindo o gráfico}
\PYG{n}{plt}\PYG{o}{.}\PYG{n}{show}\PYG{p}{(}\PYG{p}{)}
\end{sphinxVerbatim}

\end{sphinxuseclass}\end{sphinxVerbatimInput}
\begin{sphinxVerbatimOutput}

\begin{sphinxuseclass}{cell_output}
\noindent\sphinxincludegraphics{{0b44ca229dd85695f711f66603e5f05578aa98b0c0a1ebc27ef098e89fef0eba}.png}

\end{sphinxuseclass}\end{sphinxVerbatimOutput}

\end{sphinxuseclass}

\subsection{Histograma: Distribuição de uma Variável}
\label{\detokenize{chapters/ch7/ch7:histograma-distribuicao-de-uma-variavel}}
\sphinxAtStartPar
Neste exemplo, um histograma é utilizado para representar a distribuição de idades em uma amostra de pessoas.

\begin{sphinxuseclass}{cell}\begin{sphinxVerbatimInput}

\begin{sphinxuseclass}{cell_input}
\begin{sphinxVerbatim}[commandchars=\\\{\}]
\PYG{k+kn}{import} \PYG{n+nn}{matplotlib}\PYG{n+nn}{.}\PYG{n+nn}{pyplot} \PYG{k}{as} \PYG{n+nn}{plt}
\PYG{k+kn}{import} \PYG{n+nn}{numpy} \PYG{k}{as} \PYG{n+nn}{np}

\PYG{c+c1}{\PYGZsh{} Dados fictícios para ilustração}
\PYG{n}{idades} \PYG{o}{=} \PYG{n}{np}\PYG{o}{.}\PYG{n}{random}\PYG{o}{.}\PYG{n}{normal}\PYG{p}{(}\PYG{l+m+mi}{30}\PYG{p}{,} \PYG{l+m+mi}{5}\PYG{p}{,} \PYG{n}{size}\PYG{o}{=}\PYG{l+m+mi}{1000}\PYG{p}{)}

\PYG{c+c1}{\PYGZsh{} Criando o histograma}
\PYG{n}{plt}\PYG{o}{.}\PYG{n}{hist}\PYG{p}{(}\PYG{n}{idades}\PYG{p}{,} \PYG{n}{bins}\PYG{o}{=}\PYG{l+m+mi}{20}\PYG{p}{,} \PYG{n}{color}\PYG{o}{=}\PYG{l+s+s1}{\PYGZsq{}}\PYG{l+s+s1}{purple}\PYG{l+s+s1}{\PYGZsq{}}\PYG{p}{,} \PYG{n}{edgecolor}\PYG{o}{=}\PYG{l+s+s1}{\PYGZsq{}}\PYG{l+s+s1}{black}\PYG{l+s+s1}{\PYGZsq{}}\PYG{p}{)}

\PYG{c+c1}{\PYGZsh{} Adicionando rótulos e título}
\PYG{n}{plt}\PYG{o}{.}\PYG{n}{xlabel}\PYG{p}{(}\PYG{l+s+s1}{\PYGZsq{}}\PYG{l+s+s1}{Idade}\PYG{l+s+s1}{\PYGZsq{}}\PYG{p}{)}
\PYG{n}{plt}\PYG{o}{.}\PYG{n}{ylabel}\PYG{p}{(}\PYG{l+s+s1}{\PYGZsq{}}\PYG{l+s+s1}{Frequência}\PYG{l+s+s1}{\PYGZsq{}}\PYG{p}{)}
\PYG{n}{plt}\PYG{o}{.}\PYG{n}{title}\PYG{p}{(}\PYG{l+s+s1}{\PYGZsq{}}\PYG{l+s+s1}{Distribuição de Idades}\PYG{l+s+s1}{\PYGZsq{}}\PYG{p}{)}

\PYG{c+c1}{\PYGZsh{} Exibindo o gráfico}
\PYG{n}{plt}\PYG{o}{.}\PYG{n}{show}\PYG{p}{(}\PYG{p}{)}
\end{sphinxVerbatim}

\end{sphinxuseclass}\end{sphinxVerbatimInput}
\begin{sphinxVerbatimOutput}

\begin{sphinxuseclass}{cell_output}
\noindent\sphinxincludegraphics{{c18dc5a5801168adce9241bce5bf0e4ec857454d5df40b6a99fede7f6bca2a00}.png}

\end{sphinxuseclass}\end{sphinxVerbatimOutput}

\end{sphinxuseclass}

\subsection{Explorando Gráficos Estatísticos}
\label{\detokenize{chapters/ch7/ch7:explorando-graficos-estatisticos}}
\sphinxAtStartPar
Neste exemplo simplificado, abordaremos a representação gráfica de uma distribuição gaussiana bidimensional, utilizando subplots para destacar diferentes características.

\sphinxAtStartPar
Começamos definindo uma distribuição gaussiana bidimensional com média zero e uma matriz de covariância específica. A primeira subplot exibe a densidade de probabilidade, proporcionando uma visão visual da dispersão dos dados nos eixos X e Y.

\sphinxAtStartPar
As subplots seguintes exploram as projeções marginais, revelando a densidade de probabilidade ao longo do eixo X e Y, respectivamente.

\begin{sphinxuseclass}{cell}\begin{sphinxVerbatimInput}

\begin{sphinxuseclass}{cell_input}
\begin{sphinxVerbatim}[commandchars=\\\{\}]
\PYG{k+kn}{import} \PYG{n+nn}{matplotlib}\PYG{n+nn}{.}\PYG{n+nn}{pyplot} \PYG{k}{as} \PYG{n+nn}{plt}
\PYG{k+kn}{import} \PYG{n+nn}{numpy} \PYG{k}{as} \PYG{n+nn}{np}

\PYG{c+c1}{\PYGZsh{} Definindo os parâmetros da distribuição gaussiana}
\PYG{n}{mean} \PYG{o}{=} \PYG{p}{[}\PYG{l+m+mi}{0}\PYG{p}{,} \PYG{l+m+mi}{0}\PYG{p}{]}
\PYG{n}{covariance} \PYG{o}{=} \PYG{p}{[}\PYG{p}{[}\PYG{l+m+mi}{1}\PYG{p}{,} \PYG{l+m+mf}{0.5}\PYG{p}{]}\PYG{p}{,} \PYG{p}{[}\PYG{l+m+mf}{0.5}\PYG{p}{,} \PYG{l+m+mi}{1}\PYG{p}{]}\PYG{p}{]}

\PYG{c+c1}{\PYGZsh{} Criando uma grade de valores para x e y}
\PYG{n}{x} \PYG{o}{=} \PYG{n}{np}\PYG{o}{.}\PYG{n}{linspace}\PYG{p}{(}\PYG{o}{\PYGZhy{}}\PYG{l+m+mi}{3}\PYG{p}{,} \PYG{l+m+mi}{3}\PYG{p}{,} \PYG{l+m+mi}{100}\PYG{p}{)}
\PYG{n}{y} \PYG{o}{=} \PYG{n}{np}\PYG{o}{.}\PYG{n}{linspace}\PYG{p}{(}\PYG{o}{\PYGZhy{}}\PYG{l+m+mi}{3}\PYG{p}{,} \PYG{l+m+mi}{3}\PYG{p}{,} \PYG{l+m+mi}{100}\PYG{p}{)}
\PYG{n}{x}\PYG{p}{,} \PYG{n}{y} \PYG{o}{=} \PYG{n}{np}\PYG{o}{.}\PYG{n}{meshgrid}\PYG{p}{(}\PYG{n}{x}\PYG{p}{,} \PYG{n}{y}\PYG{p}{)}

\PYG{c+c1}{\PYGZsh{} Calculando os valores da distribuição gaussiana manualmente}
\PYG{n}{inv\PYGZus{}covariance} \PYG{o}{=} \PYG{n}{np}\PYG{o}{.}\PYG{n}{linalg}\PYG{o}{.}\PYG{n}{inv}\PYG{p}{(}\PYG{n}{covariance}\PYG{p}{)}
\PYG{n}{det\PYGZus{}covariance} \PYG{o}{=} \PYG{n}{np}\PYG{o}{.}\PYG{n}{linalg}\PYG{o}{.}\PYG{n}{det}\PYG{p}{(}\PYG{n}{covariance}\PYG{p}{)}
\PYG{n}{exponent} \PYG{o}{=} \PYG{o}{\PYGZhy{}}\PYG{l+m+mf}{0.5} \PYG{o}{*} \PYG{p}{(}\PYG{n}{x}\PYG{o}{*}\PYG{o}{*}\PYG{l+m+mi}{2} \PYG{o}{*} \PYG{n}{inv\PYGZus{}covariance}\PYG{p}{[}\PYG{l+m+mi}{0}\PYG{p}{,} \PYG{l+m+mi}{0}\PYG{p}{]} \PYG{o}{+} \PYG{n}{y}\PYG{o}{*}\PYG{o}{*}\PYG{l+m+mi}{2} \PYG{o}{*} \PYG{n}{inv\PYGZus{}covariance}\PYG{p}{[}\PYG{l+m+mi}{1}\PYG{p}{,} \PYG{l+m+mi}{1}\PYG{p}{]} \PYG{o}{+} \PYG{l+m+mi}{2} \PYG{o}{*} \PYG{n}{x} \PYG{o}{*} \PYG{n}{y} \PYG{o}{*} \PYG{n}{inv\PYGZus{}covariance}\PYG{p}{[}\PYG{l+m+mi}{0}\PYG{p}{,} \PYG{l+m+mi}{1}\PYG{p}{]}\PYG{p}{)}
\PYG{n}{z} \PYG{o}{=} \PYG{p}{(}\PYG{l+m+mi}{1} \PYG{o}{/} \PYG{p}{(}\PYG{l+m+mi}{2} \PYG{o}{*} \PYG{n}{np}\PYG{o}{.}\PYG{n}{pi} \PYG{o}{*} \PYG{n}{np}\PYG{o}{.}\PYG{n}{sqrt}\PYG{p}{(}\PYG{n}{det\PYGZus{}covariance}\PYG{p}{)}\PYG{p}{)}\PYG{p}{)} \PYG{o}{*} \PYG{n}{np}\PYG{o}{.}\PYG{n}{exp}\PYG{p}{(}\PYG{n}{exponent}\PYG{p}{)}

\PYG{c+c1}{\PYGZsh{} Criando subplots}
\PYG{n}{fig}\PYG{p}{,} \PYG{n}{axs} \PYG{o}{=} \PYG{n}{plt}\PYG{o}{.}\PYG{n}{subplots}\PYG{p}{(}\PYG{l+m+mi}{1}\PYG{p}{,} \PYG{l+m+mi}{3}\PYG{p}{,} \PYG{n}{figsize}\PYG{o}{=}\PYG{p}{(}\PYG{l+m+mi}{15}\PYG{p}{,} \PYG{l+m+mi}{5}\PYG{p}{)}\PYG{p}{)}

\PYG{c+c1}{\PYGZsh{} Subplot da densidade de probabilidade}
\PYG{n}{axs}\PYG{p}{[}\PYG{l+m+mi}{0}\PYG{p}{]}\PYG{o}{.}\PYG{n}{contourf}\PYG{p}{(}\PYG{n}{x}\PYG{p}{,} \PYG{n}{y}\PYG{p}{,} \PYG{n}{z}\PYG{p}{,} \PYG{n}{cmap}\PYG{o}{=}\PYG{l+s+s1}{\PYGZsq{}}\PYG{l+s+s1}{viridis}\PYG{l+s+s1}{\PYGZsq{}}\PYG{p}{)}
\PYG{n}{axs}\PYG{p}{[}\PYG{l+m+mi}{0}\PYG{p}{]}\PYG{o}{.}\PYG{n}{set\PYGZus{}title}\PYG{p}{(}\PYG{l+s+s1}{\PYGZsq{}}\PYG{l+s+s1}{Densidade de Probabilidade}\PYG{l+s+s1}{\PYGZsq{}}\PYG{p}{)}
\PYG{n}{axs}\PYG{p}{[}\PYG{l+m+mi}{0}\PYG{p}{]}\PYG{o}{.}\PYG{n}{set\PYGZus{}xlabel}\PYG{p}{(}\PYG{l+s+s1}{\PYGZsq{}}\PYG{l+s+s1}{Eixo X}\PYG{l+s+s1}{\PYGZsq{}}\PYG{p}{)}
\PYG{n}{axs}\PYG{p}{[}\PYG{l+m+mi}{0}\PYG{p}{]}\PYG{o}{.}\PYG{n}{set\PYGZus{}ylabel}\PYG{p}{(}\PYG{l+s+s1}{\PYGZsq{}}\PYG{l+s+s1}{Eixo Y}\PYG{l+s+s1}{\PYGZsq{}}\PYG{p}{)}

\PYG{c+c1}{\PYGZsh{} Projeção bidimensional no eixo X}
\PYG{n}{axs}\PYG{p}{[}\PYG{l+m+mi}{1}\PYG{p}{]}\PYG{o}{.}\PYG{n}{plot}\PYG{p}{(}\PYG{n}{x}\PYG{p}{[}\PYG{l+m+mi}{0}\PYG{p}{,} \PYG{p}{:}\PYG{p}{]}\PYG{p}{,} \PYG{n}{np}\PYG{o}{.}\PYG{n}{sum}\PYG{p}{(}\PYG{n}{z}\PYG{p}{,} \PYG{n}{axis}\PYG{o}{=}\PYG{l+m+mi}{0}\PYG{p}{)}\PYG{p}{,} \PYG{n}{color}\PYG{o}{=}\PYG{l+s+s1}{\PYGZsq{}}\PYG{l+s+s1}{blue}\PYG{l+s+s1}{\PYGZsq{}}\PYG{p}{)}
\PYG{n}{axs}\PYG{p}{[}\PYG{l+m+mi}{1}\PYG{p}{]}\PYG{o}{.}\PYG{n}{set\PYGZus{}title}\PYG{p}{(}\PYG{l+s+s1}{\PYGZsq{}}\PYG{l+s+s1}{Projeção no Eixo X}\PYG{l+s+s1}{\PYGZsq{}}\PYG{p}{)}
\PYG{n}{axs}\PYG{p}{[}\PYG{l+m+mi}{1}\PYG{p}{]}\PYG{o}{.}\PYG{n}{set\PYGZus{}xlabel}\PYG{p}{(}\PYG{l+s+s1}{\PYGZsq{}}\PYG{l+s+s1}{Eixo X}\PYG{l+s+s1}{\PYGZsq{}}\PYG{p}{)}
\PYG{n}{axs}\PYG{p}{[}\PYG{l+m+mi}{1}\PYG{p}{]}\PYG{o}{.}\PYG{n}{set\PYGZus{}ylabel}\PYG{p}{(}\PYG{l+s+s1}{\PYGZsq{}}\PYG{l+s+s1}{Densidade de Probabilidade}\PYG{l+s+s1}{\PYGZsq{}}\PYG{p}{)}

\PYG{c+c1}{\PYGZsh{} Projeção bidimensional no eixo Y}
\PYG{n}{axs}\PYG{p}{[}\PYG{l+m+mi}{2}\PYG{p}{]}\PYG{o}{.}\PYG{n}{plot}\PYG{p}{(}\PYG{n}{np}\PYG{o}{.}\PYG{n}{sum}\PYG{p}{(}\PYG{n}{z}\PYG{p}{,} \PYG{n}{axis}\PYG{o}{=}\PYG{l+m+mi}{1}\PYG{p}{)}\PYG{p}{,} \PYG{n}{y}\PYG{p}{[}\PYG{p}{:}\PYG{p}{,} \PYG{l+m+mi}{0}\PYG{p}{]}\PYG{p}{,} \PYG{n}{color}\PYG{o}{=}\PYG{l+s+s1}{\PYGZsq{}}\PYG{l+s+s1}{red}\PYG{l+s+s1}{\PYGZsq{}}\PYG{p}{)}
\PYG{n}{axs}\PYG{p}{[}\PYG{l+m+mi}{2}\PYG{p}{]}\PYG{o}{.}\PYG{n}{set\PYGZus{}title}\PYG{p}{(}\PYG{l+s+s1}{\PYGZsq{}}\PYG{l+s+s1}{Projeção no Eixo Y}\PYG{l+s+s1}{\PYGZsq{}}\PYG{p}{)}
\PYG{n}{axs}\PYG{p}{[}\PYG{l+m+mi}{2}\PYG{p}{]}\PYG{o}{.}\PYG{n}{set\PYGZus{}xlabel}\PYG{p}{(}\PYG{l+s+s1}{\PYGZsq{}}\PYG{l+s+s1}{Densidade de Probabilidade}\PYG{l+s+s1}{\PYGZsq{}}\PYG{p}{)}
\PYG{n}{axs}\PYG{p}{[}\PYG{l+m+mi}{2}\PYG{p}{]}\PYG{o}{.}\PYG{n}{set\PYGZus{}ylabel}\PYG{p}{(}\PYG{l+s+s1}{\PYGZsq{}}\PYG{l+s+s1}{Eixo Y}\PYG{l+s+s1}{\PYGZsq{}}\PYG{p}{)}

\PYG{c+c1}{\PYGZsh{} Ajustando o layout}
\PYG{n}{plt}\PYG{o}{.}\PYG{n}{tight\PYGZus{}layout}\PYG{p}{(}\PYG{p}{)}

\PYG{c+c1}{\PYGZsh{} Exibindo os subplots}
\PYG{n}{plt}\PYG{o}{.}\PYG{n}{show}\PYG{p}{(}\PYG{p}{)}
\end{sphinxVerbatim}

\end{sphinxuseclass}\end{sphinxVerbatimInput}
\begin{sphinxVerbatimOutput}

\begin{sphinxuseclass}{cell_output}
\noindent\sphinxincludegraphics{{4be045c59c3ada3704b4f17fdc18379e1c2746c8677777a79abf72a52393aee4}.png}

\end{sphinxuseclass}\end{sphinxVerbatimOutput}

\end{sphinxuseclass}
\sphinxAtStartPar
Esse código possui muitos detalhes matemáticos e de configuração, vamos analisar passo a passo:
\begin{enumerate}
\sphinxsetlistlabels{\arabic}{enumi}{enumii}{}{.}%
\item {} 
\sphinxAtStartPar
\sphinxstylestrong{Definição dos Parâmetros da Distribuição Gaussiana:}
\begin{itemize}
\item {} 
\sphinxAtStartPar
\sphinxcode{\sphinxupquote{mean = {[}0, 0{]}}}: Define a média da distribuição gaussiana bidimensional como (0, 0).

\item {} 
\sphinxAtStartPar
\sphinxcode{\sphinxupquote{covariance = {[}{[}1, 0.5{]}, {[}0.5, 1{]}{]}}}: Define a matriz de covariância da distribuição gaussiana bidimensional.

\end{itemize}

\item {} 
\sphinxAtStartPar
\sphinxstylestrong{Criação da Grade de Valores para x e y:}
\begin{itemize}
\item {} 
\sphinxAtStartPar
\sphinxcode{\sphinxupquote{x = np.linspace(\sphinxhyphen{}3, 3, 100)}}: Cria uma sequência linear de 100 valores entre \sphinxhyphen{}3 e 3 para o eixo x.

\item {} 
\sphinxAtStartPar
\sphinxcode{\sphinxupquote{y = np.linspace(\sphinxhyphen{}3, 3, 100)}}: Cria uma sequência linear de 100 valores entre \sphinxhyphen{}3 e 3 para o eixo y.

\item {} 
\sphinxAtStartPar
\sphinxcode{\sphinxupquote{x, y = np.meshgrid(x, y)}}: Cria uma grade bidimensional de valores combinando os valores de x e y.

\end{itemize}

\item {} 
\sphinxAtStartPar
\sphinxstylestrong{Cálculo dos Valores da Distribuição Gaussiana Manualmente:}
\begin{itemize}
\item {} 
\sphinxAtStartPar
\sphinxcode{\sphinxupquote{inv\_covariance = np.linalg.inv(covariance)}}: Calcula a matriz inversa da matriz de covariância.

\item {} 
\sphinxAtStartPar
\sphinxcode{\sphinxupquote{det\_covariance = np.linalg.det(covariance)}}: Calcula o determinante da matriz de covariância.

\item {} 
\sphinxAtStartPar
\sphinxcode{\sphinxupquote{exponent = \sphinxhyphen{}0.5 * (x**2 * inv\_covariance{[}0, 0{]} + y**2 * inv\_covariance{[}1, 1{]} + 2 * x * y * inv\_covariance{[}0, 1{]})}}: Calcula o expoente da função gaussiana.

\item {} 
\sphinxAtStartPar
\sphinxcode{\sphinxupquote{z = (1 / (2 * np.pi * np.sqrt(det\_covariance))) * np.exp(exponent)}}: Calcula os valores da distribuição gaussiana bidimensional.

\end{itemize}

\item {} 
\sphinxAtStartPar
\sphinxstylestrong{Criação de Subplots:}
\begin{itemize}
\item {} 
\sphinxAtStartPar
\sphinxcode{\sphinxupquote{fig, axs = plt.subplots(1, 3, figsize=(15, 5))}}: Cria uma figura com uma linha e três colunas de subplots.

\end{itemize}

\item {} 
\sphinxAtStartPar
\sphinxstylestrong{Subplot da Densidade de Probabilidade:}
\begin{itemize}
\item {} 
\sphinxAtStartPar
\sphinxcode{\sphinxupquote{axs{[}0{]}.contourf(x, y, z, cmap='viridis')}}: Plota a densidade de probabilidade utilizando contornos preenchidos.

\item {} 
\sphinxAtStartPar
Configuração de título e rótulos dos eixos.

\end{itemize}

\item {} 
\sphinxAtStartPar
\sphinxstylestrong{Projeção Bidimensional no Eixo X:}
\begin{itemize}
\item {} 
\sphinxAtStartPar
\sphinxcode{\sphinxupquote{axs{[}1{]}.plot(x{[}0, :{]}, np.sum(z, axis=0), color='blue')}}: Plota a projeção da densidade de probabilidade no eixo X.

\item {} 
\sphinxAtStartPar
Configuração de título e rótulos dos eixos.

\end{itemize}

\item {} 
\sphinxAtStartPar
\sphinxstylestrong{Projeção Bidimensional no Eixo Y:}
\begin{itemize}
\item {} 
\sphinxAtStartPar
\sphinxcode{\sphinxupquote{axs{[}2{]}.plot(np.sum(z, axis=1), y{[}:, 0{]}, color='red')}}: Plota a projeção da densidade de probabilidade no eixo Y.

\item {} 
\sphinxAtStartPar
Configuração de título e rótulos dos eixos.

\end{itemize}

\item {} 
\sphinxAtStartPar
\sphinxstylestrong{Ajuste de Layout e Exibição dos Subplots:}
\begin{itemize}
\item {} 
\sphinxAtStartPar
\sphinxcode{\sphinxupquote{plt.tight\_layout()}}: Ajusta automaticamente o layout dos subplots para evitar sobreposição.

\item {} 
\sphinxAtStartPar
\sphinxcode{\sphinxupquote{plt.show()}}: Exibe os subplots.

\end{itemize}

\end{enumerate}

\sphinxAtStartPar
Este código cria uma representação visual completa de uma distribuição gaussiana bidimensional, oferecendo diferentes perspectivas por meio de subplots.


\subsection{Salvando os graficos como figuras}
\label{\detokenize{chapters/ch7/ch7:salvando-os-graficos-como-figuras}}

\section{Exercícios}
\label{\detokenize{chapters/ch7/ch7:exercicios}}
\sphinxAtStartPar
\sphinxstylestrong{Exercícios sobre NumPy e Matplotlib:}
\begin{enumerate}
\sphinxsetlistlabels{\arabic}{enumi}{enumii}{}{.}%
\item {} 
\sphinxAtStartPar
\sphinxstylestrong{Manipulação de Arrays com NumPy:} Crie uma array bidimensional com números de 1 a 9 e, em seguida, uma array unidimensional com cinco elementos, todos iguais a 10. Imprima ambas as arrays.

\begin{sphinxVerbatim}[commandchars=\\\{\}]
\PYG{c+c1}{\PYGZsh{} Teste 1}
\PYG{n}{Entrada}\PYG{p}{:} \PYG{n}{array\PYGZus{}2d} \PYG{o}{=} \PYG{n}{np}\PYG{o}{.}\PYG{n}{array}\PYG{p}{(}\PYG{p}{[}\PYG{p}{[}\PYG{l+m+mi}{1}\PYG{p}{,} \PYG{l+m+mi}{2}\PYG{p}{,} \PYG{l+m+mi}{3}\PYG{p}{]}\PYG{p}{,} \PYG{p}{[}\PYG{l+m+mi}{4}\PYG{p}{,} \PYG{l+m+mi}{5}\PYG{p}{,} \PYG{l+m+mi}{6}\PYG{p}{]}\PYG{p}{,} \PYG{p}{[}\PYG{l+m+mi}{7}\PYG{p}{,} \PYG{l+m+mi}{8}\PYG{p}{,} \PYG{l+m+mi}{9}\PYG{p}{]}\PYG{p}{]}\PYG{p}{)}
\PYG{n}{Saída}\PYG{p}{:}
\PYG{p}{[}\PYG{p}{[}\PYG{l+m+mi}{1} \PYG{l+m+mi}{2} \PYG{l+m+mi}{3}\PYG{p}{]}
 \PYG{p}{[}\PYG{l+m+mi}{4} \PYG{l+m+mi}{5} \PYG{l+m+mi}{6}\PYG{p}{]}
 \PYG{p}{[}\PYG{l+m+mi}{7} \PYG{l+m+mi}{8} \PYG{l+m+mi}{9}\PYG{p}{]}\PYG{p}{]}

\PYG{c+c1}{\PYGZsh{} Teste 2}
\PYG{n}{Entrada}\PYG{p}{:} \PYG{n}{array\PYGZus{}unidimensional} \PYG{o}{=} \PYG{n}{np}\PYG{o}{.}\PYG{n}{full}\PYG{p}{(}\PYG{l+m+mi}{5}\PYG{p}{,} \PYG{l+m+mi}{10}\PYG{p}{)}
\PYG{n}{Saída}\PYG{p}{:} \PYG{p}{[}\PYG{l+m+mi}{10} \PYG{l+m+mi}{10} \PYG{l+m+mi}{10} \PYG{l+m+mi}{10} \PYG{l+m+mi}{10}\PYG{p}{]}
\end{sphinxVerbatim}

\item {} 
\sphinxAtStartPar
\sphinxstylestrong{Operações com Arrays:}
\begin{itemize}
\item {} 
\sphinxAtStartPar
\sphinxstylestrong{Enunciado:} Multiplique a array unidimensional \sphinxcode{\sphinxupquote{{[}1, 2, 3, 4, 5{]}}} por 2 e calcule o produto escalar da array \sphinxcode{\sphinxupquote{{[}2, 4, 6{]}}} com ela mesma. Imprima os resultados.

\item {} 
\sphinxAtStartPar
\sphinxstylestrong{Entrada 1:}

\begin{sphinxVerbatim}[commandchars=\\\{\}]
\PYG{n}{multiplicada\PYGZus{}por\PYGZus{}2} \PYG{o}{=} \PYG{n}{array\PYGZus{}1d} \PYG{o}{*} \PYG{l+m+mi}{2}
\end{sphinxVerbatim}

\sphinxAtStartPar
\sphinxstylestrong{Saída 1:}

\begin{sphinxVerbatim}[commandchars=\\\{\}]
\PYG{n}{Array} \PYG{n}{Multiplicada} \PYG{n}{por} \PYG{l+m+mi}{2}\PYG{p}{:}
\PYG{p}{[} \PYG{l+m+mi}{2}  \PYG{l+m+mi}{4}  \PYG{l+m+mi}{6}  \PYG{l+m+mi}{8} \PYG{l+m+mi}{10}\PYG{p}{]}
\end{sphinxVerbatim}

\item {} 
\sphinxAtStartPar
\sphinxstylestrong{Entrada 2:}

\begin{sphinxVerbatim}[commandchars=\\\{\}]
\PYG{n}{produto\PYGZus{}escalar} \PYG{o}{=} \PYG{n}{np}\PYG{o}{.}\PYG{n}{dot}\PYG{p}{(}\PYG{n}{array\PYGZus{}1d}\PYG{p}{,} \PYG{n}{array\PYGZus{}1d}\PYG{p}{)}
\end{sphinxVerbatim}

\sphinxAtStartPar
\sphinxstylestrong{Saída 2:}

\begin{sphinxVerbatim}[commandchars=\\\{\}]
\PYG{n}{Produto} \PYG{n}{Escalar}\PYG{p}{:}
\PYG{l+m+mi}{55}
\end{sphinxVerbatim}

\end{itemize}

\item {} 
\sphinxAtStartPar
\sphinxstylestrong{Criando Arrays com valores específicos:}
\begin{itemize}
\item {} 
\sphinxAtStartPar
\sphinxstylestrong{Enunciado:} Crie uma array unidimensional com 6 elementos espaçados uniformemente entre 0 e 1 e, em seguida, uma array bidimensional com valores todos iguais a 7. Imprima ambas as arrays.

\item {} 
\sphinxAtStartPar
\sphinxstylestrong{Entrada 1:}

\begin{sphinxVerbatim}[commandchars=\\\{\}]
\PYG{n}{linear\PYGZus{}array} \PYG{o}{=} \PYG{n}{np}\PYG{o}{.}\PYG{n}{linspace}\PYG{p}{(}\PYG{l+m+mi}{0}\PYG{p}{,} \PYG{l+m+mi}{1}\PYG{p}{,} \PYG{l+m+mi}{6}\PYG{p}{)}
\end{sphinxVerbatim}

\sphinxAtStartPar
\sphinxstylestrong{Saída 1:}

\begin{sphinxVerbatim}[commandchars=\\\{\}]
\PYG{n}{Array} \PYG{n}{Unidimensional}\PYG{p}{:}
\PYG{p}{[}\PYG{l+m+mf}{0.}  \PYG{l+m+mf}{0.2} \PYG{l+m+mf}{0.4} \PYG{l+m+mf}{0.6} \PYG{l+m+mf}{0.8} \PYG{l+m+mf}{1.} \PYG{p}{]}
\end{sphinxVerbatim}

\item {} 
\sphinxAtStartPar
\sphinxstylestrong{Entrada 2:}

\begin{sphinxVerbatim}[commandchars=\\\{\}]
\PYG{n}{array\PYGZus{}sevens} \PYG{o}{=} \PYG{n}{np}\PYG{o}{.}\PYG{n}{full}\PYG{p}{(}\PYG{p}{(}\PYG{l+m+mi}{3}\PYG{p}{,} \PYG{l+m+mi}{4}\PYG{p}{)}\PYG{p}{,} \PYG{l+m+mi}{7}\PYG{p}{)}
\end{sphinxVerbatim}

\sphinxAtStartPar
\sphinxstylestrong{Saída 2:}

\begin{sphinxVerbatim}[commandchars=\\\{\}]
\PYG{n}{Array} \PYG{n}{Bidimensional}\PYG{p}{:}
\PYG{p}{[}\PYG{p}{[}\PYG{l+m+mi}{7} \PYG{l+m+mi}{7} \PYG{l+m+mi}{7} \PYG{l+m+mi}{7}\PYG{p}{]}
 \PYG{p}{[}\PYG{l+m+mi}{7} \PYG{l+m+mi}{7} \PYG{l+m+mi}{7} \PYG{l+m+mi}{7}\PYG{p}{]}
 \PYG{p}{[}\PYG{l+m+mi}{7} \PYG{l+m+mi}{7} \PYG{l+m+mi}{7} \PYG{l+m+mi}{7}\PYG{p}{]}\PYG{p}{]}
\end{sphinxVerbatim}

\end{itemize}

\item {} 
\sphinxAtStartPar
\sphinxstylestrong{Concatenando e Empilhando Arrays:}
\begin{itemize}
\item {} 
\sphinxAtStartPar
\sphinxstylestrong{Enunciado:} Concatene as arrays \sphinxcode{\sphinxupquote{{[}1, 2, 3{]}}} e \sphinxcode{\sphinxupquote{{[}4, 5, 6{]}}} e, em seguida, empilhe verticalmente as arrays bidimensionais \sphinxcode{\sphinxupquote{{[}{[}1, 2, 3{]}, {[}4, 5, 6{]}{]}}} e \sphinxcode{\sphinxupquote{{[}{[}7, 8, 9{]}{]}}}. Imprima os resultados.

\item {} 
\sphinxAtStartPar
\sphinxstylestrong{Entrada 1:}

\begin{sphinxVerbatim}[commandchars=\\\{\}]
\PYG{n}{array\PYGZus{}concatenada} \PYG{o}{=} \PYG{n}{np}\PYG{o}{.}\PYG{n}{concatenate}\PYG{p}{(}\PYG{p}{(}\PYG{p}{[}\PYG{l+m+mi}{1}\PYG{p}{,} \PYG{l+m+mi}{2}\PYG{p}{,} \PYG{l+m+mi}{3}\PYG{p}{]}\PYG{p}{,} \PYG{p}{[}\PYG{l+m+mi}{4}\PYG{p}{,} \PYG{l+m+mi}{5}\PYG{p}{,} \PYG{l+m+mi}{6}\PYG{p}{]}\PYG{p}{)}\PYG{p}{)}
\end{sphinxVerbatim}

\sphinxAtStartPar
\sphinxstylestrong{Saída 1:}

\begin{sphinxVerbatim}[commandchars=\\\{\}]
\PYG{n}{Array} \PYG{n}{Concatenada}\PYG{p}{:}
\PYG{p}{[}\PYG{l+m+mi}{1} \PYG{l+m+mi}{2} \PYG{l+m+mi}{3} \PYG{l+m+mi}{4} \PYG{l+m+mi}{5} \PYG{l+m+mi}{6}\PYG{p}{]}
\end{sphinxVerbatim}

\item {} 
\sphinxAtStartPar
\sphinxstylestrong{Entrada 2:}

\begin{sphinxVerbatim}[commandchars=\\\{\}]
\PYG{n}{array\PYGZus{}empilhada\PYGZus{}verticalmente} \PYG{o}{=} \PYG{n}{np}\PYG{o}{.}\PYG{n}{vstack}\PYG{p}{(}\PYG{p}{(}\PYG{p}{[}\PYG{l+m+mi}{1}\PYG{p}{,} \PYG{l+m+mi}{2}\PYG{p}{,} \PYG{l+m+mi}{3}\PYG{p}{]}\PYG{p}{,} \PYG{p}{[}\PYG{l+m+mi}{4}\PYG{p}{,} \PYG{l+m+mi}{5}\PYG{p}{,} \PYG{l+m+mi}{6}\PYG{p}{]}\PYG{p}{,} \PYG{p}{[}\PYG{l+m+mi}{7}\PYG{p}{,} \PYG{l+m+mi}{8}\PYG{p}{,} \PYG{l+m+mi}{9}\PYG{p}{]}\PYG{p}{)}\PYG{p}{)}
\end{sphinxVerbatim}

\sphinxAtStartPar
\sphinxstylestrong{Saída 2:}

\begin{sphinxVerbatim}[commandchars=\\\{\}]
\PYG{n}{Array} \PYG{n}{Empilhada} \PYG{n}{Verticalmente}\PYG{p}{:}
\PYG{p}{[}\PYG{p}{[}\PYG{l+m+mi}{1} \PYG{l+m+mi}{2} \PYG{l+m+mi}{3}\PYG{p}{]}
 \PYG{p}{[}\PYG{l+m+mi}{4} \PYG{l+m+mi}{5} \PYG{l+m+mi}{6}\PYG{p}{]}
 \PYG{p}{[}\PYG{l+m+mi}{7} \PYG{l+m+mi}{8} \PYG{l+m+mi}{9}\PYG{p}{]}\PYG{p}{]}
\end{sphinxVerbatim}

\end{itemize}

\item {} 
\sphinxAtStartPar
\sphinxstylestrong{Reshape e Transposição:}
\begin{itemize}
\item {} 
\sphinxAtStartPar
\sphinxstylestrong{Enunciado:} Reshape a array unidimensional \sphinxcode{\sphinxupquote{{[}1, 2, 3, 4, 5{]}}} para uma matriz com uma coluna e cinco linhas e, em seguida, transponha a array bidimensional \sphinxcode{\sphinxupquote{{[}{[}1, 2, 3{]}, {[}4, 5, 6{]}{]}}}. Imprima os resultados.

\item {} 
\sphinxAtStartPar
\sphinxstylestrong{Entrada 1:}

\begin{sphinxVerbatim}[commandchars=\\\{\}]
\PYG{n}{array\PYGZus{}reshaped} \PYG{o}{=} \PYG{n}{array\PYGZus{}1d}\PYG{o}{.}\PYG{n}{reshape}\PYG{p}{(}\PYG{p}{(}\PYG{l+m+mi}{5}\PYG{p}{,} \PYG{l+m+mi}{1}\PYG{p}{)}\PYG{p}{)}
\end{sphinxVerbatim}

\sphinxAtStartPar
\sphinxstylestrong{Saída 1:}

\begin{sphinxVerbatim}[commandchars=\\\{\}]
\PYG{n}{Array} \PYG{n}{Reshaped}\PYG{p}{:}
\PYG{p}{[}\PYG{p}{[}\PYG{l+m+mi}{1}\PYG{p}{]}
 \PYG{p}{[}\PYG{l+m+mi}{2}\PYG{p}{]}
 \PYG{p}{[}\PYG{l+m+mi}{3}\PYG{p}{]}
 \PYG{p}{[}\PYG{l+m+mi}{4}\PYG{p}{]}
 \PYG{p}{[}\PYG{l+m+mi}{5}\PYG{p}{]}\PYG{p}{]}
\end{sphinxVerbatim}

\item {} 
\sphinxAtStartPar
\sphinxstylestrong{Entrada 2:}

\begin{sphinxVerbatim}[commandchars=\\\{\}]
\PYG{n}{array\PYGZus{}transposta} \PYG{o}{=} \PYG{n}{array\PYGZus{}2d}\PYG{o}{.}\PYG{n}{T}
\end{sphinxVerbatim}

\sphinxAtStartPar
\sphinxstylestrong{Saída 2:}

\begin{sphinxVerbatim}[commandchars=\\\{\}]
\PYG{n}{Array} \PYG{n}{Transposta}\PYG{p}{:}
\PYG{p}{[}\PYG{p}{[}\PYG{l+m+mi}{1} \PYG{l+m+mi}{4}\PYG{p}{]}
 \PYG{p}{[}\PYG{l+m+mi}{2} \PYG{l+m+mi}{5}\PYG{p}{]}
 \PYG{p}{[}\PYG{l+m+mi}{3} \PYG{l+m+mi}{6}\PYG{p}{]}\PYG{p}{]}
\end{sphinxVerbatim}

\end{itemize}

\item {} 
\sphinxAtStartPar
\sphinxstylestrong{Operações Estatísticas:}
\begin{itemize}
\item {} 
\sphinxAtStartPar
\sphinxstylestrong{Enunciado:} Calcule a média e o desvio padrão da array \sphinxcode{\sphinxupquote{{[}10, 20, 30, 40, 50{]}}} e encontre o valor máximo ao longo das colunas da array bidimensional \sphinxcode{\sphinxupquote{{[}{[}5, 8, 2{]}, {[}1, 6, 9{]}{]}}}. Imprima os resultados.

\item {} 
\sphinxAtStartPar
\sphinxstylestrong{Entrada 1:}

\begin{sphinxVerbatim}[commandchars=\\\{\}]
\PYG{n}{media\PYGZus{}array} \PYG{o}{=} \PYG{n}{np}\PYG{o}{.}\PYG{n}{mean}\PYG{p}{(}\PYG{n}{array\PYGZus{}valores}\PYG{p}{)}
\PYG{n}{desvio\PYGZus{}padrao\PYGZus{}array} \PYG{o}{=} \PYG{n}{np}\PYG{o}{.}\PYG{n}{std}\PYG{p}{(}\PYG{n}{array\PYGZus{}valores}\PYG{p}{)}
\end{sphinxVerbatim}

\sphinxAtStartPar
\sphinxstylestrong{Saída 1:}

\begin{sphinxVerbatim}[commandchars=\\\{\}]
\PYG{n}{Média}\PYG{p}{:} \PYG{l+m+mf}{30.0}
\PYG{n}{Desvio} \PYG{n}{Padrão}\PYG{p}{:} \PYG{l+m+mf}{14.1421356237}
\end{sphinxVerbatim}

\item {} 
\sphinxAtStartPar
\sphinxstylestrong{Entrada 2:}

\begin{sphinxVerbatim}[commandchars=\\\{\}]
\PYG{n}{max\PYGZus{}value\PYGZus{}colunas} \PYG{o}{=} \PYG{n}{np}\PYG{o}{.}\PYG{n}{max}\PYG{p}{(}\PYG{n}{array\PYGZus{}2d}\PYG{p}{,} \PYG{n}{axis}\PYG{o}{=}\PYG{l+m+mi}{0}\PYG{p}{)}
\end{sphinxVerbatim}

\sphinxAtStartPar
\sphinxstylestrong{Saída 2:}

\begin{sphinxVerbatim}[commandchars=\\\{\}]
\PYG{n}{Valor} \PYG{n}{Máximo} \PYG{n}{por} \PYG{n}{Coluna}\PYG{p}{:}
\PYG{p}{[}\PYG{l+m+mi}{5} \PYG{l+m+mi}{8} \PYG{l+m+mi}{9}\PYG{p}{]}
\end{sphinxVerbatim}

\end{itemize}

\item {} 
\sphinxAtStartPar
\sphinxstylestrong{Indexação e Fatiamento:}
\begin{itemize}
\item {} 
\sphinxAtStartPar
\sphinxstylestrong{Enunciado:} Acesse o terceiro elemento da array unidimensional \sphinxcode{\sphinxupquote{{[}\sphinxhyphen{}1, \sphinxhyphen{}2, \sphinxhyphen{}3, \sphinxhyphen{}4, \sphinxhyphen{}5{]}}} e faça um slice da array unidimensional \sphinxcode{\sphinxupquote{{[}\sphinxhyphen{}10, \sphinxhyphen{}20, \sphinxhyphen{}30, \sphinxhyphen{}40, \sphinxhyphen{}50{]}}} que inclua os elementos do segundo ao quarto. Imprima os resultados.

\item {} 
\sphinxAtStartPar
\sphinxstylestrong{Entrada 1:}

\begin{sphinxVerbatim}[commandchars=\\\{\}]
\PYG{n}{terceiro\PYGZus{}elemento} \PYG{o}{=} \PYG{n}{array\PYGZus{}indexacao}\PYG{p}{[}\PYG{l+m+mi}{2}\PYG{p}{]}
\end{sphinxVerbatim}

\sphinxAtStartPar
\sphinxstylestrong{Saída 1:}

\begin{sphinxVerbatim}[commandchars=\\\{\}]
\PYG{n}{Terceiro} \PYG{n}{Elemento}\PYG{p}{:} \PYG{o}{\PYGZhy{}}\PYG{l+m+mi}{3}
\end{sphinxVerbatim}

\item {} 
\sphinxAtStartPar
\sphinxstylestrong{Entrada 2:}

\begin{sphinxVerbatim}[commandchars=\\\{\}]
\PYG{n}{sliced\PYGZus{}array} \PYG{o}{=} \PYG{n}{array\PYGZus{}fatiamento}\PYG{p}{[}\PYG{l+m+mi}{1}\PYG{p}{:}\PYG{l+m+mi}{4}\PYG{p}{]}


\end{sphinxVerbatim}

\sphinxAtStartPar
\sphinxstylestrong{Saída 2:}

\begin{sphinxVerbatim}[commandchars=\\\{\}]
\PYG{n}{Array} \PYG{n}{Fatiada}\PYG{p}{:}
\PYG{p}{[}\PYG{o}{\PYGZhy{}}\PYG{l+m+mi}{20} \PYG{o}{\PYGZhy{}}\PYG{l+m+mi}{30} \PYG{o}{\PYGZhy{}}\PYG{l+m+mi}{40}\PYG{p}{]}
\end{sphinxVerbatim}

\end{itemize}

\item {} 
\sphinxAtStartPar
\sphinxstylestrong{Operações Lógicas:}
\begin{itemize}
\item {} 
\sphinxAtStartPar
\sphinxstylestrong{Enunciado:} Filtrar os elementos maiores que 5 na array \sphinxcode{\sphinxupquote{{[}3, 8, 1, 6, 2{]}}} e substituir os valores maiores que 10 na array \sphinxcode{\sphinxupquote{{[}7, 15, 2, 11, 9{]}}} por 0. Imprima os resultados.

\item {} 
\sphinxAtStartPar
\sphinxstylestrong{Entrada 1:}

\begin{sphinxVerbatim}[commandchars=\\\{\}]
\PYG{n}{array\PYGZus{}filtrada} \PYG{o}{=} \PYG{n}{array\PYGZus{}logica}\PYG{p}{[}\PYG{n}{array\PYGZus{}logica} \PYG{o}{\PYGZgt{}} \PYG{l+m+mi}{5}\PYG{p}{]}
\end{sphinxVerbatim}

\sphinxAtStartPar
\sphinxstylestrong{Saída 1:}

\begin{sphinxVerbatim}[commandchars=\\\{\}]
\PYG{n}{Array} \PYG{n}{Filtrada}\PYG{p}{:}
\PYG{p}{[} \PYG{l+m+mi}{8}  \PYG{l+m+mi}{6}\PYG{p}{]}
\end{sphinxVerbatim}

\item {} 
\sphinxAtStartPar
\sphinxstylestrong{Entrada 2:}

\begin{sphinxVerbatim}[commandchars=\\\{\}]
\PYG{n}{array\PYGZus{}substituida} \PYG{o}{=} \PYG{n}{np}\PYG{o}{.}\PYG{n}{where}\PYG{p}{(}\PYG{n}{array\PYGZus{}substituicao} \PYG{o}{\PYGZgt{}} \PYG{l+m+mi}{10}\PYG{p}{,} \PYG{l+m+mi}{0}\PYG{p}{,} \PYG{n}{array\PYGZus{}substituicao}\PYG{p}{)}
\end{sphinxVerbatim}

\sphinxAtStartPar
\sphinxstylestrong{Saída 2:}

\begin{sphinxVerbatim}[commandchars=\\\{\}]
\PYG{n}{Array} \PYG{n}{Substituída}\PYG{p}{:}
\PYG{p}{[}\PYG{l+m+mi}{7} \PYG{l+m+mi}{0} \PYG{l+m+mi}{2} \PYG{l+m+mi}{0} \PYG{l+m+mi}{9}\PYG{p}{]}
\end{sphinxVerbatim}

\end{itemize}

\item {} 
\sphinxAtStartPar
\sphinxstylestrong{Visualização de Dados com Matplotlib:}
\begin{itemize}
\item {} 
\sphinxAtStartPar
\sphinxstylestrong{Enunciado:} Crie um gráfico de linha simples com os pontos \sphinxcode{\sphinxupquote{(1, 2), (2, 4), (3, 6), (4, 8), (5, 10)}} e, em seguida, crie um gráfico de barras representando a distribuição de pontos \sphinxcode{\sphinxupquote{(1, 3), (2, 5), (3, 2), (4, 7), (5, 4)}}. Exiba ambos os gráficos.

\item {} 
\sphinxAtStartPar
\sphinxstylestrong{Entrada 1:}

\begin{sphinxVerbatim}[commandchars=\\\{\}]
\PYG{n}{plt}\PYG{o}{.}\PYG{n}{plot}\PYG{p}{(}\PYG{p}{[}\PYG{l+m+mi}{1}\PYG{p}{,} \PYG{l+m+mi}{2}\PYG{p}{,} \PYG{l+m+mi}{3}\PYG{p}{,} \PYG{l+m+mi}{4}\PYG{p}{,} \PYG{l+m+mi}{5}\PYG{p}{]}\PYG{p}{,} \PYG{p}{[}\PYG{l+m+mi}{2}\PYG{p}{,} \PYG{l+m+mi}{4}\PYG{p}{,} \PYG{l+m+mi}{6}\PYG{p}{,} \PYG{l+m+mi}{8}\PYG{p}{,} \PYG{l+m+mi}{10}\PYG{p}{]}\PYG{p}{,} \PYG{n}{label}\PYG{o}{=}\PYG{l+s+s1}{\PYGZsq{}}\PYG{l+s+s1}{Gráfico de Linha}\PYG{l+s+s1}{\PYGZsq{}}\PYG{p}{)}
\PYG{n}{plt}\PYG{o}{.}\PYG{n}{legend}\PYG{p}{(}\PYG{p}{)}
\PYG{n}{plt}\PYG{o}{.}\PYG{n}{show}\PYG{p}{(}\PYG{p}{)}
\end{sphinxVerbatim}

\sphinxAtStartPar
\sphinxstylestrong{Saída 1:}
\sphinxstyleemphasis{(Exibe o gráfico de linha)}

\item {} 
\sphinxAtStartPar
\sphinxstylestrong{Entrada 2:}

\begin{sphinxVerbatim}[commandchars=\\\{\}]
\PYG{n}{plt}\PYG{o}{.}\PYG{n}{bar}\PYG{p}{(}\PYG{p}{[}\PYG{l+m+mi}{1}\PYG{p}{,} \PYG{l+m+mi}{2}\PYG{p}{,} \PYG{l+m+mi}{3}\PYG{p}{,} \PYG{l+m+mi}{4}\PYG{p}{,} \PYG{l+m+mi}{5}\PYG{p}{]}\PYG{p}{,} \PYG{p}{[}\PYG{l+m+mi}{3}\PYG{p}{,} \PYG{l+m+mi}{5}\PYG{p}{,} \PYG{l+m+mi}{2}\PYG{p}{,} \PYG{l+m+mi}{7}\PYG{p}{,} \PYG{l+m+mi}{4}\PYG{p}{]}\PYG{p}{,} \PYG{n}{label}\PYG{o}{=}\PYG{l+s+s1}{\PYGZsq{}}\PYG{l+s+s1}{Gráfico de Barras}\PYG{l+s+s1}{\PYGZsq{}}\PYG{p}{)}
\PYG{n}{plt}\PYG{o}{.}\PYG{n}{legend}\PYG{p}{(}\PYG{p}{)}
\PYG{n}{plt}\PYG{o}{.}\PYG{n}{show}\PYG{p}{(}\PYG{p}{)}
\end{sphinxVerbatim}

\sphinxAtStartPar
\sphinxstylestrong{Saída 2:}
\sphinxstyleemphasis{(Exibe o gráfico de barras)}

\end{itemize}

\item {} 
\sphinxAtStartPar
\sphinxstylestrong{Abrindo e Visualizando Imagens:}
\begin{itemize}
\item {} 
\sphinxAtStartPar
\sphinxstylestrong{Enunciado:} Abra e exiba uma imagem de sua escolha e, em seguida, abra e exiba duas imagens diferentes lado a lado (pode ser a mesma imagem duplicada). Certifique\sphinxhyphen{}se de ter a biblioteca Matplotlib instalada.

\item {} 
\sphinxAtStartPar
\sphinxstylestrong{Entrada 1:}

\begin{sphinxVerbatim}[commandchars=\\\{\}]
\PYG{n}{imagem} \PYG{o}{=} \PYG{n}{plt}\PYG{o}{.}\PYG{n}{imread}\PYG{p}{(}\PYG{l+s+s1}{\PYGZsq{}}\PYG{l+s+s1}{caminho/para/sua/imagem.jpg}\PYG{l+s+s1}{\PYGZsq{}}\PYG{p}{)}
\PYG{n}{plt}\PYG{o}{.}\PYG{n}{imshow}\PYG{p}{(}\PYG{n}{imagem}\PYG{p}{)}
\PYG{n}{plt}\PYG{o}{.}\PYG{n}{show}\PYG{p}{(}\PYG{p}{)}
\end{sphinxVerbatim}

\sphinxAtStartPar
\sphinxstylestrong{Saída 1:}
\sphinxstyleemphasis{(Exibe a imagem)}

\item {} 
\sphinxAtStartPar
\sphinxstylestrong{Entrada 2:}

\begin{sphinxVerbatim}[commandchars=\\\{\}]
\PYG{n}{imagem1} \PYG{o}{=} \PYG{n}{plt}\PYG{o}{.}\PYG{n}{imread}\PYG{p}{(}\PYG{l+s+s1}{\PYGZsq{}}\PYG{l+s+s1}{caminho/para/outra/imagem1.jpg}\PYG{l+s+s1}{\PYGZsq{}}\PYG{p}{)}
\PYG{n}{imagem2} \PYG{o}{=} \PYG{n}{plt}\PYG{o}{.}\PYG{n}{imread}\PYG{p}{(}\PYG{l+s+s1}{\PYGZsq{}}\PYG{l+s+s1}{caminho/para/outra/imagem2.jpg}\PYG{l+s+s1}{\PYGZsq{}}\PYG{p}{)}

\PYG{n}{plt}\PYG{o}{.}\PYG{n}{subplot}\PYG{p}{(}\PYG{l+m+mi}{1}\PYG{p}{,} \PYG{l+m+mi}{2}\PYG{p}{,} \PYG{l+m+mi}{1}\PYG{p}{)}
\PYG{n}{plt}\PYG{o}{.}\PYG{n}{imshow}\PYG{p}{(}\PYG{n}{imagem1}\PYG{p}{)}

\PYG{n}{plt}\PYG{o}{.}\PYG{n}{subplot}\PYG{p}{(}\PYG{l+m+mi}{1}\PYG{p}{,} \PYG{l+m+mi}{2}\PYG{p}{,} \PYG{l+m+mi}{2}\PYG{p}{)}
\PYG{n}{plt}\PYG{o}{.}\PYG{n}{imshow}\PYG{p}{(}\PYG{n}{imagem2}\PYG{p}{)}

\PYG{n}{plt}\PYG{o}{.}\PYG{n}{show}\PYG{p}{(}\PYG{p}{)}
\end{sphinxVerbatim}

\sphinxAtStartPar
\sphinxstylestrong{Saída 2:}
\sphinxstyleemphasis{(Exibe as duas imagens lado a lado)}

\end{itemize}

\end{enumerate}

\sphinxstepscope


\chapter{Capítulo 8: Manipulação de Arquivos}
\label{\detokenize{chapters/ch8/ch8:capitulo-8-manipulacao-de-arquivos}}\label{\detokenize{chapters/ch8/ch8::doc}}

\section{Manipulação de Arquivos}
\label{\detokenize{chapters/ch8/ch8:manipulacao-de-arquivos}}

\section{Exercícios}
\label{\detokenize{chapters/ch8/ch8:exercicios}}






\renewcommand{\indexname}{Index}
\printindex
\end{document}